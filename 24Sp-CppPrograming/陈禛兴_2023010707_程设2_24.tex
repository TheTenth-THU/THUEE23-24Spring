\documentclass[10pt, a4paper, oneside, fontset=none]{ctexart}

%调用宏包
\usepackage{amsmath, amsthm, amssymb, graphicx, wrapfig, mathrsfs}
\usepackage[bookmarks=true, colorlinks, citecolor=blue, linkcolor=black]{hyperref}
\usepackage{fancyhdr}
\usepackage{color, framed, geometry, tcolorbox, nicematrix, xcolor}
\tcbuselibrary{breakable}%box跨页
\tcbuselibrary{skins}%box跨页不留边
\usepackage{makecell}
\usepackage{booktabs}
\usepackage{listings}
\usepackage{float}
\usepackage[labelfont=bf, justification=raggedleft]{caption}
\usepackage{multicol}
\usepackage{extarrows}
\usepackage{enumitem}
\usepackage{yhmath}
\usepackage[text=\includegraphics{C:/Users/16870/.vscode/LaTeX_Application/tex/23Au/图标简稿.png},angle=0]{draftwatermark}%水印
%\usepackage{tikz}

%基本字体设置
\usepackage[math-style=ISO, bold-style=ISO]{unicode-math}
%\newcommand{\symbf}[1]{\boldsymbol{#1}}
%\newcommand{\symup}[1]{\mathrm{#1}}
\setmonofont{Iosevka}
\newfontfamily{\mono}{Cascadia Mono Regular}
\setCJKmainfont{FZXSSK.TTF}[BoldFont={SourceHanSerifCN-Bold.otf}, ItalicFont={FZXKTK.TTF}, BoldItalicFont={汉仪颜楷W.ttf}]
\setCJKsansfont{汉仪文黑-45W.ttf}[BoldFont={汉仪文黑-75W.ttf}, ItalicFont={FZYanZQKSJF.TTF}]
\setCJKmonofont{LXGWNeoXiHei.ttf}
%附加字体设置
\newCJKfontfamily{\kaico}{可口可乐在乎体 楷体Coca-ColaCareFontKaiTi.TTF}
\newCJKfontfamily{\kai}{FZXKTK.TTF}[BoldFont={汉仪颜楷W.ttf}, ItalicFont={方正清刻本悦宋 简繁.TTF}, BoldItalicFont={FZYanZQKSJF.TTF}]
\newCJKfontfamily{\yan}{方正清刻本悦宋 简繁.TTF}[ItalicFont={FZYanZQKSJF.TTF}]
\newCJKfontfamily{\xiu}{方正宋刻本秀楷_GBK.TTF}[ItalicFont={方正宋刻本秀楷_GBK.TTF}, BoldFont={FZYanZQKSJF.TTF}]
\newCJKfontfamily{\run}{汉仪润圆-45W.ttf}[BoldFont={汉仪润圆-75W.ttf}, ItalicFont={汉仪润圆-45W.ttf}]
\newCJKfontfamily{\wen}{汉仪文黑-45W.ttf}[BoldFont={汉仪文黑-75W.ttf}, ItalicFont={hk4e_zh-cn.ttf}]
%文档格式
\geometry{left=1.96cm, right=1.96cm, top=3.18cm, bottom=3.18cm}
\linespread{1.4}
\newcommand{\Section}[1]{ \refstepcounter{section} \section*{*\thesection\texorpdfstring{\quad}{} #1} \addcontentsline{toc}{section}{\makebox[0pt][r]{*}\thesection\texorpdfstring{\quad}{} #1} }
\newcommand{\Subsection}[1]{ \refstepcounter{subsection} \subsection*{*\thesubsection\texorpdfstring{\quad}{} #1} \addcontentsline{toc}{subsection}{\makebox[0pt][r]{*}\thesubsection\texorpdfstring{\quad}{} #1} }
\newcommand{\Subsubsection}[1]{ \refstepcounter{subsubsection} \subsubsection*{*\thesubsubsection\texorpdfstring{\quad}{} #1} \addcontentsline{toc}{subsubsection}{\makebox[0pt][r]{*}\thesubsubsection\texorpdfstring{\quad}{} #1} }
\setlist[itemize]{leftmargin=3em, labelsep=0.25em, itemindent=0em, itemsep=0pt, parsep=0pt, topsep=3pt, partopsep=0pt}
\setlength{\columnsep}{1cm} % 分栏间距
\setlength{\columnseprule}{.4pt} % 栏分割线粗
\def\columnseprulecolor{\color{white!50!black}}

%定理环境
\theoremstyle{plain}
\newtheorem{theorem}{定理}[subsection]
\newtheorem{definition}{定义}[section]
\newtheorem{lemma}[theorem]{引理}
\newtheorem{corollary}[theorem]{推论}
\newtheorem{proposition}[theorem]{命题}

\theoremstyle{definition}
\newtheorem{example}[theorem]{例}
\newtheorem{circum}[theorem]{情形}

\newenvironment{proofs}[1][\small\proofname]{\begin{pf}[breakable, enhanced jigsaw]\begin{proof}[#1]\small\kai}{\end{proof}\end{pf}}
%\newenvironment{solution}{\begin{proofs}[\small\textit{\yan 解}]\small\renewcommand{\qedsymbol}{$\circledS$}}{\end{proofs}}
\renewcommand{\proofname}{\yan{证明}}

\newenvironment{twoc}{\vspace{-11pt}\begin{multicols}{2}}{\end{multicols}\vspace{-9pt}}
%颜色命名
\definecolor{meihong}{rgb}{0.85,0.2,0.47}
\definecolor{bali}{rgb}{0.2,0.6,0.78}
\definecolor{qinglv}{rgb}{0,0.35,0.32}
\newcommand{\colors}[1]{\color{#1!75!black}}
%tbox环境
\newtcolorbox{pr}[2][]{colback=black!5!white,colframe=white!75!black,fonttitle=\sffamily\wen\bfseries,title=#2,#1}
\newtcolorbox[use counter=definition,number within=subsection]{defi}[2][]{colback=bali!5!white,colframe=bali!75!black,fonttitle=\sffamily\wen\bfseries,title=定义~\thetcbcounter. #2,#1}
\newtcolorbox[auto counter,number within=section]{compl}[2][]{colback=bali!5!white,colframe=bali!65!black,fonttitle=\sffamily\wen\bfseries,label=#2,title=电路部件~\thetcbcounter. #2,#1, fontupper=\kai, fontlower=\kai}
\newtcolorbox[use counter=theorem,number within=subsection]{theo}[2][]{colback=meihong!5!white,colframe=meihong!75!black,fonttitle=\sffamily\wen\bfseries,fontupper=\run,title=结论~\thetcbcounter. #2,#1}
\newtcolorbox[use counter=definition,number within=subsection]{defil}[2][]{colback=bali!5!white,colframe=bali!75!black,fonttitle=\sffamily\wen\bfseries,label=#2,title=定义~\thetcbcounter. #2,#1}
\newtcolorbox[use counter=theorem,number within=subsection]{theol}[2][]{colback=meihong!5!white,colframe=meihong!75!black,fonttitle=\sffamily\wen\bfseries,fontupper=\run,label=#2,title=结论~\thetcbcounter. #2,#1}
\newtcolorbox[auto counter,number within=section]{note}[2][]{colback=qinglv!5!white,colframe=qinglv!75!black,breakable, enhanced jigsaw,fonttitle=\sffamily\wen\bfseries,title=注~\thetcbcounter. #2,#1}
\newtcolorbox{prenote}[2][]{colback=gray!5!white,colframe=gray!50!black,breakable, enhanced jigsaw, fonttitle=\sffamily\wen\bfseries,fontupper=\small\kai,title=#2,#1}
\newtcolorbox{pf}[1][]{colback=black!5!white,colframe=white!75!black, fontupper=\small\kai,#1}
%\newcommand{\mybox}[1]{\tikz[baseline=(MeNode.base)]{\node[rounded corners, fill=gray!20](MeNode){#1};}}

%代码环境
\lstset{
	xleftmargin=3em, xrightmargin=1em, 
	breaklines,                                 % 自动将长的代码行换行排版
	extendedchars=false,                        % 解决代码跨页时,章节标题,页眉等汉字不显示的问题
	rulecolor=\color[rgb]{0.8,0.8,0.8},         % 设置代码框颜色
	backgroundcolor=\color[rgb]{0.96,0.96,0.96},% 背景颜色
	keywordstyle=\colors{bali}\bfseries,        % 关键字颜色
	identifierstyle=\color{black},              % 普通标识符颜色
	commentstyle=\rm\kai\color{gray},          	% 注释颜色
	stringstyle=\colors{meihong},      			% 字符串颜色
	showstringspaces=false,                     % 不显示字符串内的空格
	numbers=none, numberstyle=\tiny\tt,         % 显示行号
	basicstyle=\linespread{1}\tt,               % 设置基本字体
	captionpos=t,                               % title在上方(在bottom即为b)
	frame=single,                               % 设置代码框形式
	showtabs=false, 
	numbersep=15pt, 
	framesep=1em, 
	language=C++, 
	columns=fixed, basewidth=0.5em, 
	tabsize=4, 
	gobble=4,
	aboveskip=8pt, belowskip=8pt, 
}
\lstdefinestyle{answer}
	{rulecolor=\color{bali!75!black}, backgroundcolor=\color{bali!15!white}, keywordstyle=\color{bali!50!black}\bfseries, morecomment=[s][\tt\bfseries\colors{bali}]{<-}{->}, }
\lstdefinestyle{output}
	{rulecolor=\color{white!75!black}, backgroundcolor=\color{white!5!black}, language={}, basicstyle=\linespread{1}\color{white}\mono\wen, identifierstyle=\linespread{1}\color{white}\mono, keywordstyle=\color{bali!25!white}\bfseries, morecomment=[s][\tt\bfseries\colors{bali}]{<-}{->}, morecomment=[l][\tt\color{white!80!black}]{//}, showspaces=false, basewidth=0.6em}

%定义格式记号
\newcommand{\hang}[1][1]{\hangafter 1 \hangindent #1em}
\newcommand{\den}[2][]{\begin{defi}{#1}\kai #2\end{defi}}
\newcommand{\din}[2][]{\begin{theo}{#1}\run #2\end{theo}}
\newcommand{\de}[2][]{\begin{defil}{#1}\kai #2\end{defil}}
\newcommand{\di}[2][]{\begin{theol}{#1}\run #2\end{theol}}
\newcommand{\dep}[3][]{\begin{defi}{#1\page{#2}}\kai #3\end{defi}}
\newcommand{\dip}[3][]{\begin{theo}{#1\page{#2}}\run #3\end{theo}}
\newcommand{\zhu}[2][]{\begin{note}{#1}\xiu #2\end{note}}
\newcommand{\tboba}[1]{\textbf{\kai\color{bali!75!black}#1}}
\newcommand{\mboba}[1]{\kai\symbf{\color{bali!75!black}#1}}
\newcommand{\tbome}[1]{\textbf{\run\color{meihong!75!black}#1}}
\newcommand{\mbome}[1]{\run\symbf{\color{meihong!75!black}#1}}
\newcommand{\prtsc}[2][14]{\begin{center}
	%\vspace{-1em}
	\includegraphics[width=#1cm]{C:/Users/16870/.vscode/LaTeX_Application/tex/24Sp/24Sp-CppPrograming/作业结果截图/#2.jpg}
	\vspace{-1em}
\end{center}}

%定义算符
\def\upint{\mathchoice%
	{\mkern13mu\overline{\vphantom{\intop}\mkern7mu}\mkern-20mu}%
	{\mkern7mu\overline{\vphantom{\intop}\mkern7mu}\mkern-14mu}%
	{\mkern7mu\overline{\vphantom{\intop}\mkern7mu}\mkern-14mu}%
	{\mkern7mu\overline{\vphantom{\intop}\mkern7mu}\mkern-14mu}%
  \int}
\def\lowint{\mkern3mu\underline{\vphantom{\intop}\mkern7mu}\mkern-10mu\int}
\newcommand*{\dif}{\mathop{}\!\symrm{d}}
\newcommand*{\e}{\symrm{e}}
\renewcommand{\i}{\symrm{i}}
\newcommand*{\R}{\mathbb{R}}
\newcommand{\dint}[1][]{\displaystyle{\int #1}}
\newcommand*{\page}[1]{\hfill P$_\text{#1}$}

\renewcommand{\rightmark}{陈禛兴 2023010707}

%标题、作者、日期
\title
{
	\textbf{计算机程序设计基础(C++)}\textit{作业}
}
\author{\zihao{5} T$^\text{T}$T}
\date{\zihao{5}\kai \today}
%----------------------------------------------------------
\begin{document}
\lineskip=0pt
\lineskiplimit=0pt
\ctexset{section={format={\Large\bfseries}}}
\renewcommand{\thesection}{第\arabic{section}题}
\renewcommand{\thesubsection}{解答}

\noindent\textbf{\Large\wen 【必做题】}

\section{}
设计一个函数模板,其中包括数据成员\texttt{T a[n]}以及对其进行排序的成员函数\texttt{sort()},模板参数\texttt{T}可实例化成字符串。

\subsection{}
\kai

若将题中「模板参数\texttt{T}可实例化成字符串」理解为\texttt{T}实例化为\textbf{字符串类\texttt{string}},由于\texttt{string}类带有关系运算符「\texttt{<}」「\texttt{>}」的重载,函数模板可以设计为:
\begin{lstlisting}[style=answer]
	#include <iostream>
	#include <string>
	using namespace std;
	
	template <class T>
	void sort(T* a, int N)
	{
		for (int i = 0; i < N - 1; i++)
		{
			for (int j = i + 1; j < N; j++)
			{
				if (a[i] > a[j])
				{
					T temp = a[i];
					a[i] = a[j];
					a[j] = temp;
				}
			}
		}
	}
\end{lstlisting}
这样,这个函数模板就既可以用于\texttt{string},也可以用于其他众多标准类型数据和重载了关系运算符「\texttt{<}」「\texttt{>}」的数据,适用范围很广。例如这个程序:
\begin{lstlisting}
	int main()
	{
		/*  按字符串类 string 实例化  */
		string strs[4] = { "alpha", "beta", "gamma", "delta" };
		sort(strs, 4);
		for (int i = 0; i < 4; i++)
			cout << strs[i] << " ";
		cout << endl;
		/*  按整型数 int 实例化  */
		int nums[5] = { 114, 514, 19, 181, 10 };
		sort(nums, 5);
		for (int i = 0; i < 5; i++)
			cout << nums[i] << " ";
		cout << endl;
		return 0;
	}
\end{lstlisting}
其输出为:
\begin{lstlisting}[style=output]
	alpha beta delta gamma
	10 19 114 181 514
\end{lstlisting}
\prtsc{12-01-1}

若将「模板参数\texttt{T}可实例化成字符串」理解为\texttt{T}实例化为\textbf{字符指针\texttt{char *}},即\texttt{T a[n]}实例化为字符串(的首地址)的数组,则这个函数模板设计为:
\begin{lstlisting}[style=answer]
	#include <iostream>
	using namespace std;
	
	template <class T>
	void sort(T* a, int N)
	{
		for (int i = 0; i < N - 1; i++)
		{
			for (int j = i + 1; j < N; j++)
			{
				int k = 0;
				while (a[i][k] == a[j][k] && a[i][k] != '\0' && a[j][k] != '\0')
					k++;
				if (a[i][k] > a[j][k])
				{
					T temp = a[i];
					a[i] = a[j];
					a[j] = temp;
				}
			}
		}
	}	
\end{lstlisting}
这个函数模板泛用性要差得多,只适用于\texttt{a}为可比较数据的二级指针的情况,即数据组的排序。例如:
\begin{lstlisting}
	int main()
	{
		/*  按常字符串 const char* 实例化  */
		const char* strs[4] = { "alpha", "beta", "gamma", "delta" };
		sort(strs, 4);
		for (int i = 0; i < 4; i++)
			cout << strs[i] << " ";
		cout << endl;
		/*  按字符数组 char* 实例化  */
		char** varstrs = new char*[4];
		for (int i = 0; i < 4; i++)
		{
			varstrs[i] = new char[20];
			cin >> varstrs[i];
		}
		sort(varstrs, 4);
		for (int i = 0; i < 4; i++)
			cout << varstrs[i] << " ";
		cout << endl;
		for (int i = 0; i < 4; i++)
			delete[] varstrs[i];
		delete[] varstrs;
		return 0;
	}	
\end{lstlisting}
其输出为:
\begin{lstlisting}[style=output]
	alpha beta delta gamma
	Hello
	MyNameIs
	World
	#include<iostream>
	#include<iostream> Hello MyNameIs World
\end{lstlisting}
\prtsc{12-01-2}

\newpage
%------------------------------------------------------------
\section{}
\normalfont

设计一个类模板,其中包括数据成员\texttt{T a[n]}以及在其中进行查找数据元素的函数\texttt{int search(T)},模板参数\texttt{T}可实例化成字符串。

\subsection{}
\kai

将题中「模板参数\texttt{T}可实例化成字符串」理解为\texttt{T}实例化为\textbf{字符串类\texttt{string}},由于\texttt{string}类带有关系运算符「\texttt{==}」的重载,类模板可以设计为:

\begin{lstlisting}[style=answer]
	#include <iostream>
	#include <string>
	using namespace std;
	
	template <class T, int n>
	class TArray
	{
	private:
		T a[n];
	public:
		TArray()
		{
			for (int i = 0; i < n; i++)
				cin >> a[i];
		}
		int search(T);
	};
	
	template <class T, int n>
	int TArray<T, n>::search(T target)
	{
		int result = -1, count = 0;
		for (int i = 0; i < n; i++)
			if (a[i] == target)
			{
				count++;
				result = i;
			}
		if (count == 1) return result;
		else if (count == 0) return -1; // not found
		else return -2; // multiple found
	}
\end{lstlisting}
同样,这个类模板既可以用于\texttt{string},也可以用于其他众多标准类型数据和重载了关系运算符「\texttt{==}」的数据,适用范围很广。例如:
\begin{lstlisting}
	int main()
	{
		TArray<string, 5> strarr;
		string target;
		cin >> target;
		int result = strarr.search(target);
		if (result == -1) cout << "Target \"" << target <<"\" is not found." << endl;
		else if (result == -2) cout << "Multiple \"" << target << "\" targets is found." << endl;
		else cout << "Target \"" << target << "\" is found at index " << result << "." << endl;
		return 0;
	}
\end{lstlisting}
运行的结果为:
\begin{lstlisting}[style=output]
	hello World TheTenth_THU
	#include <string>
	TheTenth_THU
	Target "TheTenth_THU" is found at index 2.
\end{lstlisting}
\prtsc{12-02}

\normalfont
\newpage
%---------------------------------------------------------
\noindent\textbf{\Large\wen 【选做题】}

\section{}
参考课件中「类模版派生实例代码」,用「模板类派生新的派生类」的方法(即p18第2种派生方式),重新改写下面(p19~21)的~\ref{line: end}~行程序代码:(输出提示语因编译器编码问题尚未解决已改为英文)
\begin{lstlisting}[escapeinside={(*@}{@*)}]
	#include <iostream>
	using namespace std;
	
	/* Node structure */
	class SNode
	{
	public:
		SNode(int value);
		~SNode() {};
		int m_value;
		SNode* m_next;
	};
	
	/* Class template for linked list */
	template <class Type>
	class TList
	{
	public:
		TList();
		~TList();
		// Insert a new node with value
		virtual bool Insert(Type value); 
		// Delete the first node with value
		bool Delete(Type value); 
		// Check if the list contains a node with value
		bool Contain(Type value); 
		// Print all the nodes in the list
		void Print(); 
	protected:
		SNode* m_head;
	};
	
	/* Class template for set */
	template <class Type>
	class TSet : public TList<Type>
	{
	public:
		// Overload for TSet: Insert a new node with value, check if it already exists first
		bool Insert(Type value); 
	};
	
	/* Definition */
	SNode::SNode(int value)
	{
		m_value = value;
		m_next = NULL;
	}
	template <class Type>
	TList<Type>::TList()
	{
		m_head = NULL;
	}
	template <class Type>
	TList<Type>::~TList() // Destructor to delete all nodes still in the list
	{
		SNode* p = m_head;
		for (; p != NULL; )
		{
			m_head = p->m_next;
			delete p;
			p = m_head;
		}
	}
	template <class Type>
	bool TList<Type>::Insert(Type value)
	{
		SNode* pTemp = new SNode(value);
		if (pTemp == NULL) return false; // Memory allocation failed
		pTemp->m_next = m_head;
		m_head = pTemp;
		return true;
	}
	template <class Type>
	bool TList<Type>::Delete(Type value)
	{
		SNode* p1, * p2;
		if (m_head->m_value == value)
		{
			p1 = m_head->m_next;
			delete m_head;
			m_head = p1;
			return true;
		}
		else
		{
			for (p1 = m_head, p2 = m_head->m_next; p2 != NULL; ) // Traverse the entire list
			{
				if (p2->m_value == value)
				{
					p1->m_next = p2->m_next;
					delete p2;
					return true;
				}
				else
				{
					p1 = p1->m_next;
					p2 = p2->m_next;
				}
			}
		}
		return false; // Value not found in the list
	}
	template <class Type>
	bool TList<Type>::Contain(Type value)
	{
		for (SNode* p = m_head; p != NULL; p = p->m_next)
		{
			if (p->m_value == value) return true;
		}
		return false;
	}
	template <class Type>
	void TList<Type>::Print()
	{
		cout << "The values of the nodes are:";
		for (SNode* p = m_head; p != NULL; p = p->m_next)
		{
			cout << " " << p->m_value << "; ";
		}
		cout << endl;
	}
	template <class Type>
	bool TSet<Type>::Insert(Type value)
	{
		if (!(TList<Type>::Contain(value)) && TList<Type>::Insert(value)) 
			return true; // Value not in the list, insert it successfully
		return false; // Value already exists in the list or insert failed for other reasons
	}
	
	// Test code
	void main()
	{
		TList<int> sIntList;
		sIntList.Insert(12);
		sIntList.Insert(24);
		sIntList.Insert(48);
		sIntList.Insert(96);
		sIntList.Insert(24); // Insert a duplicate value 24
		sIntList.Print();
		sIntList.Delete(24); // Delete one of the duplicate nodes with value 24
		sIntList.Print();
	
		TSet<int> sIntSet;
		sIntSet.Insert(12);
		sIntSet.Insert(24);
		sIntSet.Insert(48);
		sIntSet.Insert(96);
		sIntSet.Insert(24); // Insert a duplicate value 24
		sIntSet.Print();
		sIntSet.Delete(24); // Delete one of the duplicate nodes with value 24
		sIntSet.Print();
	
		cin.get(); // Pause the console window to see the output
	}(*@\label{line: end}@*)
\end{lstlisting}
另外,链表插入要求是在指定的结点之后插入新结点。结点指定是指输入结点序号值。例如,在3号结点之后插入等。
\subsection{}
\kai

改用「模板类派生新的派生类」的方法,就是把用\texttt{int}类型实例化的\texttt{TList}类模板作为基类去派生集合类,此时集合类还使用\texttt{Type}虚拟类型已无意义,故直接改名为\texttt{intSet}。得到程序如下,其中主要修改在第~\ref{line: Change 1}、\ref{line: Change 2s}~\ref{line: Change 2f}、\ref{line: Change 3s}~\ref{line: Change 3f}~行:
\begin{lstlisting}[style=answer, numbers=left, escapeinside={(*@}{@*)}]
	#include <iostream>
	using namespace std;
	
	/* Node structure */
	class SNode
	{
	public:
		SNode(int value);
		~SNode() {};
		int m_value;
		SNode* m_next;
	};
	SNode::SNode(int value)
	{
		m_value = value;
		m_next = NULL;
	}
	
	/* Class template for linked list */
	template <class Type>
	class TList
	{
	public:
		TList();
		~TList();
		// Insert a new node with value after the specified index (default is the head of the list)
		virtual bool Insert(Type value, int index = -1); (*@\label{line: Change 1}@*)
		// Delete the first node with value
		bool Delete(Type value); 
		// Check if the list contains a node with value
		bool Contain(Type value); 
		// Print all the nodes in the list
		void Print(); 
	protected:
		SNode* m_head;
	};
	template <class Type>
	TList<Type>::TList()
	{
		m_head = NULL;
	}
	template <class Type>
	TList<Type>::~TList() // Destructor to delete all nodes still in the list
	{
		SNode* p = m_head;
		for (; p != NULL; )
		{
			m_head = p->m_next;
			delete p;
			p = m_head;
		}
	}
	template <class Type>
	bool TList<Type>::Insert(Type value, int index) (*@\label{line: Change 2s}@*)
	{
		SNode* pTemp = new SNode(value);
		if (pTemp == NULL) return false; // Memory allocation failed
		if (index == -1) // Insert at the head of the list
		{
			pTemp->m_next = m_head;
			m_head = pTemp;
			return true;
		}
		for (SNode* p = m_head; p != NULL; p = p->m_next)
		{
			index--;
			if (index == -1) // Insert at the specified index
			{
				pTemp->m_next = p->m_next;
				p->m_next = pTemp;
				return true;
			}
		}
		return false; // Given index is out of range
	}(*@\label{line: Change 2f}@*)
	template <class Type>
	bool TList<Type>::Delete(Type value)
	{
		SNode* p1, * p2;
		if (m_head->m_value == value)
		{
			p1 = m_head->m_next;
			delete m_head;
			m_head = p1;
			return true;
		}
		else
		{
			for (p1 = m_head, p2 = m_head->m_next; p2 != NULL; ) // Traverse the entire list
			{
				if (p2->m_value == value)
				{
					p1->m_next = p2->m_next;
					delete p2;
					return true;
				}
				else
				{
					p1 = p1->m_next;
					p2 = p2->m_next;
				}
			}
		}
		return false; // Value not found in the list
	}
	template <class Type>
	bool TList<Type>::Contain(Type value)
	{
		for (SNode* p = m_head; p != NULL; p = p->m_next)
		{
			if (p->m_value == value) return true;
		}
		return false;
	}
	template <class Type>
	void TList<Type>::Print()
	{
		cout << "The values of the nodes are:";
		for (SNode* p = m_head; p != NULL; p = p->m_next)
		{
			cout << " " << p->m_value << "; ";
		}
		cout << endl;
	}
	
	/* Class template for set of integers */
	template <class Type>
	class intSet : public TList<int>
	{
	public:
		// Overload for intSet: Insert a new node with value, check if it already exists first
		bool Insert(int value, int index = -1);
	};
	template <class Type>
	bool intSet<Type>::Insert(int value, int index) (*@\label{line: Change 3s}@*)
	{
		if (!(TList<int>::Contain(value)) && TList<int>::Insert(value, index))
			return true; // Value not in the list, insert it successfully
		return false; // Value already exists in the list or insert failed for other reasons
	} (*@\label{line: Change 3f}@*)
	
	// Test code
	void main()
	{
		TList<int> sIntList;
		sIntList.Insert(12); // Insert a new node with value 12 at the head of the list
		sIntList.Insert(24, 0); // Insert a new node with value 24 after the head of the list
		sIntList.Insert(48); // Insert a new node with value 48 at the head of the list
		sIntList.Insert(96, 2); // Insert a new node with value 96 after the second node
		sIntList.Insert(24, 1); // Insert a duplicate value 24 after the first node
		sIntList.Print();
		sIntList.Delete(24); // Delete one of the duplicate nodes with value 24
		sIntList.Print();
	
		intSet<int> sIntSet;
		sIntSet.Insert(12); // Insert a new node with value 12 at the head of the list
		sIntSet.Insert(24, 0); // Insert a new node with value 24 after the head of the list
		sIntSet.Insert(48); // Insert a new node with value 48 at the head of the list
		sIntSet.Insert(96, 2); // Insert a new node with value 96 after the second node
		sIntSet.Insert(24, 1); // Insert a duplicate value 24 after the first node
		sIntSet.Print();
		sIntSet.Delete(24); // Delete one of the duplicate nodes with value 24
		sIntSet.Print();
	
		cin.get(); // Pause the console window to see the output
	}
\end{lstlisting}
程序运行结果为:
\begin{lstlisting}[style=output]
	The values of the nodes are: 48;  12;  24;  24;  96;
	The values of the nodes are: 48;  12;  24;  96;
	The values of the nodes are: 48;  12;  24;  96;
	The values of the nodes are: 48;  12;  96;
	 
\end{lstlisting}
\prtsc{12-03}


%------------------------------------------------------------
\end{document}