\documentclass[10pt, a4paper, oneside, fontset=none]{ctexart}
%调用宏包
\usepackage{amsmath, amsthm, amssymb, graphicx, mathrsfs}
\usepackage[bookmarks=true, colorlinks, citecolor=blue, linkcolor=black]{hyperref}
\usepackage{color, framed, geometry, tcolorbox, nicematrix}
\tcbuselibrary{breakable}%box跨页
\tcbuselibrary{skins}%box跨页不留边
\usepackage{makecell, booktabs}
\usepackage[labelfont={bf}]{caption}
\usepackage{multicol}
\usepackage{enumitem}
\usepackage{extarrows}
\usepackage{esint}
\usepackage{yhmath}
\usepackage{dashrule}% 虚线分割线
\usepackage[text=\includegraphics{C:/Users/16870/.vscode/LaTeX_Application/tex/23Au/图标简稿.png},angle=0]{draftwatermark}%水印
%\usepackage{tikz}
%基本字体设置
\catcode`\,=\active
\def ,{\textup{,}\hskip0.5em }
\setCJKmainfont{FZXSSK.TTF}[BoldFont={SourceHanSerifCN-Bold.otf}, ItalicFont={FZXKTK.TTF}, BoldItalicFont={汉仪颜楷W.ttf}]
\setCJKsansfont{汉仪文黑-45W.ttf}[BoldFont={汉仪文黑-75W.ttf}, ItalicFont={FZYanZQKSJF.TTF}]
\setCJKmonofont{LXGWNeoXiHei.ttf}
%附加字体设置
\newCJKfontfamily{\kaico}{可口可乐在乎体 楷体Coca-ColaCareFontKaiTi.TTF}
\newCJKfontfamily{\kai}{FZXKTK.TTF}[BoldFont={汉仪颜楷W.ttf}, ItalicFont={方正清刻本悦宋 简繁.TTF}, BoldItalicFont={FZYanZQKSJF.TTF}]
\newCJKfontfamily{\yan}{方正清刻本悦宋 简繁.TTF}[ItalicFont={FZYanZQKSJF.TTF}]
\newCJKfontfamily{\xiu}{方正宋刻本秀楷_GBK.TTF}[ItalicFont={方正宋刻本秀楷_GBK.TTF}, BoldFont={FZYanZQKSJF.TTF}]
\newCJKfontfamily{\run}{汉仪润圆-45W.ttf}[BoldFont={汉仪润圆-75W.ttf}, ItalicFont={汉仪润圆-45W.ttf}]
\newCJKfontfamily{\wen}{汉仪文黑-45W.ttf}[BoldFont={汉仪文黑-75W.ttf}, ItalicFont={hk4e_zh-cn.ttf}]
%文档格式
\geometry{left=2.54cm, right=2.54cm, top=3.18cm, bottom=3.18cm}
\setcounter{tocdepth}{3}
\setcounter{secnumdepth}{4}
\linespread{1.4}
\numberwithin{equation}{section}
\renewcommand{\theparagraph}{\Alph{paragraph})}
\newcommand{\Section}[1]{ \refstepcounter{section} \section*{*\thesection\texorpdfstring{\quad}{} #1} \addcontentsline{toc}{section}{\makebox[0pt][r]{*}\thesection\texorpdfstring{\ \ }{} #1} }
\newcommand{\Subsection}[1]{ \refstepcounter{subsection} \subsection*{*\thesubsection\texorpdfstring{\quad}{} #1} \addcontentsline{toc}{subsection}{\makebox[0pt][r]{*}\thesubsection\texorpdfstring{\ \ }{} #1} }
\newcommand{\Subsubsection}[1]{\refstepcounter{subsubsection}\subsubsection*{*\thesubsubsection\texorpdfstring{\quad}{}#1} \addcontentsline{toc}{subsubsection}{\makebox[0pt][r]{*}\thesubsubsection\texorpdfstring{\ \ }{}#1}}
\setlist[itemize]{leftmargin=3em, labelsep=0.25em, itemindent=0em, itemsep=0pt, parsep=0pt, topsep=3pt, partopsep=0pt}
\setlist[description]{leftmargin=4em, labelsep=1em, itemindent=-1em, itemsep=0pt, parsep=0pt, topsep=3pt, partopsep=0pt}
%定理环境
\theoremstyle{plain}
\newtheorem{theorem}{定理}[subsection]
\newtheorem{definition}{定义}[section]
\newtheorem{lemma}[theorem]{引理}
\newtheorem{corollary}[theorem]{推论}
\newtheorem{proposition}[theorem]{命题}

\theoremstyle{definition}
\newtheorem{examplein}[theorem]{\run 例题}
\newtheorem{circum}[theorem]{情形}

\newcommand{\exampleparameter}{0}
\newenvironment{example}[1][0]{% 0/1: no space; 2/3: 5pt space
	\renewcommand{\exampleparameter}{#1}
	\ifnum \exampleparameter>1
		\vspace{10pt}
	\fi
	\hrule
	\vspace{3pt}
	\noindent\hdashrule{\linewidth}{0.5pt}{2pt}
	\vspace{-2em}
	\begin{examplein}
}{% 0/2: -0.5pt space; 1/3: 5pt space
	\end{examplein}
	\vspace{-1em}
	\noindent\hdashrule{\linewidth}{0.5pt}{2pt}\vspace{3pt}
	\hrule
	\ifnum 1=\exampleparameter
		\vspace{10pt}
	\else
		\ifnum 3=\exampleparameter
			\vspace{10pt}
		\else
			\vspace{-0.5pt}
		\fi
	\fi
}

\newenvironment{proofs}[1][\proofname]{\begin{pf}[breakable, enhanced jigsaw]\adjline\begin{proof}[\small#1]\small\kai}{\end{proof}\end{pf}}
\newenvironment{solution}{\begin{proofs}[\small\textit{\yan 解}]\small\renewcommand{\qedsymbol}{$\circledS$}}{\end{proofs}}
\renewcommand{\proofname}{\yan{证明}}
%颜色命名
\definecolor{meihong}{rgb}{0.85,0.2,0.47}
\definecolor{bali}{rgb}{0.2,0.6,0.78}
\definecolor{qinglv}{rgb}{0,0.35,0.32}
%box环境
\newtcolorbox[use counter=definition,number within=section]{defi}[2][]{colback=bali!5!white,colframe=bali!75!black,fonttitle=\sffamily\wen\bfseries,title=重要定义~\thetcbcounter. #2,#1}
\newtcolorbox[use counter=theorem,number within=subsection]{theo}[2][]{colback=meihong!5!white,colframe=meihong!75!black,fonttitle=\sffamily\wen\bfseries,fontupper=\run,title=重要定理~\thetcbcounter. #2,#1}
\newtcolorbox[use counter=definition,number within=section]{defil}[2][]{colback=bali!5!white,colframe=bali!75!black,fonttitle=\sffamily\wen\bfseries,label=#2,title=重要定义~\thetcbcounter. #2,#1}
\newtcolorbox[use counter=theorem,number within=subsection]{theol}[2][]{colback=meihong!5!white,colframe=meihong!75!black,fonttitle=\sffamily\wen\bfseries,fontupper=\run,label=#2,title=重要定理~\thetcbcounter. #2,#1}
\newtcolorbox[auto counter,number within=section]{note}[2][]{colback=qinglv!5!white,colframe=qinglv!75!black,fonttitle=\sffamily\wen\bfseries,title=注~\thetcbcounter. #2,#1}
\newtcolorbox{prenote}[2][]{colback=qinglv!5!white,colframe=qinglv!75!black,fonttitle=\bfseries,title=#2,#1}
\newtcolorbox{pf}[1][]{colback=black!5!white,colframe=white!75!black,#1}
%\newcommand{\mybox}[1]{\tikz[baseline=(MeNode.base)]{\node[rounded corners, fill=gray!20](MeNode){#1};}}

%定义格式记号
\newcommand{\hang}[1][1]{\hangafter 1 \hangindent #1em \noindent}
\newcommand{\adjline}[1][4]{	\lineskiplimit=3pt
	\lineskip=3pt
	\abovedisplayskip=#1pt
	\belowdisplayskip=#1pt}
\newcommand{\den}[2][]{\begin{defi}{#1}\adjline
	\kai #2\end{defi}}
\newcommand{\din}[2][]{\begin{theo}{#1}\adjline
	\run #2\end{theo}}
\newcommand{\de}[2][]{\begin{defil}{#1}\adjline
	\kai #2\end{defil}}
\newcommand{\di}[2][]{\begin{theol}{#1}\adjline
	\run #2\end{theol}}
\newcommand{\dep}[3][]{\begin{defi}{#1\page{#2}}\adjline
	\kai #3\end{defi}}
\newcommand{\dip}[3][]{\begin{theo}{#1\page{#2}}\adjline
	\run #3\end{theo}}
\newcommand{\zhu}[2][]{\begin{note}{#1}\adjline
	\xiu #2\end{note}}
\newcommand{\colors}[1]{\color{#1!75!black}}
\newcommand{\tboba}[1]{\textbf{\kai\color{bali!75!black}#1}}
\newcommand{\mboba}[1]{\kai\boldsymbol{\color{bali!75!black}#1}}
\newcommand{\tbome}[1]{\textbf{\run\color{meihong!75!black}#1}}
\newcommand{\mbome}[1]{\run\boldsymbol{\color{meihong!75!black}#1}}
\newcommand{\tboqi}[1]{\textbf{\xiu\color{qinglv!75!black}#1}}
\newcommand{\mboqi}[1]{\xiu\boldsymbol{\color{qinglv!75!black}#1}}
%定义算符
\def\upint{\mathchoice%
	{\mkern13mu\overline{\vphantom{\intop}\mkern7mu}\mkern-20mu}%
	{\mkern7mu\overline{\vphantom{\intop}\mkern7mu}\mkern-14mu}%
	{\mkern7mu\overline{\vphantom{\intop}\mkern7mu}\mkern-14mu}%
	{\mkern7mu\overline{\vphantom{\intop}\mkern7mu}\mkern-14mu}%
 \int}
\def\lowint{\mkern3mu\underline{\vphantom{\intop}\mkern7mu}\mkern-10mu\int}
\renewcommand{\cfrac}[2]{\genfrac{}{}{}{0}{\raisebox{0.6em}{$#1$}}{\raisebox{-0.8em}{$#2$}}}
\newcommand{\dif}{\mathop{}\!\mathrm{d}}
\newcommand{\e}{\mathrm{e}}
\renewcommand{\i}{\mathrm{i}}
\newcommand{\R}{\mathbb{R}}
\newcommand{\tp}{^\mathrm{T}}
\renewcommand{\v}[1]{\vec{\boldsymbol{#1}}}
\newcommand{\V}[2][-0.665]{\vec{#2\hspace{#1em}#2\hspace{#1em}#2}}
\newcommand{\dint}[1][]{\displaystyle\int #1}
\newcommand{\page}[1]{\hfill P$_\text{#1}$}
\newcommand{\sumi}[1][n]{\sum\limits_{#1=1}^\infty}
\renewenvironment{cases}[1][l]{\left\{\,\begin{NiceArray}{#1}}{\end{NiceArray}\right.}

%标题、作者、日期
\title
{
	\textbf{高等微积分(2)}\textit{知识与方法}
}
\author{\zihao{5} T$^\text{T}$T}
\date{\zihao{5}\kai \today}
%----------------------------------------------------------
\begin{document}

\maketitle
\begin{multicols}{2}
	\begin{flushleft}
		\tableofcontents
	\end{flushleft}
\end{multicols}

\adjline

\newpage
%----------------------------------------------------------
\setcounter{section}{6}
\section{常数项级数}

\subsection{无穷级数的基本性质}

\de[常数项级数]{
	数列~$\{u_n\}$~的形式无穷和$u_1+u_2+\cdots+u_n+\cdots$称为以$u_n$为通项的\tboba{级数},记作$\sum\limits_{n=1}^{\infty}u_n$或者$\sum\limits_{n=1}^{+\infty}u_n$。
	$\forall n\ge 1$,定义$ S_n=\sum\limits_{k=1}^nu_k$,称之为$ \sum\limits_{n=1}^{\infty}u_n$的\tboba{部分和}。
	若数列$\{S_n\}$收敛到$S \in \mathbb{R}$或者发散,则相应地称级数$ \sum\limits_{n=1}^{\infty}u_n$~\tboba{收敛}到$S\in\mathbb{R}$或者\tboba{发散}。
}

\zhu[对级数理论的进一步理解]{
	\hang[2](1)\textbf{级数理论与数列理论完全一致。}由级数 $ \sum\limits_{n=1}^{\infty}u_n$ 可以来构造部分和数列 $\{S_n\}$,反过来,若 $\{S_n\}$ 为任意的数列,定义 $u_1=S_1$,且 $u_n=S_n-S_{n-1}\ (n\ge 2)$,那么 $\{S_n\}$ 就是级数$ \sum\limits_{n=1}^{\infty}u_n$的部分和数列,且级数和 $ \sum\limits_{k=1}^{\infty}u_k=\lim\limits_{n\rightarrow \infty}S_n$。
 
	\hang[2](2)\textbf{级数理论是广义积分理论的特殊情形。}$\forall n \ge 1$以及$\forall x\in [n-1, n)$,定义 $f(x)=u_n$,于是 $\forall N\ge 1$,我们均有
	$\dint_0^{N}f(x)\dif x
	=\textstyle\sum\limits_{n=1}^{N}\dint_{n-1}^{n}f(x)\dif x 
	=\textstyle\sum\limits_{n=1}^Nu_n$。
	由此我们立刻得,级数 $ \sum\limits_{n=1}^{\infty} u_n$收敛当且仅当广义积分 $ \dint_0^{+\infty}f(x)\dif x$收敛, 此时还有
	$\sum\limits_{n=1}^{\infty}u_n=\dint_0^{+\infty}f(x)\dif x$。
}

任意地去掉、添加或者改变级数的有限多项,不会改变其敛散性,但会改变该级数的和。

\di[收敛级数的结合性]{
	如果级数$\sumi u_n = S$,则对于任意严格递增的自然数数列$\{n_k\}$(约定$n_0=0$),均有
	\begin{equation}
		sumi[k] \sum\limits_{n=n_{k-1}+1}^{n_k} u_n=S
	\end{equation}\
}
\begin{proofs}
	$\sumi[k] \sum\limits_{n=n_{k-1}+1}^{n_k} u_n
	=\lim\limits_{K\to\infty} \sum\limits_{k=1}^{K}\sum\limits_{n=n_{k-1}+1}^{n_k} u_n
	=\lim\limits_{K\to\infty} \sum\limits_{n=1}^{n_K} u_n=S$。
\end{proofs}
一般地,定理~\ref{收敛级数的结合性}~的逆命题不成立,如$\sumi (-1)^n$发散,但$\sumi[k] \bigl((-1)^{2k-1}+(-1)^{2k}\bigr)=0$。\textbf{无穷级数的结合性一般只在收敛性的前提下才成立。}此外,如果 $\forall k\ge 1$,和式 $v_k=\sum\limits_{n=n_{k-1}+1}^{n_k}u_n$中的项 恒 $\ge 0$或者恒 $\le 0$,由夹逼定理({\kai 若$n_{k-1} \le n \le n_k$,则$S_n(u)$介于$S_{k-1}(v),S_k(v)$之间})可知无穷级数的结合性也成立。

\di[级数收敛的必要条件]{
	若级数$\sumi u_n$收敛,则$\lim\limits_{n \to \infty} u_n=0$。
}
\begin{proofs}
	$\lim\limits_{n \to \infty} u_n=\lim\limits_{n \to \infty}(S_n-S_{n-1})=\lim\limits_{n \to \infty}S_n-\lim\limits_{n \to \infty}S_{n-1}=0$。
\end{proofs}

\subsection{数项级数的判敛法则}

\subsubsection{数列衍生的判敛法则}

\di[数项级数的Cauchy准则]{
	级数 $\sumi[k]u_k$收敛,当且仅当$\forall \varepsilon>0$,$\exists N>0$ 使得 $\forall m> n\ge N$,均有
	\begin{equation}
	|S_m-S_n|=\left|\sum_{k=n+1}^mu_k\right|<\varepsilon
	\end{equation}
}

\di[非负项级数的单调有界定理]{
	非负项级数收敛,当且仅当其部分和数列有上界。
}
\begin{corollary}
	若非负项级数$\sumi[k]u_k$发散,则$\sumi[k]u_k=+\infty$。
\end{corollary}
\di[非负项级数的积分判别法]{
	设 $f:[1,+\infty)\rightarrow [0,+\infty)$ 为\textbf{单调}函数,则 $\sumi f(n)$ 收敛当且仅当 $ \dint_1^{+\infty}f(x)\dif x$ 收敛。
}
\begin{proofs}
	\adjline[2]
	(1)\textbf{充分性}。设 $ \dint_1^{+\infty}f(x)\dif x$ 收敛,
	则 $\forall n\ge 1$,均有
	\begin{align*}
	S_n&=\sum_{k=1}^nf(k) 
	\le f(1)+\sum_{k=2}^n\int_{k-1}^{k}f(x)\dif x \\[-4pt]
	&=f(1) + \int_1^{n} f(x)\dif x 
	\le f(1) + \int_1^{+\infty} f(x)\dif x 
	<+\infty
	\end{align*}
	于是由单调有界定理可知 级数 $\sumi f(n)$ 收敛。

	(2)\textbf{必要性}。设级数 $ \sum\limits_{n=1}^{\infty}f(n)$ 收敛,
	则 $\forall A\ge 1$,均有
	\begin{equation*}
	\int_1^{A}f(x)\dif x \le \int_1^{[A]+1}f(x)\dif x 
	=\sum_{k=1}^{[A]}\int_{k}^{k+1}f(x)\dif x 
	=S_{[A]} 
	\le \sum_{n=1}^{\infty}f(n) 
	<+\infty
	\end{equation*}
	于 是 我 们 由 单调有界定理可知 $ \dint_1^{+\infty}f(x)\dif x$ 收敛。

	当 $f$ 为 单 调 递 增 时,证明类似,但此时如果级数$\sumi f(n)$ 收敛,则 $ \lim\limits_{n\rightarrow\infty}f(n)=0$,进而可得 $f\equiv 0$。	
\end{proofs}

\subsubsection{比较、比率、根值判别法}

\di[非负项级数的比较判别法]{
	设有非负项级数$\sumi u_n,\sumi v_n$满足$u_n=O(v_n) ~ (n \to \infty)$,也即存在$N>0$以及$C>0$,使得$\forall n>N$,均有$|u_n| \le C|v_n|$。那么:

	(1)若$\sumi v_n$收敛,则$\sumi u_n$收敛;

	(2)若$\sumi u_n$发散,则$\sumi v_n$发散。
}
\begin{corollary}
	设有正项级数$\sumi u_n,\sumi v_n$满足$\lim\limits_{n\to\infty} \dfrac{u_n }{v_n }=c$,那么:

	(1)若$0<c<+\infty$,则$\sumi u_n,\sumi v_n$同敛散;
	
	(2)若$c=0$且$\sumi v_n$收敛,则$\sumi u_n$收敛;

	(3)若$c=+\infty$且$\sumi v_n$发散,则$\sumi u_n$发散。
\end{corollary}
\zhu[]{
	在正项级数中,两个等价无穷小数列具有相同的敛散性。但一般常数项级数没有这个性质,反例如$\sumi \dfrac{(-1)^n}{\sqrt{n}}$与$\sumi \left(\dfrac{(-1)^n}{\sqrt{n}} + \dfrac{1}{n}\right)$敛散性不同,但$\lim\limits_{n \to \infty} \cfrac{\dfrac{(-1)^n}{\sqrt{n}}}{\dfrac{(-1)^n}{\sqrt{n}} + \dfrac{1}{n}}=1$。
}
\begin{example}
	判断级数$\sum\limits_{n=1}^{\infty} \dfrac{1}{(n+1)\sqrt{n}+n\sqrt{n+1}}$的敛散性。
\end{example}
\begin{solution}
	当 $n\rightarrow \infty$ 时,我们有
	\begin{equation*}
	\frac{1}{(n+1)\sqrt{n}+n\sqrt{n+1}}
	=\frac{1}{n^{\frac{3}{2}}}\cdot \frac{1}{\left(1+\frac{1}{n}\right)+\sqrt{1+\frac{1}{n}}} \sim\frac{1}{2n^{\frac{3}{2}}}
	\end{equation*}
	故所求级数收敛,并且我们还有 
	\begin{align*}
	\sum_{n=1}^{\infty}\frac{1}{(n+1)\sqrt{n}+n\sqrt{n+1}}
	&=\sum_{n=1}^{\infty}\frac{1}{\sqrt{n(n+1)}(\sqrt{n+1}+\sqrt{n})}\\
	&=\sum_{n=1}^{\infty}\frac{\sqrt{n+1}-\sqrt{n}}{\sqrt{n(n+1)}} 
	=\sum_{n=1}^{\infty}\left(\frac{1}{\sqrt{n}}-\frac{1}{\sqrt{n+1}}\right) \\
	&=\lim_{N\rightarrow\infty}\sum_{n=1}^{N}\left(\frac{1}{\sqrt{n}}-\frac{1}{\sqrt{n+1}}\right)
	=\lim_{N\rightarrow\infty}\left(1-\frac{1}{\sqrt{N+1}}\right) 
	=1 \qedhere
	\end{align*}
\end{solution}

\begin{example}
	判断级数 $\sum\limits_{n=1}^{\infty}\left(\dfrac{1}{n}-\ln\left(1+\dfrac{1}{n}\right)\right)$ 的敛散性。
\end{example}
\begin{solution}
	由带 Peano 余项的 Taylor 展开$\ln(1+x)=x-\dfrac{x^2}{2}+o(x^2)$,
	当 $n\rightarrow \infty$ 时,我们有
	\begin{equation*}
	\frac{1}{n}-\ln\left(1+\frac{1}{n}\right)
	=\frac{1}{n}-\left(\frac{1}{n}-\frac{1}{2n^2}\left(1+o(1)\right)\right)
	=\frac{1}{2n^2}\left(1+o(1)\right)
	\end{equation*}
	而级数 $ \sum\limits_{n=1}^{\infty}\dfrac{1}{2n^2}$ 收敛,于是由比较判别法立刻可知级数 $\sum\limits_{n=1}^{\infty}\left(\dfrac{1}{n}-\ln\left(1+\dfrac{1}{n}\right)\right)$收敛。
\end{solution}

\begin{example}
	设 $p,q\in\mathbb{R}$,讨论$ \sum\limits_{n=2}^{\infty}\dfrac{1}{n^p(\ln n)^q}$ 的敛散性。
\end{example}
\begin{solution}
	\adjline
	$\forall n\ge 2$,令 $u_n=\dfrac{1}{n^p(\ln n)^q}$。

	\hang[2](1)当 $p>1$ 时,$\forall n\ge 2$,令 $v_n=\dfrac{1}{n^{\frac{1}{2}(1+p)}}$,则
	$
	\lim\limits_{n\rightarrow\infty}\dfrac{u_n}{v_n}
	=\lim\limits_{n\rightarrow\infty}\dfrac{1}{n^{\frac{1}{2}(p-1)}(\ln n)^q} =0
	$,
	而 $\dfrac{1}{2}(1+p)>1$,由此知 $ \sum\limits_{n=2}^{\infty}v_n$ 收敛,进而由比较判别法立刻可知$ \sum\limits_{n=2}^{\infty}u_n$ 收敛。

	(2)当 $p<1$ 时,$\forall n\ge 2$,令 $v_n=\dfrac{1}{n^{\frac{1}{2}(1+p)}}$,则
	$
	\lim\limits_{n\rightarrow\infty}\dfrac{u_n}{v_n}
	=\lim\limits_{n\rightarrow\infty}\dfrac{n^{\frac{1}{2}(1-p)}}{(\ln n)^q} 
	=+\infty
	$,
	而 $\dfrac{1}{2}(1+p)<1$,故 $ \sum\limits_{n=2}^{\infty}v_n$ 发散,进而 $ \sum\limits_{n=2}^{\infty}u_n$ 发散。

	(3)当 $p=1$ 时,由于级数 $ \sum\limits_{n=2}^{\infty}u_n$与广义积分
	$
	\dint_2^{+\infty} \frac{\dif x}{x(\ln x)^q}
	\xlongequal{t=\ln x} \int_{\ln 2}^{+\infty}\frac{\dif t}{t^q}
	$
	同敛散,由此可知当 $q>1$ 时,级数 $ \sum\limits_{n=2}^{\infty}u_n$ 收敛,而当 $q\le 1$ 时,级数 $ \sum\limits_{n=2}^{\infty}u_n$ 发散。
\end{solution}

\di[正项级数的比率判别法(d'Alembert判别法)]{
	设正项数列$\{u_n\}$满足$\limsup\limits_{n \to \infty} \dfrac{u_{n+1}}{u_n} = \rho \in \R \cup \{+\infty\}$,那么

	(1)若$\rho < 1$,则级数$\sumi u_n$收敛;

	(2)若$\rho > 1$,则级数$\sumi u_n$发散。

	若$\rho = 1$,则d'Alembert判别法无法判断级数$\sumi u_n$的敛散性。
}
\begin{proofs}
	与等比级数应用比较判别法即得。
\end{proofs}

\begin{example}
	判断级数 $ \sumi \dfrac{n!}{3^n},\sumi\dfrac{n!}{n^n}$ 的敛散性。
\end{example}
\begin{solution}
	当 $n\rightarrow \infty$ 时,我们有
	\begin{equation*}
	\cfrac{\ \dfrac{(n+1)!}{3^{n+1}}\ }{\dfrac{n!}{3^n}}=\cfrac{n+1}{3} \rightarrow+\infty
	\qquad
	\cfrac{\ \dfrac{(n+1)!}{(n+1)^{n+1}}\ }{\dfrac{n!}{n^n}}
	=\left(\frac{n}{n+1}\right)^n =\frac{1}{(1+\frac{1}{n})^n} \rightarrow \frac{1}{e} <1
	\end{equation*}
	故级数 $ \sumi\dfrac{n!}{3^n}$ 发散,
	而级数 $ \sumi\dfrac{n!}{n^n}$ 收敛。
\end{solution}

\di[非负项级数的根值判别法(Cauchy判别法)]{
	设非负数列$\{u_n\}$满足$\limsup\limits_{n \to \infty} \sqrt[n]{u_n} = \rho \in \R \cup \{+\infty\}$,那么

	(1)若$\rho < 1$,则级数$\sumi u_n$收敛;

	(2)若$\rho > 1$,则级数$\sumi u_n$发散。

	若$\rho = 1$,则 Cauchy 判别法无法判断级数$\sumi u_n$的敛散性。
}
\begin{proofs}
	\adjline
	\abovedisplayskip=2pt
	\belowdisplayskip=2pt
	\hang[2](1)设 $\rho<1$,任取 $q \in (\rho, 1)$,由题设可知
	\begin{equation*}
	q>\limsup_{n\rightarrow\infty}\sqrt[n]{u_n} =\lim_{n\rightarrow\infty}\sup_{k\ge n}\sqrt[k]{u_k}
	\end{equation*}
	于是由数列极限的保序性可知 $\exists N>0$ 使得
	$q>\sup\limits_{k\ge N}\sqrt[k]{u_k}$,
	从而 $\forall n > N$,均有 $\sqrt[n]{u_n} <q$,
	故我们有 $u_n<q^{n}$。
	而级数 $ \sumi q^{n}$ 收敛,
	因此由比较判别法立刻可知 
	级数 $ \sumi u_n$ 收敛。
	
	(2)设 $\rho>1$,任取 $q\in (1, \rho)$,由题设可知
	\begin{equation*}
	q<\limsup_{n\rightarrow\infty}\sqrt[n]{u_n} =\lim_{n\rightarrow\infty}\sup_{k\ge n}\sqrt[k]{u_k}
	\end{equation*}
	由递降数列的极限的性质可知, $\forall n\ge 1$, 均有
	$q<\sup\limits_{k\ge n}\sqrt[k]{u_k}$,
	由此可以构造严格递增自然数数列 $\{n_k\}$ 使得
	$q<(u_{n_k})^{\frac{1}{n_k}}$,则 $u_{n_k}>q^{n_k} >1$,从而数列 $\{u_{n_k}\}$
	不收敛到 $0$,进而可知级数 $ \sumi u_n$ 发散。
\end{proofs}

\zhu[比率、根值判别法的适用]{
	\hang 若正项级数$\sumi u_n$满足比率判别法的收敛条件$\limsup\limits_{n \to \infty} \dfrac{u_{n+1}}{u_n} = \rho < 1$,则有 
	\begin{align*}
		\limsup\limits_{n \to \infty} \sqrt[n]{u_n} 
		&= \limsup\limits_{n \to \infty} \e^\frac{\ln u_n}{n}
		\xlongequal{\textup{Stolz}} \limsup\limits_{n \to \infty} \e^\frac{\ln u_{n+1} - \ln u_n}{(n+1)-n} \\[-3pt]
		&= \limsup\limits_{n \to \infty} \e^{\ln \frac{u_{n+1}}{u_n}}
		= \limsup\limits_{n \to \infty} \dfrac{u_{n+1}}{u_n} = \rho < 1
	\end{align*}
	可知其也满足根值判别法的收敛条件。

	\hang 进一步,若正项级数$\sumi u_n$满足根值判别法的收敛条件$\limsup\limits_{n \to \infty} \sqrt[n]{u_n} = \rho < 1$,则对任意的$r \in (\rho,1)$,都有
	$\limsup\limits_{n \to \infty} \sqrt[n]{\dfrac{u_n}{r^n}} = \limsup\limits_{n \to \infty} \dfrac{\sqrt[n]{u_n}}{r} = \dfrac{\rho}{r} < 1$,于是级数$\sumi \dfrac{u_n}{r^n}$收敛,则$\lim\limits_{n\to\infty}\dfrac{u_n}{r^n} = 0$。由此可知,满足比率判别法或根值判别法的收敛条件的级数的收敛速度快于几何级数$\sumi \dfrac{1}{r^n}$。
}

\subsubsection{一般数项级数的判敛法则}
\begin{theorem}
	若级数$\sumi |u_n|$收敛,则级数$\sumi u_n$收敛。
\end{theorem}
\de[绝对收敛]{
	若级数$\sumi |u_n|$收敛,则称级数$\sumi u_n$~\tboba{绝对收敛};若级数$\sumi u_n$收敛而级数$\sumi |u_n|$不收敛,则称级数$\sumi u_n$~\tboba{条件收敛}。
}

\di[绝对收敛的交换性]{
	如果级数$\sumi u_n = S$绝对收敛,$\varphi: \mathbb{N}^* \to \mathbb{N}^*$为正整数集到其自身的双射,则$\sumi u_{\varphi(n)} = S$。
}
\begin{proofs}
	$\forall N\ge 1$,$\sum\limits_{n=1}^{N}|u_{\varphi(n)}|\le \sum\limits_{k=1}^{\infty}|u_k| <+\infty$,从而由单调有界定理可知$\sum\limits_{n=1}^{\infty}u_{\varphi(n)}$绝对收敛。
	又 $\forall \varepsilon > 0$,
	由题设可知$\exists N>0$使得$\forall n\ge N$,均有
	$\sum\limits_{k=n+1}^{\infty}|u_k|<\varepsilon$。

	令 $K = \max\limits_{1\le j\le N}\varphi^{-1}(j)$,
	则 $\forall 1\le j\le N$,我们均有$1\le \varphi^{-1}(j)\le K$。于是$\forall n>K$,我们有
	\abovedisplayskip=2pt
	\belowdisplayskip=3pt
	\begin{equation*}
	\left|\sum_{k=1}^nu_{\varphi(k)}-S\right|
	=\left|\sum_{k=1}^nu_{\varphi(k)}-\sum_{j=1}^{\infty}u_j\right| 
	=\left|\sum_{k=1}^nu_{\varphi(k)}-\sum_{j=1}^{N}u_j-\sum_{j=N+1}^{\infty}u_j\right| 
	\le \sum_{j=N+1}^{\infty}|u_j| <\varepsilon
	\end{equation*}
	从而由级数和的定义可知 
	$\sumi[k] u_{\varphi(k)}=S$,
	故所证结论成立。
\end{proofs}
\begin{theorem}
给定级数$\sumi a_n$条件收敛,则适当交换它的各项的次序,可以使其收敛到任意事先指定的实数$S$,也可以使其发散到$+\infty$ 或$-\infty$。
\end{theorem}

\begin{lemma}
	\textup{\textbf{Able 引理}}\quad 设$\{b_n\}$单调,记$S_k=\sum\limits_{i=1}^k a_k$,且有$\forall k \in \mathbb{N}^*,|S_k| \le M$,那么 
	\begin{equation}
		\left|\sum_{k=1}^n a_kb_k\right| \le M\left(|b_1|+2|b_n|\right)
	\end{equation}
\end{lemma}

\di[数项级数的Dirichlet判别准则]{
	如果级数$\sumi u_n$的部分和数列有界,并且数列$\{v_n\}$ 单调且趋于$0$,则级数$\sumi u_nv_n$ 收敛。
}
\begin{proofs}
	$\forall n\ge 1$,定义 $S_n= \sum\limits_{k=1}^nu_k$,那么由题设立刻可知 $\exists M>0$ 使得 $\forall n\ge 1$,我们均有 $|S_n|<M$。$\forall \varepsilon>0$,同样由题设可知数列 $\{v_n\}$ 趋于 $0$,于是$\exists N>0$ 使得$\forall n\ge N$,我们均会有$|v_n|<\dfrac{\varepsilon}{4M}$。进而可知,$\forall m>n>N$,我们有
	\begin{align*}
		& \left|\sum\limits_{k=n}^m {u_k}v_k\right|
		=\left|\sum\limits_{k=n}^m{(S_{k}-S_{k-1})}v_k\right| 
		=\left|\sum\limits_{k=n}^m {S_{k}}v_k-\sum\limits_{k=n}^m {S_{k-1}}v_k\right| \\
		&=\left|\sum\limits_{k=n}^m {S_{k}}v_k-\sum\limits_{k=n-1}^{m-1}{S_{k}}v_{k+1}\right| 
		=\left|{S_{m}}v_m+\sum\limits_{k=n}^{m-1}{S_{k}}(v_k-v_{k+1})-{S_{n-1}}v_n\right| \\
		& \le |{S_{m}}||v_m|+|{S_{n-1}}||v_n|+\sum\limits_{k=n}^{m-1}|{S_{k}}||v_k-v_{k+1}| \\
		& \le \frac{\varepsilon}{4}+\frac{\varepsilon}{4}+M{\sum\limits_{k=n}^{m-1}|v_k-v_{k+1}|} 
		=\frac{\varepsilon}{2}+M{\left|\sum\limits_{k=n}^{m-1}(v_k-v_{k+1})\right|} \\
		&=\frac{\varepsilon}{2}+M|v_n-v_m| 
		\le \frac{\varepsilon}{2}+M(|v_n|+|v_m|) 
		\le \frac{\varepsilon}{2}+M\cdot 2\cdot \frac{\varepsilon}{4M} 
		=\varepsilon
	\end{align*} 
	从而由Cauchy准则可知 级数$\sum\limits_{n=1}^{\infty}u_nv_n$  收敛。
\end{proofs}
由Dirichlet判别准则,易知级数$\sumi (-1)^n \dfrac{1}{n}$、$\sumi (-1)^n \dfrac{1}{\sqrt{n}}$、$\sumi (-1)^n \dfrac{1}{n\ln n}$均为条件收敛。

\di[数项级数的Abel判别准则]{
	如果级数$\sumi u_n$收敛,并且数列$\{v_n\}$ 单调有界,则级数$\sumi u_nv_n$ 收敛。
}
\begin{proofs}
	设$\lim\limits_{n \to \infty} v_n = v$,则由Dirichlet判别准则知$\sumi u_n(v_n-v)$收敛。
\end{proofs}
\di[Leibniz判别准则]{
	如果正项数列$\{v_n\}$单调下降趋于0,则交错级数$\sumi (-1)^{n-1}v_n$收敛。
}

\begin{example}
	设$x\in\mathbb{R}$,判断$\sum\limits_{n=1}^{\infty}(-1)^n\dfrac{x^n}{n}$  的敛散性。
\end{example}
\begin{solution}
	当 $|x|<1$ 时,由于 $\lim\limits_{n\rightarrow\infty}\left(\dfrac{|x|^n}{n}\right)^{\frac{1}{n}}=|x|<1$,原级数绝对收敛;
	
	当$|x|>1$时,由于$\lim\limits_{n\rightarrow\infty}(-1)^n\dfrac{x^n}{n} = \infty \neq 0$,原级数发散;

	当$x=1$时,由Leibniz判别法可知 $\sum\limits_{n=1}^{\infty}(-1)^n\dfrac{1}{n}$  收敛;
	
	当$x=-1$时,原级数的通项变为$(-1)^n\dfrac{(-1)^n}{n}=dfrac{1}{n}$,由此可知此时级数$\sum\limits_{n=1}^{\infty}(-1)^n\dfrac{x^n}{n}$  发散。
\end{solution}

\begin{example}
	判断级数$\sumi \dfrac{\cos n }{n }$的敛散性。
\end{example}
\begin{solution}
	由积化和差公式$2\sin a\cos b=\sin(a+b)+\sin (a-b)$,有$\cos n = \dfrac{\sin(\frac{1}{2}+n)+\sin(\frac{1}{2}-n)}{2\sin \frac{1}{2}}$,从而
	\abovedisplayskip=2pt
	\belowdisplayskip=3pt
	$$\left|\sum_{n=1}^N \cos n\right|
	=\left|\sum_{n=1}^N \dfrac{\sin(n+\frac{1}{2})-\sin(n-\frac{1}{2})}{2\sin \frac{1}{2}}\right|
	=\left|\frac{\sin(N+\frac{1}{2})-\sin \frac{1}{2}}{2\sin \frac{1}{2}}\right| 
	\le \frac{\left|\sin(N+\frac{1}{2})\right|+\left|\sin \frac{1}{2}\right|}{2\sin \frac{1}{2}} \le \frac{1}{\sin\frac{1}{2}}$$
	而$\left\{\dfrac{1}{n}\right\}$递降趋于$0$,则由Dirichlet判别准则知原级数收敛。
	
	又 $\dfrac{|\cos n|}{n}\ge \dfrac{\cos^2n}{n} =\dfrac{1+\cos 2n}{2n}$,并且同样也由Dirichlet 判别准则可知 $ \sum\limits_{n=1}^{\infty}\dfrac{\cos 2n}{2n}$收敛,则$ \sum\limits_{n=1}^{\infty}\dfrac{\cos^2n}{n}=+\infty$发散,故原级数为条件收敛。
\end{solution}

\begin{example}
	若$f\in\mathscr{C}^{(2)}[-1,1]$ 使得$\lim\limits_{x\rightarrow 0}\dfrac{f(x)}{x}=0$,求证:
	级数$\sum\limits_{n=1}^{\infty}f\left(\dfrac{1}{n}\right)$绝对收敛。  
\end{example}
\begin{proofs}
	因$f$连续,故$f(0)=\lim\limits_{x\rightarrow 0}\dfrac{f(x)}{x}\cdot x=0$.   又$f'(0)=\lim\limits_{x\rightarrow 0}\dfrac{f(x)-f(0)}{x}  =0$,
	进而由$f\in\mathscr{C}^{(2)}[-1,1]$以及L'Hospital法则得 
	\abovedisplayskip=2pt
	\belowdisplayskip=3pt
	\begin{equation*}
	\lim\limits_{n\rightarrow\infty}\frac{f(\frac{1}{n})}{\frac{1}{n^2}}=\lim\limits_{x\rightarrow 0}\frac{f(x)}{x^2} 
	=\lim\limits_{x\rightarrow 0}\frac{f'(x)}{2x} 
	=\lim\limits_{x\rightarrow 0}\frac{1}{2}f''(x) 
	=\frac{1}{2}f''(0)
	\end{equation*}
	但$ \sum\limits\limits_{n=1}^{\infty}\dfrac{1}{n^2}$收敛, 
	由比较法则可知题设绝对收敛。
\end{proofs}

\newpage
\section{函数项级数}

\subsection{函数列与函数项级数的收敛性}

\de[函数列的收敛域与极限]{
	设$I\subseteq\R$为非空集合,  而$\{v_n(x)\},x\in I$为定义
	在$I$上的一列函数,  称为$I$上的\tboba{函数列}。
	
	\hang[2](1)设$x_0\in I$,若数列$\{v_n(x_0)\}$收敛,  则称点$x_0$
	为上述函数列的\tboba{收敛点},  否则称为\tboba{发散点};记$J$是由上述函数列的所有收敛点组成的集合,  称为该函数列的\tboba{收敛域}。

	(2)$\forall x\in J$,  定义$v(x)=\lim\limits_{n\rightarrow\infty}v_n(x)$,由此得到的定义在$J$上的函数$v$称为函数列的\tboba{极限函数};

	(3)若$\exists M>0$ 使得 $\forall n\ge 1$以及$\forall x\in I$,  均有$|v_n(x)|\le M$,则称函数列$\{v_n\}$ 在$I$上\tboba{一致有界}; 

	\hang[2](4)如果$\forall \varepsilon>0$,  $\exists N>0$   使得$\forall n>N$ 以及 $\forall x\in J$,  均 有$|v_n(x)-v(x)|<\varepsilon$,则也称函数列$\{v_n\}$ 在$J$ 上\tboba{一致收敛}到它的极限函数$v$;此即,$\forall \varepsilon>0$, $\exists N>0$   使$\forall n>N$,  均有$\sup\limits_{x\in J}|v_n(x)-v(x)|<\varepsilon$;亦即,$\mboba{\lim\limits_{n\rightarrow\infty}\sup\limits_{x\in J}|v_n(x)-v(x)|=0}$。
}
\begin{example}
	\textbf{收敛但不一致收敛的函数列。}$\forall n\ge 1$及$\forall x\in [0,1]$,  令$v_n(x)=x^n$,则 
	\begin{equation*}
	\lim_{n\rightarrow\infty}v_n(x)=v(x)=
	\begin{cases}[ll]
		0,&\textrm{若}~x\in [0,1),\\
		1,&\textrm{若}~x=1
	\end{cases}
	\end{equation*}
	又$\forall n\ge 1$,  我们有
	\begin{equation*}
	\sup_{x\in [0,1]}|v_n(x)-v(x)|
	=\sup_{x\in [0,1)}x^n 
	=1
	\end{equation*}
	故函数列$\{x^n\}$在$[0,1]$上收敛,但非一致收敛;然而,$\forall \delta \in (0,1)$,有
	\begin{equation*}
	\sup_{x\in [0,1-\delta]}|v_n(x)-v(x)|
	=\sup_{x\in [0,1-\delta]}x^n 
	=(1-\delta)^n
	\end{equation*}
	故函数列$\{x^n\}$在$[0,1-\delta]$上一致收敛。
\end{example}

\de[函数项级数]{
	设$I\subseteq\mathbb{R}$为非空集合,  而$\{u_n\}$为定义在$I$上的一列函数,  我们称形式和
	$
	\sum\limits_{n=1}^{\infty}u_n(x)=u_1(x)+u_2(x)+\cdots+u_n(x)+\cdots
	$
	为$I$上的\tboba{函数项级数}。
	
	(1)设$x_0\in I$,若级数$\sum\limits_{n=1}^{\infty}u_n(x_0)$ 收敛, 
	则称$x_0$为上述函数项级数的\tboba{收敛点},  否则称为\tboba{发散点};记$J$为上述函数项级数所有收敛点组成的集合,  称为该函数项级数的\tboba{收敛域};

	(2)$\forall x\in J$,  令$  S(x)=\sum\limits_{n=1}^{\infty}u_n(x)$,  由此得到$J$上函数$S$,  称为上述函数项级数的\tboba{和函数};

	(3)称函数项级数$ \sum\limits_{n=1}^{\infty}u_n(x)$ 在$J$ 上为\tboba{一致收敛}, 若$\{S_n(x)\}$ 在$J$ 上 一 致 收 敛, 其 中$  S_n(x)=\sum\limits_{k=1}^n u_k(x)$;此即,$\forall \varepsilon>0$,  $\exists N>0$  使得$\forall x\in J$以及$\forall n>N$, 
	\begin{equation*}
		|S_n(x)-S(x)|=\left|\sum_{k=1}^{n}u_k(x)-S(x)\right|=\left|\sum_{k=n+1}^{\infty}u_k(x)\right| <\varepsilon
	\end{equation*}
	亦即,$\lim\limits_{n\rightarrow\infty}\sup\limits_{x\in J}\left|\sum\limits_{k=n+1}^{\infty}u_k(x)\right|=0$。
}
\di[函数项级数一致收敛的必要条件]{
	若函数项级数 $ \sum\limits_{n=1}^{\infty} u_n$ 在$J$  上一致收敛,
	则$\{ u_n \}$在$J$上一致趋于$0$ 。
}
\begin{proofs}
	\adjline
	由题设立刻知,  $\forall \varepsilon>0$, 
	$\exists N>0$  使得$\forall m\ge n>N$以及$\forall x\in J$,  均有
	$$\left|\sum\limits_{k=n}^m u_k(x)\right| = |S_m(x)-S_n(x)| \le |S_m(x)-S(x)| + |S_n(x)-S(x)| <2\varepsilon$$
	特别地,  $\forall n>N$ 以及$\forall x\in J$,  均有  $ |u_n(x)|<\varepsilon$, 
	这表明函数列$\{u_n\}$在$J$上一致趋于$0$。
\end{proofs}

\begin{example}
	求$ \sum\limits_{n=1}^{\infty}\dfrac{(-1)^n}{n}\left(\dfrac{1}{2x+1}\right)^n$的收敛域。
\end{example}
\begin{solution}
	$\forall x\in\mathbb{R}\setminus\left\{-\dfrac{1}{2}\right\}$,  我们均有
	\abovedisplayskip=0pt
	\begin{equation*}
	\lim_{n\rightarrow\infty}\left|\frac{(-1)^n}{n}\left(\frac{1}{2x+1}\right)^n\right|^{\frac{1}{n}}
	=\lim_{n\rightarrow\infty}\frac{1}{\sqrt[n]{n}|2x+1|} 
	= \frac{1}{|2x+1|}
	\end{equation*}
	由根值判别法可知,  原级数在$|2x+1|>1$也即
	$x>0$ 或$x<-1$时收敛,  而$x\in (-1,0)$ 时发散;

	当$x=0$ 时,原级数变为$ \sum\limits_{n=1}^{\infty} \dfrac{(-1)^n}{n}$,  则由Leibniz判别法可知它收敛;

	当$x=-1$时,原级数变为 $ \sum\limits_{n=1}^{\infty}\dfrac{1}{n}$,  发散。

	故收敛域为$(-\infty,-1)\cup [0,+\infty)$。
\end{solution}
\begin{example}
	\textbf{收敛但不一致收敛的函数项级数。}几何级数$ \sum\limits_{n=1}^{\infty}x^{n-1}$为$\mathbb{R}$上的函数项级数,它的收敛域为$(-1,1)$,  而和函数为$S(x)=\dfrac{1}{1-x}$。$\forall n\ge 1$,  我们有
	\begin{equation*}
	\sup_{x\in (-1,1)}\left|\sum_{k=1}^{n}x^{k-1}-S(x)\right|
	=\sup_{x\in (-1,1)}\left|\frac{1-x^n}{1-x}-\frac{1}{1-x}\right|
	=\sup_{x\in (-1,1)}\frac{|x|^n}{|1-x|} 
	=+\infty
	\end{equation*}
	由此可知几何级数$\sum\limits_{n=1}^{\infty}x^{n-1}$在$(-1,1)$上收敛, 但在$(-1,1)$上不为一致收敛。
\end{example}

\di[Weierstrass 判别法]{
	若存在非负常数项收敛级数$ \sum\limits_{n=1}^{\infty} M_n$使得$\forall n\ge 1$ 以及$\forall x\in J$,均有
	$|u_n(x)|\le M_n$,则函数项级数$ \sum\limits_{n=1}^{\infty}u_n(x)$在$J$上绝对收敛且一致收敛。通常称$ \sum\limits_{n=1}^{\infty}M_n$为级数$ \sum\limits_{n=1}^{\infty}u_n(x)$的\tbome{控制级数}。
}
\begin{proofs}
	$\forall x\in J$,  由比较判别法知级数$ \sum\limits_{n=1}^{\infty}u_n(x)$绝对收敛。又$ \sum\limits_{n=1}^{\infty} M_n$ 收敛, 则$ \lim\limits_{n\rightarrow\infty}\sum\limits_{k=n+1}^{\infty}  M_k=0$。
	
	但$\forall n\ge 1$,  我们有 
	\begin{equation*}
		0\le \sup_{x\in J}\left|\sum_{k=n+1}^{\infty}u_k(x)\right| 
		\le \sum_{k=n+1}^{\infty}\sup_{x\in J}|u_k(x)| 
		\le \sum_{k=n+1}^{\infty}M_k
	\end{equation*}
	于是由夹逼原理可得$ \lim\limits_{n\rightarrow\infty}\sup\limits_{x\in J}\left|\sum\limits_{k=n+1}^{\infty}u_k(x)\right| = 0$。因此所证结论成立。
\end{proofs}

\begin{example}
	证明:$ \sum\limits_{n=1}^{\infty} x^2\e^{-nx}$ 在$[0,+\infty)$ 上一致收敛。
\end{example}
\begin{proofs}
	$\forall n\ge 1$ 以及$\forall x\in\mathbb{R}$,  令$u_n(x)=x^2\e^{-nx}$,  则
	$u_n'(x)=2x\e^{-nx}-nx^2\e^{-nx} =(2-nx)x\e^{-nx}$,
	故$u_n'$ 在$\left(0,\dfrac{2}{n}\right)$ 上严格正而在$\left(\dfrac{2}{n},+\infty\right)$ 上严格负, 
	则$u_n$在$[0,+\infty)$上的最大值点为$x=\dfrac{2}{n}$,  也即
	$\forall x\ge 0$,  我们有$0\le u_n(x)\le \dfrac{4}{n^2}\e^{-2}$。
	而$ \sum\limits_{n=1}^{\infty}\dfrac{4}{n^2}\e^{-2}$收敛,  于是由Weierstrass判别法可知原函数项级数在$[0,+\infty)$上一致收敛。	
\end{proofs}

\di[函数项级数的Dirichlet判别准则]{
	如 果在某个区间 $I$ 上,  函 数 项 级 数$ \sum\limits_{n=1}^{\infty}u_n$的 部 分和函数列为\textbf{一致有界},而 函 数 列$\{v_n\}$\textbf{单调且一致趋于\,0},则$ \sum\limits_{n=1}^{\infty}u_nv_n$在$I$上一致收敛。
}
\di[函数项级数的Abel判别准则]{
	如果 在某个区间 $I$ 上,  函 数 项 级 数$ \sum\limits_{n=1}^{\infty}u_n$为\textbf{一  致  收  敛},而  函数列$\{v_n\}$\textbf{单调并且一致有界},则函数项级数$ \sum\limits_{n=1}^{\infty}u_nv_n$在$I$上一致收敛。
}
\begin{example}
	证明: $\sum\limits_{n=1}^{\infty}\dfrac{\sin (nx)}{n}$在$[\delta,2\pi-\delta]$上一致收敛,其中我们假设$\delta\in (0, \pi)$。
\end{example}
\begin{proofs}
	\adjline
	$\forall n\ge 1$以及$\forall x\in [\delta,2\pi-\delta]$,我们有
	$$
	\left|\sum\limits_{k=1}^n\sin (kx)\right|
	\le \left|\cfrac{\cos\left(n+\frac{1}{2}\right)x-\cos \frac{x}{2}}{2\sin\frac{x}{2}}\right|
	\le \frac{1}{\sin\frac{x}{2}} \le \frac{1}{\sin\frac{\delta}{2}}
	$$
	而$\left\{\dfrac{1}{n}\right\}$单调且一致趋于$0$,于是由Dirichlet判别准则知,原函数项级数在$[\delta,2\pi-\delta]$上一致收敛。
\end{proofs}

\subsection{一致收敛的函数项级数的和函数的性质}



\di[连续函数列一致收敛极限函数的连续性]{
	设$I\subseteq\mathbb{R}$为非空集合,而$\{v_n\}$为定义在$I$上的连续函数列,并且在$I$上一致收敛到函数$v$,则$v$在$I$上连续。
}
\begin{proofs}
	\adjline
	固定$x_0\in I$,$\forall \varepsilon>0$, 由一致收敛性可知,
	$\exists N>0$ 使得$\forall n\ge N$以及$\forall x\in I$, 均有
	$$|v_n(x)-v(x)|<\frac{\varepsilon}{3}$$
	由于$v_N$在点$x_0$连续, 则$\exists \delta>0$ 使得$\forall x\in I$,
	当$|x-x_0|<\delta$时, 均有$|v_N(x)-v_N(x_0)|<\dfrac{\varepsilon}{3}$	。由此,
	$$
		|v(x)-v(x_0)|
		\le \underbrace{|v(x)-v_N(x)|}_\text{一致收敛性}
		+\underbrace{|v_N(x)-v_N(x_0)|}_{v_N\text{的连续性}}+\underbrace{|v_N(x_0)-v(x_0)|}_\text{一致收敛性}<\varepsilon
	$$
	故$v$在点$x_0$连续,由$x_0$的任意性知所证成立。 		
\end{proofs}

\di[一致收敛连续函数列的数列极限与函数极限可交换性]{
	若连续函数列$\{v_n(x)\}$在区间$I$上一致收敛到$v(x)$,则对任意$x_0 \in I$,有
	\begin{equation}
		{\color{meihong!50!black}\boldsymbol{\lim_{I\ni x\rightarrow x_0}}}  {\colors{meihong}\boldsymbol{\lim_{n\rightarrow \infty}}} v_n(x)
		={\colors{meihong}\boldsymbol{\lim_{n\rightarrow\infty}}}  {\color{meihong!50!black}\boldsymbol{\lim_{I\ni x\rightarrow x_0}}}v_n(x)			
	\end{equation}
}
\begin{proofs}
	$
		{\color{meihong!50!black}\boldsymbol{\lim\limits_{I\ni x\rightarrow x_0}}}  {\colors{meihong}\boldsymbol{\lim\limits_{n\rightarrow \infty}}} v_n(x)
		=\lim\limits_{I\ni x\rightarrow x_0}v(x)
		=v(x_0)
		=\lim\limits_{n\rightarrow\infty}v_n(x_0)
		={\colors{meihong}\boldsymbol{\lim\limits_{n\rightarrow\infty}}}  {\color{meihong!50!black}\boldsymbol{\lim\limits_{I\ni x\rightarrow x_0}}}v_n(x)			
	$。
\end{proofs}

\begin{corollary}
	如果定义在$(a,b)$上的连续函数列$\{v_n(x)\}$
	在$(a,b)$内的\textbf{任意闭子区间}上一致收敛到函数$v(x)$(这种性质称为\textbf{内闭一致收敛}),则函数$v(x)$在区间$(a,b)$上连续
	且为上述函数列在$(a,b)$上的极限函数。
\end{corollary}

\di[一致收敛连续函数项级数的极限与级数求和可交换性]{
	假设$I\subseteq\mathbb{R}$为非空集合,而$\{u_n\}$为$I$上的连续
	函数组成的函数列,使函数项级数$ \sum\limits_{n=1}^{\infty}u_n$在$I$上
	一致收敛到函数$S$,则$S$在$I$上连续。此即,对任意$x_0 \in I$,
	\begin{equation}
		{\color{meihong!50!black}\boldsymbol{\lim_{I\ni x\rightarrow x_0}}}  {\colors{meihong}\boldsymbol{\sumi}} v_n(x)
		={\colors{meihong}\boldsymbol{\sumi}}  {\color{meihong!50!black}\boldsymbol{\lim_{I\ni x\rightarrow x_0}}}v_n(x)
	\end{equation}
}
\begin{proofs}
	\adjline[5]
	$\forall n\ge 1$以及$\forall x\in I$,令$ S_n(x)=\sum\limits_{k=1}^n u_k(x)$,则函数列$\{S_n\}$在$I$上连续  且在$I$上一致收敛到函数$S$,故$S$在$I$上连续。 或,对任意$x_0 \in I$,
	\begin{align*}
		{\color{meihong!50!black}\boldsymbol{\lim_{I\ni x\rightarrow x_0}}}  {\colors{meihong}\boldsymbol{\sumi}} v_n(x)
		&={\color{meihong!50!black}\boldsymbol{\lim_{I\ni x\rightarrow x_0}}}  {\colors{meihong}\boldsymbol{\lim_{n\rightarrow \infty}}} S_n(x)
		=\lim\limits_{I\ni x\rightarrow x_0}S(x)
		=S(x_0)\\[-4pt]
		&=\sumi v_n(x_0)
		={\colors{meihong}\boldsymbol{\sumi}}  {\color{meihong!50!black}\boldsymbol{\lim_{I\ni x\rightarrow x_0}}}v_n(x)
		\qedhere
	\end{align*}
\end{proofs}

\di[一致收敛连续函数项级数的积分与级数求和可交换性]{
	假设$\{u_n\}$为$[a,b]:=I$上的连续函数组成的函数列,使得函数项级数$ \sum\limits_{n=1}^{\infty}u_n$在$[a,b]$上一致收敛到函数$S$,则$S$在$I$上连续且$\forall x\in [a,b]$,均有
	\begin{equation}
		\int_a^x S(t) \dif t ={\color{meihong!50!black}\boldsymbol{\int_a^x}}  {\colors{meihong}\boldsymbol{\sumi}} v_n(t) \dif t
		={\colors{meihong}\boldsymbol{\sumi}}  {\color{meihong!50!black}\boldsymbol{\int_a^x}}v_n(t) \dif t
	\end{equation}
	并且右边作为变量$x$的函数项级数在$[a,b]$上一致收敛。
}
\begin{proofs}
	\adjline[5]
	$\forall \varepsilon>0$,由题设条件立刻知,$\exists N>0$使得$\forall m>N$以及$\forall t\in [a,b]$,均有
	$\left|\sum\limits_{n=1}^mu_n(t)-S(t)\right|<\dfrac{\varepsilon}{b-a+1}$。
	于是$\forall x\in [a,b]$,我们有
	\begin{align*}
	\left|\sum_{n=1}^{m}\int_a^xu_n(t)\dif t-\int_a^x S(t)\dif t\right|
	&=\left|\int_a^x\left(\sum_{n=1}^{m} u_n(t)-S(t)\right)\dif t\right| \\[-4pt]
	&\le \int_a^x\left|\sum_{n=1}^{m} u_n(t)-S(t)\right|\dif t 
	\le \int_a^x\frac{\varepsilon}{b-a+1} \dif t 
	<\varepsilon
	\end{align*}
	因此所证结论成立。
\end{proofs}

\di[一致收敛连续函数项级数的求导与级数求和可交换性]{
	设$\{u_n\}$为$(a,b)$上的连续可导函数列。假设

	(1)$\exists x_0\in (a,b)$使得级数$ \sumi u_n(x_0)$收敛;
	
	(2)函数项级数$ \sumi \dfrac{\dif u_n(x)}{\dif x}$在$(a,b)$上一致收敛,
	
	那么$ \sumi u_n$在$(a,b)$上\tbome{内闭}一致收敛,和函数$S$在$(a,b)$上连续可导,且$\forall x\in (a,b)$,我们有
	\begin{equation}
		{\colors{meihong}\boldsymbol{\sumi}}  {\color{meihong!50!black}\boldsymbol{\dfrac{\dif}{\dif x}}}u_n(x)
		={\color{meihong!50!black}\boldsymbol{\dfrac{\dif}{\dif x}}}  {\colors{meihong}\boldsymbol{\sumi}} u_n(x)
		=\dfrac{\dif}{\dif x}S(x)
	\end{equation}
}
\begin{proofs}
	\adjline
	由于$\{u_n'\}$为区间$(a,b)$上的连续函数列,
	而函数项级数$ \sum\limits_{n=1}^{\infty}u_n'$在$(a,b)$上一致收敛,则
	由极限与级数求和可交换性可知,它的和函数$\sigma=\sum\limits_{n=1}^{\infty}u_n'$在$(a,b)$上连续;进而再利用积分与级数求和可交换性可知,$\forall x\in (a,b)$,我们有
	\begin{equation*}
	\int_{x_0}^x\sigma(t)\dif t=\sum_{n=1}^{\infty}\int_{x_0}^xu_n'(t)\dif t 
	=\sum_{n=1}^{\infty}\left(u_n(x)-u_n(x_0)\right)
	\end{equation*}
	且右边的函数项级数在$(a,b)$上内闭一致收敛。而其中由题设可知,级数$ \sum\limits_{n=1}^{\infty}u_n(x_0)$收敛,则由级数的
	线性性,可知$\sum\limits_{n=1}^{\infty}u_n(x)$也收敛,设其和为$S(x)$,  
	故$ \dint_{x_0}^x\sigma(t)\dif t=S(x)-S(x_0)$。  又$\sigma$连续,于是
	$S$可导,且$  S'(x)=\sigma(x)=\sum\limits_{n=1}^{\infty}u_n'(x)$,从而$S$为连续可导函数,故所证结论成立。
\end{proofs}

\begin{example}
	证明:$  S(x)=\sum\limits_{n=1}^{\infty}\dfrac{\cos (nx)}{n^2} \in \mathscr{C}^{(1)}(0,2\pi)$。
\end{example}
\begin{proofs}
	$\forall n\ge1$以及$\forall x\in(0,2\pi)$,
	令$u_n(x)=\dfrac{\cos (nx)}{n^2}$,
	则$u_n$在$(0,2\pi)$上连续可导,且$u'_n(x)=-\dfrac{\sin (nx)}{n}$。  
	又$ \sum\limits_{n=1}^{\infty}u'_n$在$(0,2\pi)$上内闭一致收敛  并且常数项级数$ \sumi u(\pi) =\sum\limits_{n=1}^{\infty}\dfrac{\cos (n\pi) }{n^2}
	=\sum\limits_{n=1}^{\infty}\dfrac{(-1)^n}{n^2}$收敛,
	因此和函数$S$在$(0,2\pi)$上连续可导。
\end{proofs}

\subsection{幂级数}

\subsubsection{幂级数的收敛半径}

\de[幂级数]{
	设$\{a_n\}$为常数项数列,而$x_0\in\mathbb{R}$,我们称$\sum\limits_{n=0}^{\infty}a_n(x-x_0)^n$形如的函数项级数为\tboba{幂级数}。   
}
出于简便,我们通常取$x_0=0$,即$\sum\limits_{n=0}^{\infty}a_n x^n$。一般情形可由此特殊情形通过平移而得到。

\di[Abel定理]{
	设$x_0\in\mathbb{R}\setminus\{0\}$,
	$\{a_n\}$为常数项数列,若$\{a_nx_0^n\}$有界,
	则幂级数$  \sum\limits_{n=0}^{\infty}a_nx^n$
	在$\left(-|x_0|,|x_0|\right)$内绝对收敛且内闭一致收敛。
}
\begin{proofs}
	\adjline
	(1)证绝对收敛:由题设可知,$\exists M>0$   使得$\forall n\ge 1$,均有$|a_nx_0^n|\le M$。   从而$\forall x\in (-|x_0|,|x_0|)$,我们有
	\begin{equation*}
	|a_nx^n|=|a_nx_0^n|\left|\frac{x}{x_0}\right|^n  \le {M}\left|\frac{x}{x_0}\right|^n
	\end{equation*}
	又在$\left(-|x_0|,|x_0|\right)$内$\left|\dfrac{x}{x_0}\right|<1$,则由比较判别法知$  \sum\limits_{n=0}^{\infty}|a_nx^n|$收敛。

	(2)证内闭一致收敛:取任意的闭区间$[a,b]\subset (-|x_0|,|x_0|)$,尝试证明级数在$[a,b]$内一致收敛。但对于这样的 $[a,b]$,必定存在一个常数 $r\in (0,|x_0|)$,
	使得
	$[a,b]\subset [-r,r]\subset  (-|x_0|,|x_0|)$,
	进而只需证明 
	幂级数$ \sum\limits_{n=0}^{\infty}a_nx^n$在$[-r,r]$内一致收敛即可。

	固定这样的$r\in (0,|x_0|)$。   $\forall x\in [-r,r]$,我们有
	\begin{equation*}
	|a_nx^n|=|a_nx_0^n|\left|\frac{x}{x_0}\right|^n  \le M\left|\frac{x}{x_0}\right|^n \le M\left|\frac{r}{x_0}\right|^n
	\end{equation*}
	又因为$\left|\dfrac{r}{x_0}\right|<1$,于是由Weierstrass判别法可知
	幂级数$  \sum\limits_{n=0}^{\infty}a_nx^n$在$[-r,r]$上一致收敛,进而知
	该幂级数在$(-|x_0|,|x_0|)$的任意的闭子区间上一致收敛,即内闭一致收敛。
\end{proofs}
\begin{corollary}
	如果幂级数$  \sum\limits_{n=0}^{\infty}a_nx^n$在点$x_0\in\mathbb{R}\setminus\{0\}$处——
	
	(1)收敛,那么它在$(-|x_0|,|x_0|)$内绝对收敛且内闭一致收敛; 

	(2)发散,那么它在$\bigl[-|x_0|,|x_0|\bigr]$\textbf{外}发散,即$\forall x\in\mathbb{R}$,若$|x|>|x_0|$,则级数$  \sum\limits_{n=0}^{\infty}a_nx^n$发散。
\end{corollary}

由此,幂级数$  \sum\limits_{n=0}^{\infty}a_nx^n$的收敛域
是一个区间。事实上,只能有以下三种可能性:

(1)仅在点$x=0$处收敛; 

(2)在任意点$x\in\mathbb{R}$收敛; 

(3)$\exists R>0$,使得$|x|<R$时幂级数在点$x$处收敛,而$|x|>R$时幂级数在点$x$处发散;至于在点$x=\pm R$处,幂级数可为收敛或发散。 

\de[收敛半径]{
	幂级数收敛域的半径称为它的\tboba{收敛半径}。

	即,$R\in[0,+\infty]$恰好为幂级数$\sum\limits_{n=0}^{\infty}a_nx^n$的收敛半径当且仅当下列性质都成立:
	
	(1)当$|x|<R$时,级数$  \sum\limits_{n=0}^{\infty}a_nx^n$收敛;  
	
	(2)当$|x|>R$时,级数$  \sum\limits_{n=0}^{\infty}a_nx^n$发散。 
	
	我们称$(-R,R)$为\tboba{收敛开区间}。
}
在收敛开区间基础上,为得到收敛域,还需讨论幂级数在点$x=\pm R$处的收敛性。

在上述三种情形,幂级数的收敛半径分别为$0$,$+\infty$和$R$。

\di[收敛半径的求出]{
	幂级数$\sum\limits_{n=0}^{\infty}a_nx^n$的收敛半径为$R=\dfrac{1}{\rho}$,其中
	\begin{equation}
		\rho=\limsup\limits_{n\rightarrow\infty}\sqrt[n]{|a_n|}
	\end{equation}
	且约定$\dfrac{1}{0}=+\infty$,$\dfrac{1}{+\infty}=0$。
}
\begin{proofs}
	$\forall x\in\mathbb{R}$,我们有$\limsup\limits_{n\rightarrow\infty}\sqrt[n]{|a_n||x|^n}=\rho |x|$。 

	则当$|x|<R$时,由根值判别法立刻可知,级数$\sum\limits_{n=0}^{\infty}a_nx^n$收敛;

	而当$|x|>R$时,$\limsup\limits_{n \to \infty} |a_nx^n| = +\infty$,即通项$a_nx^n$不收敛到$0$,级数$\sum\limits_{n=0}^{\infty}a_nx^n$发散。

	故$R$为幂级数$\sum\limits_{n=0}^{\infty}a_nx^n$的收敛半径。
\end{proofs}
\begin{corollary}
	幂级数$\sum\limits_{n=0}^{\infty}a_nx^n$的收敛半径为$R=\dfrac{1}{\rho}$,若有下面条件之一:

	(1)极限$\rho=\lim\limits_{n\rightarrow\infty}\sqrt[n]{|a_n|}$存在或为$+\infty$;

	(2)极限$\rho=\lim\limits_{n\rightarrow\infty}\dfrac{|a_{n+1}|}{|a_n|}$存在或者为$+\infty$。
\end{corollary}

\begin{example}
	求幂级数$\sum\limits_{n=1}^{\infty}(-1)^n\frac{(2x)^n}{\sqrt{n}}$的收敛域。 
\end{example}
\begin{solution}
	\adjline
	由题设可知  所求收敛半径为
	\begin{equation*}
	R=\lim_{n\rightarrow\infty}\left|\dfrac{(-1)^n\dfrac{2^n}{\sqrt{n}}}{(-1)^{n+1}\dfrac{2^{n+1}}{\sqrt{n+1}}}\right| 
	=\lim_{n\rightarrow\infty}\frac{\sqrt{n+1}}{2\sqrt{n}} 
	=\frac{1}{2}
	\end{equation*}
	在$x=\dfrac{1}{2}$处,幂级数变为$\sum\limits_{n=1}^{\infty}\dfrac{(-1)^n}{\sqrt{n}}$,而由Leibniz判别法可知该级数收敛。  在$x=-\dfrac{1}{2}$处,幂级数变为$\sum\limits_{n=1}^{\infty}\dfrac{1}{\sqrt{n}}$,
	该级数发散。故收敛域为$\left(-\dfrac{1}{2},\!\dfrac{1}{2}\right]$。
\end{solution}

\di[Abel第二定理]{
	若幂级数$\sum\limits_{n=0}^{\infty} a_nx^n$的收敛半径为$R\in (0,+\infty)$并且在$x=R$处收敛,则$\forall r\in (0,R)$,幂级数在$[-r,R]$上\tbome{一致收敛}。
}
\begin{proofs}
	固定$r\in (0,R)$。  $\forall x\in [r,R]$,定义
	$u_n(x)=a_nR^n$,$v_n(x)=\left(\dfrac{x}{R}\right)^n$,
	那么
	
	(1)函数项级数$\sum\limits_{n=1}^{\infty}u_n(x)$是常数项级数,关于$x\in [r,R]$一致收敛;
	
	(2)函数列$\{v_n\}$单调,且我们还有$|v_n|\le 1$。
	
	从而,由Abel判别准则(定理~\ref{函数项级数的Abel判别准则})立刻可知,幂级数
	$\sum\limits_{n=0}^{\infty}a_nx^n$
	在$[r,R]$上为一致收敛。  
	
	而又由Abel定理(定理~\ref{Abel定理})可知,上述幂级数在$[-r,r]$上一致收敛,于是该幂级数在$[-r,R]$上一致收敛。
\end{proofs}

\subsubsection{幂级数的性质}

\di[幂级数的四则运算性质]{
	设幂级数$\sum\limits_{n=0}^{\infty}a_nx^n,\sum\limits_{n=0}^{\infty}b_nx^n$的收敛半径分别为$R_1>0,R_2>0$,令$R=\min(R_1,R_2)$,则

	(1)\tbome{线性性}\quad $\forall \lambda,\mu\in\mathbb{R}$以及$\forall x\in (-R,R)$,
	\begin{equation}
		\lambda\sum_{n=0}^{\infty}a_nx^n+\mu\sum_{n=0}^{\infty}b_nx^n
		=\sum_{n=0}^{\infty}(\lambda a_n+\mu b_n)x^n
	\end{equation}
	右边的收敛半径在$R_1\neq R_2,\lambda\mu\neq 0$时等于$R$,但当$R_1=R_2$时,却有可能严格大于$R$;

	(2)\tbome{乘法}\quad $\forall x\in (-R,R)$,均有
	\begin{equation}
	\left(\sum_{m=0}^{\infty}a_mx^m\right)\left(\sum_{n=0}^{\infty}b_nx^n\right)
	=\sum_{k=0}^{\infty}\left(\sum_{i+j=k}a_ib_j\right)x^k
	\end{equation}
	右边幂级数的收敛半径可严格大于$R$;

	(3)\tbome{除法}\quad 当$b_0\neq 0$时,在原点的某个邻域内:  
	$\dfrac{\sum\limits_{n=0}^{\infty}a_nx^n}{\sum\limits_{i=0}^{\infty}b_ix^i}=\sum\limits_{j=0}^{\infty}c_jx^j$,其中$\forall n\ge 0$,系数$c_j$由下式定义: 
	\begin{equation}
	a_n=\sum_{i+j=n}b_ic_j =\sum_{i=0}^nb_ic_{n-i}
	\end{equation}
	由此我们可以递归地确定$c_n$,即$c_n=\dfrac{1}{b_0}\left(a_n-\sum\limits_{i=1}^{n}b_ic_{n-i}\right)$。
}
\di[幂级数的逐项积分性质]{
	\adjline
	若$\sum\limits_{n=0}^{\infty}a_nx^n$的收敛半径为$R>0$,则其
	和函数$S\in\mathscr{C}(-R,R)$且$\forall x\in (-R,R)$,均有
	\begin{equation}
	\int_0^xS(t)\dif t=\sum_{n=0}^{\infty}a_n\int_0^xt^n\dif t
	=\sum_{n=0}^{\infty}\frac{a_n}{n+1}x^{n+1}
	\end{equation}
	并且右边幂级数的收敛半径依然为$R$。
}
\begin{proofs}
	\adjline
	任取$x\in (-R,R)$,于是$\exists r\in (0,R)$  使得$x\in (-r,r)$。  由Abel定理可知,幂级数$\sum\limits_{n=0}^{\infty}a_nx^n$在$[-r,r]$上一致收敛,并且其通项为连续函数。则由极限与级数求和可交换性可知,和函数$S$
	在$[-r,r]$上连续。特别地,它在点$x$处连续。 
	于是$S\in\mathscr{C}(-R,R)$,随后再由积分与级数求和可交换性立刻可得
	\begin{equation*}
	\int_0^xS(t)\dif t
	=\sum_{n=0}^{\infty}a_n\int_0^xt^n\dif t 
	=\sum_{n=0}^{\infty}\frac{a_n}{n+1}x^{n+1}
	\end{equation*}
	由根值判别法,右边幂级数收敛半径的倒数为
	\begin{equation*}
	\limsup_{n\rightarrow\infty}{\sqrt[n+1]{\frac{|a_n|}{n+1}}} 
	=\limsup_{n\rightarrow\infty}\frac{{\sqrt[n+1]{|a_n|}}}{\sqrt[n+1]{n+1}} 
	=\frac{1}{R} \qedhere
	\end{equation*}
\end{proofs}
\di[幂级数的逐项求导性质]{
	\adjline
	如果$\sum\limits_{n=0}^{\infty}a_nx^n$的
	收敛半径为$R>0$,那么
	其和函数$S\in\mathscr{C}^{(\infty)}(-R,R)$,并且$\forall x\in (-R,R)$
	以及$\forall k\ge 0$,我们有
	\begin{equation}
		S^{(k)}(x)=\sum_{n=k}^{\infty}n(n-1)\cdots (n-k+1)a_nx^{n-k}
	\end{equation}
	并且右边幂级数的收敛半径依然为$R$。
}
\begin{proofs}
	\adjline
	对$k\ge 0$应用数学归纳法,往证:$S\in\mathcal{C}^{(k)}(-R,R)$并且$S^{(k)}$满足上述等式。

	当$k=0$时,由前面定理可知此时所证成立。 
	假设所证结论对$k\ge 0$成立。  $\forall n\ge k$,令
	\begin{equation*}
	u_n(x)=n(n-1)\cdots (n-k+1)a_nx^{n-k}
	\end{equation*}
	则$u_n$在$\mathbb{R}$上可导,并且$u_k'\equiv 0$。而当$n>k$时,
	\begin{equation*}
	u_n'(x)=n(n-1)\cdots (n-k+1)(n-k)a_nx^{n-k-1}
	\end{equation*}
	再注意到我们有
	\begin{equation*}
	\limsup_{n\rightarrow\infty}\left(n(n-1)\cdots (n-k+1)(n-k)|a_n|\right)^{\frac{1}{n-k-1}}
	\xlongequal{\lim\limits_{n\to\infty}(n-i)^{\frac{1}{n-k-1}} = 1}
	\limsup_{n\rightarrow\infty}\sqrt[n-k-1]{|a_n|} =1/R
	\end{equation*}
	于是知幂级数$\sum\limits_{n=k+1}^\infty u_n'(x)$的收敛半径为$R$,即由前面定理可知,它的和函数在$(-R,R)$上连续 
	并且$\forall x\in (-R,R)$,我们均有
	\begin{equation*}
	\int_0^x\left( \sum_{n=k+1}^{\infty}u_n'(t)\right)\dif t
	=\sum_{n=k+1}^{\infty}\int_0^xu_n'(t)\dif t
	=\sum_{n=k+1}^{\infty}\left(u_n(x)-u_n(0)\right) 
	=\sum_{n=k+1}^{\infty}u_n(x)
	\end{equation*}
	由归纳假设可知,右边等于$S^{(k)}(x)-k!a_k$,即$S^{(k)}(x)=k!a_k+\dint_0^x\left( \textstyle\sum\limits_{n=k+1}^{\infty}u_n'(t)\right)\dif t$,进而可知$S^{(k)}$可导,并且$\forall x\in (-R,R)$,我们均有
	\begin{equation*}
	S^{(k+1)}(x)=\left(S^{(k)}-k!a_k\right)'(x) =\sum_{n=k+1}^{\infty}u_n'(x)
	\end{equation*}
	由此可知要证的结论对$k+1$成立,进而由数学
	归纳法可知要证的结论对所有$k\ge 0$均成立。
\end{proofs}
\ref{幂级数的逐项积分性质}~和~\ref{幂级数的逐项求导性质}~这两个定理表明,幂级数在其收敛域的内部可进行\textbf{任意多次}积分和求导,这些运算与级数求和运算\textbf{可交换次序}且\textbf{不改变收敛半径}。

\begin{example}
	求幂级数$\sum\limits_{n=1}^{\infty}(-1)^{n+1}n(n+1)x^n$的收敛域以及和函数,并由此计算$\sum\limits_{n=1}^{\infty}(-1)^{n+1}\dfrac{n(n+1)}{2^n}$。
\end{example}
\begin{solution}
	\adjline
	题设幂级数的收敛半径等于$\lim\limits_{n\rightarrow\infty} \dfrac{1}{\sqrt[n]{|(-1)^{n+1}n(n+1)|}}=1$。 

	$\forall x\in (-1,1)$,我们已知$\dfrac{1}{1+x}=\sum\limits_{n=0}^{\infty}(-1)^nx^n$,且该幂级数的收敛半径为$1$。于是,由幂级数求导与求和可交换性可知
	$-\dfrac{1}{(1+x)^2}=\sum\limits_{n=1}^{\infty}(-1)^n nx^{n-1}$,
	进而
	$\dfrac{x^2}{(1+x)^2}=\sum\limits_{n=1}^{\infty}(-1)^{n+1} nx^{n+1}$,
	再求导有
	$\sum\limits_{n=1}^{\infty}(-1)^{n+1} (n+1)nx^{n} = \dfrac{2x}{(1+x)^3}$,收敛半径不变。于是所求$\sum\limits_{n=1}^{\infty}(-1)^{n+1}\dfrac{n(n+1)}{2^n} = \sum\limits_{n=1}^{\infty}(-1)^{n+1} (n+1)n\left(\dfrac{1}{2}\right)^{n} = \dfrac{8}{27}$。
\end{solution}

\subsubsection{幂级数展开}

设$R>0$,$x_0\in\mathbb{R}$,给定区间$(x_0-R, x_0+R)$上的函数$f$。类比一元微分学的Taylor多项式,我们寻求这样的幂级数$\sum\limits_{n=0}^\infty a_n(x-x_0)^n$,使得其在$(x_0-R, x_0+R)$上收敛到$f(x)$。
\de[函数的Taylor级数]{
	设$R>0$,$x_0\in\mathbb{R}$,给定区间$(x_0-R, x_0+R)$上的函数$f$。若存在幂级数$\sum\limits_{n=0}^\infty a_n(x-x_0)^n=f(x)$,则称幂级数$\sum\limits_{n=0}^\infty a_n(x-x_0)^n$为$f$在$x_0$处的~\tboba{Taylor级数}。当$x_0=0$时,该幂级数也称为~\tboba{Maclaurin级数}。
}

如果展式$f(x)=\sum\limits_{n=0}^\infty a_n(x-x_0)^n$成立,则由幂级数的性质可知$\mboba{f\in \mathscr{C}^{(\infty)}(x_0-R,x_0+R)}$,且$\forall k\ge 0$
以及$\forall x\in (x_0-R,x_0+R)$,我们均有
\begin{equation}
	f^{(k)}(x)=\sum_{n=k}^{\infty}n(n-1)\cdots (n-k+1)a_n(x-x_0)^{n-k}
\end{equation}
特别地,我们有$f^{(k)}(x_0)=k!a_k$。由此我们可知,如果函数$f$在点$x_0$处有Taylor级数展开,那么它的系数可由$f$来唯一确定。

\zhu[特例]{
	$f$在点$x_0$的邻域内为$\mathscr{C}^{(\infty)}$类并不意味着$f$ 在点$x_0$处有Taylor级数展开。
	
	考虑函数
	$
	f(x)=\begin{cases}[ll]
		\exp\left(-\dfrac{1}{x^2}\right),&x\neq 0,\\[-6pt]
		0,&x=0
	\end{cases} \in \mathscr{C}^{(\infty)}
	$,但$\forall n\ge0$,
	均有$f^{(n)}(0)=0$,这表明$f$不能在原点处展开成Taylor级数。
}

假设$f$在点$x_0$的邻域内为$\mathcal{C}^{(\infty)}$类,我们形式地记
\begin{equation}
f(x)\sim \sum_{n=0}^{\infty}\frac{f^{(n)}(x_0)}{n!}(x-x_0)^n
\end{equation}
并将右边称为$f$在点$x_0$的\textbf{Taylor级数}。

下面记Taylor级数的部分和$\sum\limits_{n=0}^{N}\dfrac{f^{(n)}(x_0)}{n!}(x-x_0)^n =: T_N(x)$。$\forall n\ge 1$,由带Lagrange余项的Taylor公式可知,
存在$\xi_{n+1}$介于$x_0,x$之间,使得
\begin{equation}
f(x)={\color{meihong!50!black}\boldsymbol{\underbrace{\sum_{k=0}^n\frac{f^{(k)}(x_0)}{k!}(x-x_0)^k}_{T_n(x)}} }+\mbome{ \underbrace{\frac{f^{(n+1)}(\xi_{n+1})}{(n+1)!}(x-x_0)^{n+1}}_{r_n(x)} } 
\end{equation}
于是我们有
\di[函数展成Taylor级数的充要条件]{
	假设$f\in \mathscr{C}^{(\infty)}(x_0-R,x_0+R)$,那么$f$在点$x_0$处的Taylor级数在$(x_0-R,x_0+R)$内收敛到$f$~\tbome{当且仅当}~$f$在点$x_0$处的Taylor展式余项$r_n(x)$随$n\rightarrow \infty$而趋于$0$。
}
\begin{corollary}
	$\forall x\in(x_0-R,x_0+R)$,如果存在$N>0$及$M>0$使得$\forall n>N$,$\boldsymbol{|f^{(n+1)}(\xi_{n+1})|\le M}$,则
	$
	f(x)=\sum\limits_{k=0}^{\infty}\dfrac{f^{(k)}(x_0)}{k!}(x-x_0)^k
	$。
\end{corollary}
\begin{corollary}
	假设$f\in \mathscr{C}^{(\infty)}(x_0-R,x_0+R)$,若存在整数$N>0$以及$M>0$,使得对任意整数$n>N$以及对任意$\xi\in (x_0-R,x_0+R)$,均有
	$\boldsymbol{|f^{(n+1)}(\xi)|\le M}$,
	则$\forall x\in(x_0-R,x_0+R)$,我们有
	$f(x)=\sum\limits_{k=0}^{\infty}\dfrac{f^{(k)}(x_0)}{k!}(x-x_0)^k$。
\end{corollary}

\di[常用函数的Taylor级数展开]{
	(1)$\mbome{\e^x=\sum\limits_{n=0}^{\infty}\dfrac{x^n}{n!}},x\in\mathbb{R}$。

	\hang[2](2)$\mbome{\sin x=\sum\limits_{n=0}^{\infty}(-1)^n\dfrac{x^{2n+1}}{(2n+1)!}},x\in\mathbb{R}$;\quad
	$\mbome{\cos x=\sum\limits_{n=0}^{\infty}(-1)^n\dfrac{x^{2n}}{(2n)!}},x\in\mathbb{R}$;\\
	$\sinh x=\dfrac{1}{2}(\e^x-\e^{-x})=\sum\limits_{n=0}^{\infty}\dfrac{x^{2n+1}}{(2n+1)!},x\in\mathbb{R}$;\quad
	$\cosh x=\dfrac{1}{2}(\e^x+\e^{-x})=\sum\limits_{n=0}^{\infty}\dfrac{x^{2n}}{(2n)!},x\in\mathbb{R}$。

	\hang[2](3)$\mbome{f(x)=(1+x)^{\alpha}=\sum\limits_{n=0}^{\infty}\displaystyle\binom{\alpha}{n}x^n}$,其中对$\alpha\in\R$有$
	\displaystyle\binom{\alpha}{n}:=\frac{1}{n!}\alpha(\alpha-1)\cdots (\alpha-n+1)
	$,收敛域为$\begin{cases}[ll]
		\mathbb{R},& \alpha \in \mathbb{N},\\
		(-1,1),& \alpha \le -1,\\
		(-1,1],& -1 < \alpha < 0,\\
		[-1,1],& \alpha \in \R_+ \setminus \mathbb{N} \text{。}\\
	\end{cases}$ 特别地,有: \\
	$\dfrac{1}{1-x}=\sum\limits_{n=0}^{\infty}x^n,x\in (-1,1)$;\quad
	$\dfrac{1}{1+x}=\sum\limits_{n=0}^{\infty}(-1)^n x^n,x\in (-1,1)$;\\
	$\sqrt{1+x}=\sum\limits_{n=0}^{\infty}(-1)^{n-1}\dfrac{(2n-3)!!}{(2n)!!} x^n,x\in [-1,1]$;\\
	$\dfrac{1}{\sqrt{1+x}}=\sum\limits_{n=0}^{\infty}(-1)^{n}\dfrac{(2n-1)!!}{(2n)!!} x^n,x\in (-1,1]$。

	\hang[2](4)$\mbome{\ln(1+x)=\sum\limits_{n=0}^{\infty}(-1)^n\dfrac{x^{n+1}}{n+1}},x\in (-1,1]$;\quad
	$\ln(1-x)=-\sum\limits_{n=0}^{\infty}\dfrac{x^{n+1}}{n+1},x\in [-1,1)$;\\
	$\ln\dfrac{1+x}{1-x}=2\sum\limits_{n=0}^{\infty}\dfrac{x^{2n+1}}{2n+1},x\in (-1,1)$。

	(5)$\arctan x=\sum\limits_{n=0}^{\infty}(-1)^n\dfrac{x^{2n+1}}{2n+1},x\in [-1,1]$。
	
	(6)$\arcsin x=\sum\limits_{n=0}^{\infty}\dfrac{(2n-1)!!}{(2n)!!}\dfrac{x^{2n+1}}{2n+1},x\in [-1,1]$。
}
\begin{proofs}[\yan(\textup{\bf 3})的证明]
	\adjline
	当 $\alpha$ 是自然数的时候,$f(0)=1$,且
	$$ f^{k}(0)=\begin{cases}[ll]
		n(n-1)(n-2)\cdots (n-k+1),& k=1,2,\cdots ,n,\\
		0,& k=n+1,\cdots
	\end{cases}$$
	此时,$f(x)$ 的幂级数就是 $x$ 的 $n$ 此多项式:
	\begin{align*}
	(1+x)^\alpha &=1+nx+\frac{n(n-1)}{2!}x^2+\cdots + \frac{n(n-1)\cdots 2}{(n-1)!} x^{n-1}+x^n \\
	&= 1+ \binom{n}{1}x +\binom{n}{2} x^2+\cdots + \binom{n}{n-1}x{n-1}+x^n
	\end{align*}
	其中
	$\displaystyle\binom{n}{k}=\dfrac{n(n-1)\cdots (n-k+1)}{k!}$
	就是组合数,上面就是中学的二项式定理。

	下面假设 $\alpha \in   \mathbb{R}\setminus \mathbb{N}_+$。
	因为
	$$ f^{(n)}(x)=\alpha(\alpha-1)\cdots (\alpha-n+1)(1+x)^{\alpha-n},\quad \quad n=1,2,\cdots $$
	于是
	$$ f(0)=1,f'(0)=\alpha,f''(0)=\alpha(\alpha-1),\cdots, f^{n}(0)=\alpha(\alpha-1)\cdots (\alpha-n+1),\cdots $$
	得到一个形式的Taylor级数
	$$ 1+\alpha x+\frac{\alpha(\alpha-1)}{2!} x^2+\cdots + \frac{\alpha(\alpha-1)\cdots (\alpha-n+1)}{n!} x^n+\cdots$$
	它的收敛半径为
	$R=\lim\limits_{n\to \infty}\dfrac{|a_n|}{|a_{n+1}|}=\lim\limits_{n\to \infty}\dfrac{|n+1|}{|\alpha-n|}=1$,
	故在开区间 $(-1,1)$内,上述幂级数收敛,记它的和函数是 $S(x)$ 。
	
	下面说明$S(x)=(1+x)^\alpha$,但为了避免分析 $R_n(x)$,我们采取以下办法:容易验证上面$f(x)$满足
	$f'(x)=\dfrac{\alpha}{1+x} f(x)$,
	即$f(x)$ 是常微分方程$ y'-\dfrac{\alpha}{1+x} y=0$的解
	现在我们验证$ S(x)$	也是该ODE的解。

	因为
	$$S'(x)=\alpha\left[ 1+(\alpha-1)x+\frac{(\alpha-1)(\alpha-2)}{2!}x^2+\cdots +\frac{(\alpha-1)\cdots (\alpha-n)}{(n-1)!}x^{n-1}+\cdots \right]$$
	等式两端同乘以 $(1+x)$,得到 $x^n$ 的系数为
	$$ \alpha\left[ \frac{(\alpha-1)(\alpha-2)\cdots (\alpha-n}{n!}+ \frac{(\alpha-1)(\alpha-2)\cdots (\alpha-n+1}{(n-1)!}\right]=\alpha \frac{\alpha(\alpha-1)(\alpha-2)\cdots (\alpha-n+1)}{n!}$$
	于是
	$$ (1+x)S'(x)=\alpha\left[ 1+\alpha x+\frac{\alpha(\alpha-1}{2!}x^2+\cdots \right]=\alpha S(x)$$
	这就说明 $S(x)$ 是ODE的解。因为这个ODE是一阶齐次ODE,解空间是一维的,所以
	$$ S(x)=C(1+x)^\alpha$$
	再由 $S(0)=1$ 可以确定 $C=1$,从而  $S(x)=(1+x)^\alpha$	。		
\end{proofs}

\begin{example}
	求$f(x)=\dfrac{x-1}{4-x}$在点$x=1$的幂级数展式,
	并对任意整数$n\ge 0$,计算$f^{(n)}(1)$。
\end{example}
\begin{solution}
	\adjline
	当$|x-1|<3$时,我们有
	\begin{align*}
	f(x)=\frac{x-1}{3-(x-1)} 
	=\frac{x-1}{3}\cdot \frac{1}{1-\frac{x-1}{3}}
	=\frac{x-1}{3}\cdot \sum_{k=0}^{\infty}\left(\frac{x-1}{3}\right)^k 
	=\sum_{n=1}^{\infty}\frac{1}{3^n}(x-1)^n
	\end{align*}
	由此立刻可知$f(1)=0$,且$\forall n\ge 1$,
	我们有
	$f^{(n)}(1)=\dfrac{1}{3^n}\cdot n! 
	=\dfrac{n!}{3^n}$。
\end{solution}

\subsection{Fourier级数}

\subsubsection{形式Fourier级数}

\de[函数的内积]{
	$\forall f,g\in\mathscr{C}[a,b]$,$(f,g):=\dint_a^bf(x)g(x) \dif x$
	称之为$f,g$的\tboba{内积}。如果$(f,g)=0$,则称$f$与$g$~\tboba{正交},记作$f\perp g$。
}
\de[函数的范数]{
	$\forall f\in\mathscr{C}[a,b]$,
	$\|f\|=\sqrt{(f,f)} :=\left(\dint_a^b f^2(x) \dif x\right)^{\frac{1}{2}}$
	称为$f$的\tboba{范数}。	
}
$f\equiv 0$  当且仅当 $\|f\|=0$;$\forall f,g \in\mathscr{C}[a,b]$,均有$\|f+g\|\le \|f\|+\|g\|$。

\begin{definition}
	$\varLambda=\{1,\cos x,\sin x,\cdots,\cos nx,\sin nx,\cdots\}$称为\textbf{三角函数系}。
\end{definition}
三角函数系有如下的性质:
\begin{lemma}
	\textbf{\textup{正交性}}\quad 三角函数系$\Lambda$在$[a,a+2\pi]$上为非零的正交函数系,即$\forall f,g \in \varLambda$,若$f\neq g$,则$$\dint_a^{a+2\pi}f(x)g(x) \dif x=0$$
\end{lemma}
事实上,$\forall n,m\ge 1$且$n\neq m$有
\begin{align*}
	&\int_a^{a+2\pi}1\dif x=2\pi,
	\int_a^{a+2\pi}\cos^2 nx\dif x=\int_a^{a+2\pi}\sin^2 nx\dif x=\pi,\\[-4pt]
	&\int_a^{a+2\pi}\cos nx\dif x =\int_a^{a+2\pi}\sin nx\dif x=0,\\[-4pt]
	&\int_a^{a+2\pi}\cos mx\cos nx\dif x =\int_a^{a+2\pi}\cos mx\sin nx\dif x=\int_a^{a+2\pi}\sin mx\sin nx\dif x=0
\end{align*}
\begin{lemma}
	\textbf{\textup{完全性}}\quad 如果$f\in\mathscr{C}(\mathbb{R})$是以$2\pi$为周期的周期函数使得$\forall g\in \varLambda$,均有$ \dint_a^{a+2\pi}f(x)g(x)\dif x=0$,则我们有$f\equiv 0$。
\end{lemma}
由此可知,$\varLambda$当中的元素在$\mathbb{R}$上线性无关,且$\varLambda$就是$\mathscr{C}(\mathbb{R})$中以$2\pi$为周期的周期函数空间的一组基。

\de[三角级数]{
	形如$\dfrac{a_0}{2}+\sum\limits_{n=1}^{\infty}(a_n\cos nx+b_n\sin nx)$
	称为\tboba{三角级数}。
}
假设$f$是以$2\pi$为周期的周期函数且
$f(x)=\dfrac{a_0}{2}+\sum\limits_{k=1}^{\infty}(a_k\cos kx+b_k\sin kx)$
在$[-\pi,\pi]$上一致收敛,则由积分与级数求和可交换性立刻可知 
\begin{equation*}
a_n=\frac{1}{\pi}\int_{-\pi}^{\pi}f(x)\cos nx\dif x\ (n\ge 0),\qquad
b_n=\frac{1}{\pi}\int_{-\pi}^{\pi}f(x)\sin nx\dif x\ (n\ge 1)
\end{equation*}
\de[Fourier系数]{
	假设$f: \mathbb{R}\rightarrow\mathbb{R}$是以$2\pi$为周期的周期
	函数且$f\big|_{[-\pi,\pi]}\in\mathscr{R}[-\pi,\pi]$,则称
	\begin{equation}
		a_n(f)=\frac{1}{\pi}\int_{-\pi}^{\pi}f(x)\cos nx\dif x\ (n\ge 0),\qquad
		b_n(f)=\frac{1}{\pi}\int_{-\pi}^{\pi}f(x)\sin nx\dif x\ (n\ge 1)
	\end{equation}
	为$f$的~\tboba{Fourier系数},并记
	\begin{equation}
		f(x)\sim \frac{a_0(f)}{2}+\sum_{n=1}^{\infty}\left(\mboba{a_n(f)}\cos nx+\mboba{b_n(f)}\sin nx\right)
	\end{equation}
	为$f$的\tboba{形式Fourier级数}。
}
若$f$为偶函数,则$\forall n\ge 1$,$b_n(f)=0$,此时
$\forall n\ge 0,
a_n(f)=\dfrac{2}{\pi}\dint_0^{\pi}f(x)\cos nx\dif x,b_n(f)=0$,
相应的Fourier级数称为\textbf{余弦级数};若$f$为奇函数,则$\forall n\ge 0$,$a_n(f)=0$,此时
$\forall n\ge 1,a_n(f)=0,
b_n(f)=\dfrac{2}{\pi}\dint_0^{\pi}f(x)\sin nx\dif x$,
相应的Fourier级数称为\textbf{正弦级数}。

\begin{example}
	证明:如果级数
	$\dfrac{|a_0|}2+\sum\limits_{k=1}^\infty\left(|a_k|+|b_k|\right)<+\infty$,那么级数
	$\dfrac{a_0}2+\sum\limits_{k=1}^\infty\left(a_k\cos kx+b_k\sin kx\right)$必为某个周期为 $2\pi$ 的函数的傅里叶级数。
\end{example}
\begin{proofs}\adjline
	我们熟知如下引理:
	\begin{lemma}
		周期为 $2\pi$ 的$n$次三角多项式$T_n(x) = \sum\limits_{k=0}^n \left(a_k \cos kx + b_k \sin kx\right)$的\textup{Fourier}级数就是其本身。\label{本身级数}
	\end{lemma}
	\vspace{-1em}
	由非负项级数
	$\dfrac{|a_0|}2+\sum\limits_{k=1}^\infty\left(|a_k|+|b_k|\right)$有上界,知其收敛,则由 Weierstrass判别法,
	$$
	\left|\dfrac{a_0}2\right|+\sum\limits_{k=1}^\infty\left|a_k\cos kx+b_k\sin kx\right| 
	\le \dfrac{\left|a_0\right|}2+\sum\limits_{k=1}^\infty\left(\left|a_k\cos kx\right|+\left|b_k\sin kx\right|\right) 
	\le \dfrac{|a_0|}2+\sum\limits_{k=1}^\infty\left(|a_k|+|b_k|\right)
	$$
	知$\dfrac{a_0}2+\sum\limits_{k=1}^\infty\left(a_k\cos kx+b_k\sin kx\right)$绝对且一致收敛于某个函数$S(x)$。对引理~\ref{本身级数}~取$n \to \infty$,即知$S(x)$是所求的函数,即级数
	$\dfrac{a_0}2+\sum\limits_{k=1}^\infty\left(a_k\cos kx+b_k\sin kx\right)$是函数$S(x)$的傅里叶级数。
\end{proofs}

\subsubsection{Fourier级数的性质及收敛性}
\begin{lemma}
	\textbf{\textup{Riemann-Lebesgue引理}}\quad 若$f$在$[a,b]$上可积,则
	\begin{equation}
	\lim\limits_{\lambda\rightarrow \infty}\int_a^b f(x)\sin (\lambda x)\dif x
	=\lim\limits_{\lambda\rightarrow \infty}\int_a^b f(x)\cos (\lambda x)\dif x=0
	\end{equation}
\end{lemma}
这个引理保证了,若$f:\mathbb{R}\rightarrow\mathbb{R}$是以$2\pi$为
周期的周期函数且在$[-\pi,\pi]$上可积,则
$\lim\limits_{n\rightarrow \infty}a_n(f)=\lim\limits_{n\rightarrow \infty}b_n(f)=0$。

\di[Dirichlet-Jordan定理]{
	假设$f$是以$2\pi$为周期的周期函数。
	如果$f$在$[-\pi,\pi]$上逐段单调有界或逐段可微,那么$\forall x\in \mathbb{R}$,函数$f$的Fourier级数在点$x$处收敛到
	$S(x)=\dfrac{1}{2}\left(f(x-0)+f(x+0)\right)$。
	\begin{definition}
		考虑函数$f:[a,b]\rightarrow\mathbb{R}$及区间$[a,b]$的
		分割$a=x_0<x_1<\cdots <x_n=b$。

		(1)若$f$在每个子区间$(x_{j-1},x_j)$上单调,则称	函数$f$为\textbf{逐段(或分段)单调}。
		
		(2)若$f$在每个子区间$[x_{j-1},x_j]$上可微,则称函数$f$为\textbf{逐段(或分段)可微};此时$f$在$[a,b]$上\textbf{逐段连续},因此$f$为有界函数。
	\end{definition}
}
更一般地,就有
\begin{theorem}
	\textbf{\textup{收敛性定理}\quad}设$f$是以$2\pi$为周期的周期函数,并且在$[-\pi,\pi]$上可积。如果函数$f$ 在$(-\pi,\pi)$上连续且逐段单调有界,或者有有界导数,那么$f$的\textup{Fourier}级数在$(-\pi,\pi)$的任意闭子区间上一致收敛到$f$本身。
\end{theorem}

\zhu[非$\R$上周期函数的延拓]{
	对于定义在区间$(-\pi,\pi)$上的任意函数$f$,尽管它并不是定义在$\mathbb{R}$上并且以$2\pi$为周期的函数,我们仍然可以将$f$延拓成为以$2\pi$为周期的函数, 从而定义其Fourier系数\!(若相关积分均存在)\!, 由此得到形式Fourier级数。

	为此,我们取$f(\pi)=f(-\pi)$为任意常数,再将$f$以$2\pi$为周期从区间$[-\pi,\pi]$延拓到整个$\mathbb{R}$上。随后再来考虑延拓后的函数$f$的Fourier级数。若此时函数$f$满足Dirichlet-Jordan定理的条件,则$\forall x\in (-\pi,\pi)$,函数$f$的Fourier级数在点$x$收敛到$S(x)=\dfrac{1}{2}\left(f(x-0)+f(x+0)\right)$,在两个端点$x=\pm \pi$收敛到$\dfrac{1}{2}\left(f(-\pi+0)+f(\pi-0)\right)$。
}
\begin{example}
	$\forall x\in (-\pi,\pi)$,定义$f(x)=\e^{-x}$。求函数$f$在$(-\pi,\pi)$内的Fourier级数并讨论其收敛性。
\end{example}
\begin{solution}\adjline
	由定义可知
	\begin{equation*}
	a_0=\frac{1}{\pi}\int_{-\pi}^{\pi}\e^{-x}\dif x 
	=\frac{1}{\pi}(\e^{\pi}-\e^{-\pi }) 
	=\frac{2}{\pi}\sinh\pi
	\end{equation*}
	而$\forall n\ge 1$,我们也有
	\begin{align*}
	a_n&=\frac{1}{\pi}\int_{-\pi}^{\pi}\e^{-x}\cos (nx)\dif x 
	=(-1)^n\frac{2}{\pi(1+n^2)}\sinh\pi,\\[-6pt]
	b_n&=\frac{1}{\pi}\int_{-\pi}^{\pi}\e^{-x}\sin (nx)\dif x 
	=(-1)^n\frac{2n}{\pi(1+n^2)}\sinh\pi
	\end{align*}
	由于$f$在$(-\pi,\pi)$上可微,则$\forall x\in (-\pi,\pi)$,
	\begin{align*}
	\e^{-x}&=\frac{a_0}{2}+\sum_{n=1}^{\infty}\left(a_n\cos (nx)+b_n\sin (nx)\right)  \\[-6pt]
	&=\frac{2\sinh\pi}{\pi}\left(\frac{1}{2}+
	\sum_{n=1}^{\infty}\left(\frac{(-1)^n}{1+n^2}\cos (nx)+\frac{(-1)^nn}{1+n^2}\sin (nx)\right)\right)
	\end{align*}
	上述Fourier级数在点$x=\pm\pi$处收敛到
	$\dfrac{1}{2}(f(-\pi+0)+f(\pi-0))=\dfrac{1}{2}(\e^{\pi}+\e^{-\pi})$。
\end{solution}
特别地,在$x=0$处,
则有$1=\dfrac{2\sinh\pi}{\pi}\left(\dfrac{1}{2}+\sum\limits_{n=1}^{\infty}\dfrac{(-1)^n}{1+n^2}\right)$,
故$$\sum\limits_{n=1}^{\infty}\dfrac{(-1)^n}{1+n^2}=\dfrac{\pi}{2\sinh\pi}-\dfrac{1}{2}$$
而在点$x=\pi$处,我们有
$\dfrac{2\sinh\pi}{\pi}\left(\dfrac{1}{2}+\sum\limits_{n=1}^{\infty}\dfrac{1}{1+n^2}\right)=\dfrac{1}{2}(\e^{\pi}+\e^{-\pi})=\cosh\pi$,于是
\begin{equation*}
\sum\limits_{n=1}^{\infty}\frac{1}{1+n^2}=\frac{\pi}{2\tanh \pi}-\frac{1}{2}
\end{equation*}	
\zhu[一般周期函数的Fourier级数]{
	假设$\ell>0$  而$T=2\ell$。对周期为$T$的周期函数,
	我们可以相应地引入在任何长度为$T$的区间上
	均为正交的三角函数系
	$
	\Big\{1,\cos \dfrac{\pi}{\ell} x,\sin \dfrac{\pi}{\ell} x,\cdots,\cos \dfrac{n\pi}{\ell} x,\sin \dfrac{n\pi}{\ell} x,\cdots\Big\}
	$。
	关于该函数系,前面介绍的所有结论依然成立。例如对Dirichlet-Jordan定理,只需将$\pi$换成$\ell$。
	
	特别地,我们可以类似定义:
	\begin{equation}
	a_n(f)=\frac{1}{\ell}\int_{-\ell}^{\ell}f(x)\cos \frac{n\pi}{\ell} x\dif x\ (n\ge 0),\quad
	b_n(f)=\frac{1}{\ell}\int_{-\ell}^{\ell}f(x)\sin \frac{n\pi}{\ell} x\dif x\ (n\ge 1)
	\end{equation}
	称为$f$的Fourier系数,并记 
	\begin{equation}
	f(x)\sim \frac{a_0(f)}{2}+\sum_{n=1}^{\infty}\left(a_n(f)\cos \frac{n\pi}{\ell} x +b_n(f) \sin \frac{n\pi}{\ell} x \right)
	\end{equation}
	称上述函数项级数为$f$的(形式)Fourier级数。
}
若只给定$f(x)$ 是定义在 $(0,L)$ 上的函数,为求其 周期为$2L$的Fourier级数,有两种延拓方式:
\begin{itemize}
	\item \textbf{奇延拓\quad}$\forall x\in (-L,L)$,定义
	$F(x)=\begin{cases}[ll]
		f(x),&\textrm{若}\ x\in (0,L),\\
		-f(-x),&\textrm{若}\ x\in (-L,0),	
	\end{cases}$
	此时 $\forall n\ge 0$,$a_n=0$,而$\forall n\ge 1$,我们则有
	$b_n=\dfrac{2}{L}\dint_0^{L}f(x)\sin \dfrac{n\pi}{L}x\dif x$,相应的Fourier级数为正弦级数。

	\item \textbf{偶延拓\quad}$\forall x\in (-L,L)$,定义
	$F(x)=\begin{cases}[ll]
		f(x),&\textrm{若}\ x\in (0,L),\\
		f(-x),&\textrm{若}\ x\in (-L,0),	
	\end{cases}$
	此时 $\forall n\ge 0$,$b_n=0$,而$\forall n\ge 1$,我们则有
	$a_n=\dfrac{2}{L}\dint_0^{L}f(x)\cos \dfrac{n\pi}{L}x\dif x$,相应的Fourier级数为余弦级数。
\end{itemize}

\begin{example}
	$\forall x\in[0,2]$,令$f(x)=2-x$。将$f$在$[0,2]$上展成以$4$为周期的余弦级数并求和函数。
\end{example}
\begin{solution}\adjline
	首先将$f$偶延拓而定义
	$F(x)=\begin{cases}[ll]
		2-x,&\textrm{若}\ x\in [0,2],\\
		2+x,&\textrm{若}\ x\in [-2,0],
	\end{cases}$
	此时$T=4$,$\ell=2$。故$F$的Fourier系数满足$b_n=0\ (n\ge 1)$。另外,我们还有
	\begin{equation*}
	a_0=\frac{2}{\ell}\int_0^{\ell}F(x)\dif x  
	=\int_0^2(2-x)\dif x  =2
	\end{equation*}

	$\forall n\ge 1$,我们有
	\begin{equation*}
	a_n=\frac{2}{\ell}\int_0^{\ell}F(x)\cos\frac{n\pi}{\ell}x\dif x
	=\int_0^2(2-x)\cos\frac{n\pi}{2}x\dif x 
	=\left(\frac{2}{n\pi}\right)^2(1-\cos (n\pi))
	=\left(\frac{2}{n\pi}\right)^2\left(1-(-1)^n\right)
	\end{equation*}
	由于函数$F$在$[-2,2]$上为连续并且分段可微,而$F(-2)=F(2)$,于是$\forall x\in [0,2]$,我们有
	\begin{equation*}
	f(x)=2-x 
	=1+\sum_{k=1}^{\infty}\frac{8}{(2k-1)^2\pi^2}\cos\frac{(2k-1)\pi}{2}x
	\qedhere
	\end{equation*}
\end{solution}
特别地,在点$x=0$处,我们有
$2=1+\sum\limits_{k=1}^{\infty}\dfrac{8}{(2k-1)^2\pi^2}$,
由此立刻可得$\sum\limits_{k=1}^{\infty}\dfrac{1}{(2k-1)^2}=\dfrac{\pi^2}{8}$。

\subsubsection{Fourier级数的平方平均收敛}
对任意的整数$n\ge 1$,我们令
\begin{equation*}
	\varLambda_n=\{1,\cos x,\sin x,\cdots,\cos (nx),\sin (nx)\}
\end{equation*}
如果将$\varLambda_n$所张成的实线性空间记作$\mathcal{W}_n$,那么$\mathcal{W}_n$为$\mathscr{R}[-\pi,\pi]$的$2n+1$维子空间。
\di[最佳逼近定理]{
	$\forall f\in\mathscr{R}[-\pi,\pi]$,令
	\begin{equation}
		S_n(f)=\frac{a_0(f)}{2}+\sum_{k=1}^{n}\left(a_k(f)\cos (kx)+b_k(f)\sin (kx)\right)
	\end{equation}
	则$\|f-S_n(f)\|=\min\limits_{g\in \mathcal{W}_n}\|f-g\|$,且最小值仅在$g=S_n(f)$处达到,此时有$ (f-S_n(f)) \perp \mathcal{W}_n$。
}
\begin{proofs}\adjline
	对任意的整数$0\le k\le n$,我们有
	\begin{align*}
	\bigl(f-S_n(f),\cos (kx)\bigr)
	&=\int_{-\pi}^{\pi}f(x)\cos (kx)\dif x-\int_{-\pi}^{\pi}S_n(f)(x)\cos (kx)\dif x  \\[-6pt]
	&=\pi a_k(f)-\pi a_k(f)  =0
	\end{align*}
	同样,$\forall 1\le k\le n$,均有$\bigl(f - S_n(f),\sin (kx)\bigr) = 0$。于是$f-S_n(f)$与$\varLambda_n$中的任意元素正交,从而由线性性可知,$f-S_n(f)$与$\mathcal{W}_n$中的任意元素正交,也就是说$f-S_n(f) \perp \mathcal{W}_n$。

	$\forall g\in \mathcal{W}_n$,定义$F_n=f-S_n(f)$,$G_n=g-S_n(f)$,则$G_n\in \mathcal{W}_n$,从而$(F_n,G_n)=0$,且我们有
	\begin{equation*}
	\|f-g\|^2=\|(f-S_n(f))-(g-S_n(f))\|^2 
	=\|F_n-G_n\|^2 
	=\|F_n\|^2+\|G_n\|^2 \ge \|F_n\|^2
	\end{equation*}
	上式恰好表明我们有
	\begin{equation*}
	\min\limits_{g\in \mathcal{W}_n}\|f-g\|=\|F_n\| =\|f-S_n(f)\|
	\end{equation*}
	并且仅当$\|G_n\|^2 = 0$,即$g=S_n(f)$时取到最小值。
\end{proofs}
\begin{lemma}
	\textup{\textbf{Bessel不等式\quad}}
	$\forall f\in\mathscr{R}[-\pi,\pi]$,均有
	\begin{equation}
		\frac{1}{2}\left(a_0(f)\right)^2+\sum_{k=1}^{\infty}\left(\left(a_k(f)\right)^2+\left(b_k(f)\right)^2\right)
		\le \frac{1}{\pi}\int_{-\pi}^{\pi}\left(f(x)\right)^2\dif x
	\end{equation}
\end{lemma}
\begin{proofs}
	对任意整数$n\ge 1$,我们有
	\begin{align*}
		0&\le \|f-S_n(f)\|^2 
		=\|f\|^2+\|S_n(f)\|^2-2(f,S_n(f)) \\[-6pt]
		&=\int_{-\pi}^{\pi}\left(f(x)\right)^2\dif x 
		+ \boldsymbol{\color{meihong!50!black}\int_{-\pi}^{\pi}\left(S_n(f)(x)\right)^2\dif x} 
		- \mbome{2\int_{-\pi}^{\pi}f(x)\left(\frac{a_0}{2}+\sum_{k=1}^{n}\left(a_k\cos (kx)+b_k\sin (kx)\right)\right)\dif x} \displaybreak\\[-6pt]
		&=\int_{-\pi}^{\pi}\left(f(x)\right)^2\dif x
		+ \boldsymbol{\color{meihong!50!black}\pi\left(\frac{a_0^2}{2}+\sum_{k=1}^n(a_k^2+b_k^2)\right)}
		- \mbome{\pi  \left(a_0^2+2\sum_{k=1}^{n}\left(a_k^2+b_k^2\right)\right)} \\[-6pt]
		&=\int_{-\pi}^{\pi}\left(f(x)\right)^2\dif x-\pi\left(\frac{a_0^2}{2}+\sum_{k=1}^n(a_k^2+b_k^2)\right)
	\end{align*}
	由此立得  
	\begin{equation*}
	\frac{a_0^2}{2}+\sum_{k=1}^n(a_k^2+b_k^2)\le \frac{1}{\pi}\int_{-\pi}^{\pi}\left(f(x)\right)^2\dif x
	\end{equation*}
	随后让$n\rightarrow\infty$,可知
	\begin{equation*}
	\frac{a_0^2}{2}+\sum_{k=1}^{\infty}(a_k^2+b_k^2)\le \frac{1}{\pi}\int_{-\pi}^{\pi}\left(f(x)\right)^2\dif x \qedhere
	\end{equation*}
\end{proofs}
\begin{corollary}
	$\forall f\in\mathscr{R}[-\pi,\pi]$,级数$\dfrac{1}{2}\left(a_0(f)\right)^2+\sum\limits_{k=1}^{\infty}\left(\left(a_k(f)\right)^2+\left(b_k(f)\right)^2\right)$收敛。
\end{corollary}
\di[Parseval等式]{
	$\forall f\in\mathscr{R}[-\pi,\pi]$,均有
	\begin{equation}
	\frac{1}{2}\left(a_0(f)\right)^2+\sum_{k=1}^{\infty}\left(\left(a_k(f)\right)^2+\left(b_k(f)\right)^2\right)
	=\frac{1}{\pi}\int_{-\pi}^{\pi}\left(f(x)\right)^2\dif x
	\end{equation}	
}
\begin{corollary}
	\textup{\textbf{唯一性\quad}}若$f,g\in\mathscr{R}[-\pi,\pi]$有相同的\textup{Fourier}级数,则$f,g$几乎处处相等;若$f,g\in\mathscr{C}[-\pi,\pi]$有相同的\textup{Fourier}级数,则$f \equiv g$。
\end{corollary}
\begin{proofs}
	由于$a_0(f-g)=a_0(f)-a_0(g)=0$,而且$\forall n\ge 1$,同样也有$a_n(f-g)=0$,$b_n(f-g)=0$,于是由Parseval等式可知$\dint_{-\pi}^{\pi}\left(f(x)-g(x)\right)^2\dif x=0$。
\end{proofs}
\di[Fourier级数在范数意义下的收敛(平方平均收敛)]{
	$\forall f\in\mathscr{R}[-\pi,\pi]$,均有$\lim\limits_{n\rightarrow\infty}\|S_n(f)-f\|=0$。
}
\begin{proofs}
	$\|f-S_n(f)\|^2=\dint_{-\pi}^{\pi}\left(f(x)\right)^2\dif x-\pi\left(\dfrac{a_0^2}{2}+\sum\limits_{k=1}^n(a_k^2+b_k^2)\right)$,再让$n\rightarrow\infty$。
\end{proofs}
\begin{corollary}
	\textup{\textbf{广义Parseval等式\quad}}$\forall f,g\in\mathscr{R}[-\pi,\pi]$,均有
	\begin{equation}
	\frac{1}{2}a_0(f)a_0(g)+\sum_{k=1}^{\infty}\left(a_k(f)a_k(g)+b_k(f)b_k(g)\right)=\frac{1}{\pi}\int_{-\pi}^{\pi}f(x)g(x)\dif x
	\end{equation}
\end{corollary}
\begin{proofs}
	对任意的整数$n\ge 1$,我们有
	\begin{align*}
	\frac{1}{\pi}\int_{-\pi}^{\pi} f(x)g(x)\dif x
	&=\frac{1}{\pi}\int_{-\pi}^{\pi}\left(f(x)-S_n(f)\right)g(x)\dif x
	+\frac{1}{\pi}\int_{-\pi}^{\pi} S_n(f)(x)g(x)\dif x \\[-6pt]
	&=\frac{1}{\pi}\int_{-\pi}^{\pi}\left(f(x)-S_n(f)\right)g(x)\dif x \\[-6pt]
	&\quad+\frac{1}{\pi}\int_{-\pi}^{\pi} \left(\frac{a_0(f)}{2}+\sum_{k=1}^{n}\left(a_k(f)\cos (kx)+b_k(f)\sin (kx)\right)\right)g(x)\dif x \\[-6pt]
	&=\frac{1}{\pi}\int_{-\pi}^{\pi}\left(f(x)-S_n(f)\right)g(x)\dif x +\frac{a_0(f)}{2}\frac{1}{\pi}\int_{-\pi}^{\pi}g(x)\dif x \\[-6pt]
	&\quad+ \sum_{k=1}^{n}\left(a_k(f)\frac{1}{\pi}\int_{-\pi}^{\pi}\cos (kx)g(x)\dif x +b_k(f)\frac{1}{\pi}\int_{-\pi}^{\pi}\sin (kx)g(x)\dif x \right)\\[-6pt]
	&=\frac{1}{\pi}\int_{-\pi}^{\pi}\left(f(x)-S_n(f)\right)g(x)\dif x+\frac{1}{2}a_0(f)a_0(g)+\sum_{k=1}^{n}\left(a_k(f)a_k(g)+b_k(f)b_k(g)\right)
	\end{align*}
	由Cauchy不等式,我们立刻有
	\begin{align*}
	\left|\int_{-\pi}^{\pi}\left(f(x)-S_n(f)\right)g(x)\dif x\right|
	&\le \int_{-\pi}^{\pi}|f(x)-S_n(f)|\cdot |g(x)|\dif x \\[-6pt]
	&\le \left(\int_{-\pi}^{\pi}|f(x)-S_n(f)|^2\dif x\right)^{\frac{1}{2}}\cdot
	\left(\int_{-\pi}^{\pi}|g(x)|^2\dif x\right)^{\frac{1}{2}}
	= \|f-S_n(f)\| \cdot \|g\|
	\end{align*}
	又$\lim\limits_{n\rightarrow\infty}\|f-S_n(f)\|=0$,于是由夹逼原理可知所证结论成立。
\end{proofs}

\di[Fourier级数求和与积分的可交换性]{
	$\forall f\in\mathscr{R}[-\pi,\pi]$ 以及$\forall a,x\in [-\pi,\pi]$,均有
	\begin{equation}
	\int_{a}^{x}f(t)\dif t
	=\frac{1}{2}a_0(f)(x-a)
	+\sum_{k=1}^{\infty}\int_a^x\left(a_k(f)\cos (kt)+b_k(f)\sin (kt)\right)\dif t
	\end{equation}
	也即Fourier级数求和(即便它\tbome{不是点态收敛}时也成立)总是可以与积分交换次序。
}
\begin{proofs}
	固定$a,x\in [-\pi,\pi]$。不失一般性,我们可假设$a<x$。$\forall t\in [-\pi,\pi]$,令$g(t)=\begin{cases}[ll]
		1,&\textrm{若}\ t\in [a,x],\\
		0,&\textrm{若}\ t\notin [a,x],
	\end{cases}$
	则$g$在$[-\pi,\pi]$上可积。于是我们有
	\begin{align*}
	\int_{a}^{x}f(t)\dif t=\int_{-\pi}^{\pi}f(t)\mbome{g(t)}\dif t 
	&=\frac{\pi}{2}a_0(f)\mbome{a_0(g)}
	+\pi\sum_{k=1}^{\infty}\left(a_k(f)\mbome{a_k(g)}+b_k(f)\mbome{b_k(g)}\right) \\[-6pt]
	&=\frac{\pi}{2}a_0(f)\mbome{\int_a^x \dif t}
	+\pi\sum_{k=1}^{\infty}\left(a_k(f)\mbome{\int_a^x \cos kt \dif t}+b_k(f)\mbome{\int_a^x \sin kt \dif t}\right) \\[-6pt]
	&=\frac{1}{2}a_0(f)(x-a)
	+\sum_{k=1}^{\infty}\int_a^x\left(a_k(f)\cos (kt)+b_k(f)\sin (kt)\right)\dif t \qedhere
	\end{align*}
\end{proofs}

\newpage
%------------------------------------------------------------
\section{多元函数及其连续性}

\subsection{Euclid 空间}

\de[Euclid 空间]{
	设$n\ge 1$为整数,定义
	$\mathbb{R}^n=\big\{(x_1,\cdots,x_n)\ |\ x_j\in\mathbb{R},1\le j\le n\big\}$,称为$\mboba{n}$~\tboba{维Euclid空间}。
	
	对于$X=(x_1,\cdots,x_n)\in\mathbb{R}^n$,定义$\|X\|_n:=\sqrt{\sum\limits_{j=1}^n|x_j|^2}$,称为$X$的\tboba{范数},在不产生混淆时,记作$\|X\|$。
	$\forall X=(x_1,\cdots,x_n),Y=(y_1,\cdots,y_n)\in\mathbb{R}^n$,定义
	$d(X,Y):=\|X-Y\|$为$X,Y$之间的\tboba{距离},即有$\|X-Y\|=\sqrt{(x_1-y_1)^2+\cdots+(x_n-y_n)^2}$。
}

\subsubsection{$\R^n$的拓扑性质}

\begin{definition}
	固定$X_0\in\mathbb{R}^n$,$\delta>0$,定义
	\begin{itemize}
		\item $\boldsymbol{B(X_0,\delta)}=\big\{X\in\mathbb{R}^n\hskip -2pt\ |\ \|X-X_0\|<\delta \big\}$,称为点$X_0$的$\boldsymbol{\delta}$\textbf{-邻域},也称为以$X_0$为中心以$\delta$
		为半径的\textbf{开球}。

		\item $\boldsymbol{\mathring{B}(X_0,\delta)}=\big\{X\in\mathbb{R}^n\ |\ 0<\|X-X_0\|<\delta \big\}$,称为$X_0$的\textbf{去心}~$\boldsymbol{\delta}$\textbf{-邻域}。
	\end{itemize}
\end{definition}
\begin{definition}
	固定$S\subseteq \mathbb{R}^n$,$X_0\in\mathbb{R}^n$,定义
	\begin{itemize}
	\item 如果$\exists \delta>0$使得$B(X_0,\delta)\subseteq S$,则称
	点$X_0$为$S$的一个\textbf{内点}。

	\item 如果$\exists \delta>0$使得$B(X_0,\delta)\cap S=\varnothing$,
	则称点$X_0$为$S$的一个\textbf{外点}。

	\item 若$X_0$既不为$S$的内点,也不为其外点,则称$X_0$为$S$的一个\textbf{边界点}。等价地,点$X_0$为$S$的边界点当且仅当$\forall \delta>0$,均有
	$B(X_0,\delta)\cap (\mathbb{R}^n\setminus S)\neq \varnothing$,
	$B(X_0,\delta)\cap S\neq \varnothing$。

	\item 若$\forall \delta>0$,均有$\mathring{B}(X_0,\delta)\cap S\neq \varnothing$,则称$X_0$为$S$的一个\textbf{极限点}。
	\end{itemize}
\end{definition}
\de[开集、闭集]{
	若$S$的每点均为内点,则称$S$为\tboba{开集};若$\mathbb{R}^n\setminus S$为开集,则称$S$为\tboba{闭集}。
	\begin{itemize}[leftmargin=1em]
		\item 由$S$的所有内点组成的集合称为它的\tboba{内部},记作$\mathring{S}$,也记作$\mathrm{Int}\mathop{} S$,这是一个开集。
		
		\item 由$S$的所有外点组成的集合称为它的\tboba{外部},记作$\mathrm{Ext}\mathop{} S$,这是一个开集。
		
		\item 由$S$的所有边界点组成的集合称为$S$的\tboba{边界},记作$\partial S$,这是一个闭集。
		
		\item $\overline{S}:=\partial S\cup S$为$S$的\tboba{闭包},它为闭集。
	\end{itemize}
}
$\mathbb{R}^n$为$\mathrm{Int}\mathop{}S$,$\partial S$和$\mathrm{Ext}\mathop{}S$的不交并。$\varnothing$,$\mathbb{R}^n$既为开集,也为闭集。


\begin{theorem}
	$S\subseteq \mathbb{R}^n$为开集当且仅当它为若干个开球的并。
\end{theorem}
\begin{proofs}
	(1)\textbf{充分性}。假设$S$为若干个开球的并,即
	$S=\bigcup\limits_{i\in J} B(X_i,\delta_i)$,
	则$\forall X\in S$,存在$X_{i_0}\in S$和$\delta_{i_0}>0$使得$X\in B(X_{i_0},\delta_{i_0})\subseteq S,X_{i_0}\in S$。	
	令$\eta=\delta_{i_0}-d(X,X_{i_0})=\delta_{i_0}-\|X-X_{i_0}\|>0$,则只需证$B(X,\eta)\subseteq S$。
	
	事实上,$\forall W \in  B(X,\eta) $,$ \|W-X\|<\eta $,由三角不等式就有 $ \|W-X_{i_0}\|\leq \|W-X\|+\|X- X_{i_0}\|<\eta + d(X,X_{i_0})=\delta_{i_0}$。
	于是$\forall X\in S$,$\forall W \in  B(X,\eta) $,$W\in B(X_{i_0},\delta_{i_0})$,即$\forall X\in S$,$B(X,\eta)\subseteq S$,故$S$为开集。

	(2)\textbf{必要性}。若$S$为开集,则$\forall X\in S$,$\exists \delta_X>0$使得	$B(X,\delta_X)\subseteq S$,则$S=\bigcup\limits_{X\in S}B(X,\delta_X)$。
\end{proofs}
\begin{corollary}
	任意多个开集的并还是开集,任意多个闭集的交还是闭集。
\end{corollary}
\begin{theorem}
	有限多个开集的交为开集。
\end{theorem}
\begin{proofs}
	设$S=\bigcap\limits_{j=1}^kS_j$,其中$S_j$为开集。
	对任意	$X\in S$以及任意$1\le j\le k$,
	因$X\in S_j$ 且$S_j$为开集,则$\exists \delta_j>0$ 使得$B(X,\delta_j)\subseteq S_j$。令
	$\delta=\min\limits_{1\le j\le k}\delta_j$,
	则$B(X,\delta)=\bigcap\limits_{j=1}^kB(X,\delta_j)\subseteq S$。故所证成立。
\end{proofs}
\begin{corollary}
	有限多个闭集的并为闭集。
\end{corollary}
\de[连通集]{
	给定集合$D\subseteq\mathbb{R}^n$,如果$\forall X,Y\in D$,均存在$D$中的折线将$X,Y$连接起来,就称集合$D\subseteq\mathbb{R}^n$为\tboba{连通集};否则,则称为\tboba{非连通集}。

	$\mathbb{R}^n$中非空的连通开集称为\tboba{开区域},开区域的闭包称为\tboba{闭区域}。
}

\subsubsection{$\R^n$中点列的收敛}

\de[点列收敛]{
	设$\{X_k\}$为$\mathbb{R}^n$中的点列,而$A\in\mathbb{R}^n$。
	\begin{itemize}[leftmargin=1em]
		\item 称$\{X_k\}$~\tboba{收敛}到$A$,
		若$\forall \varepsilon>0$,$\exists N\in\mathbb{N}$,使得
		$\forall k>N$,均有$\|X_k-A\|<\varepsilon$,此时记$\lim\limits_{k\rightarrow\infty}X_k=A$。

		\item 称$\{X_k\}$为~\tboba{Cauchy序列},
		若$\forall \varepsilon>0$,
		$\exists N\in\mathbb{N}$,
		使得$\forall k,l>N$,均有$\|X_k-X_l\|<\varepsilon$。
	\end{itemize}
}
对$\mathbb{R}^n$中的点列$\{X_k\}$及其极限$A\in\mathbb{R}^n$,约定记号:$X_k=(x_k^{(1)},\ldots,x_k^{(n)})$,$A=(a^{(1)},\ldots,a^{(n)})$。

\di[点列收敛的分量判别]{
	$\lim\limits_{k\rightarrow\infty}X_k=A$当且仅当对于任意的整数$1\le j\le n$,均有$\lim\limits_{k\rightarrow\infty}x_k^{(j)}=a^{(j)}$。
}
\begin{proofs}
	(1)\textbf{必要性。}由题设知,
	$\forall \varepsilon>0$,$\exists N\in\mathbb{N}$,使得
	$\forall k>N$,我们有$\|X_k-A\|<\varepsilon$,因而对任意的
	$1\le j\le n$,我们有$|x_k^{(j)}-a^{(j)}|\le \|X_k-A\|<\varepsilon$,
	也即我们有$\lim\limits_{k\rightarrow\infty}x_k^{(j)}=a^{(j)}$。
	
	(2)\textbf{充分性。}由题设可得知,
	$\forall \varepsilon>0$以及$1\le j\le n$,
	$\exists N_j\in\mathbb{N}$, 使得$\forall k>N_j$,
	均有$|x_k^{(j)}-a^{(j)}|<\frac{\varepsilon}{\sqrt{n}}$。令$N=\max\limits_{1\le j\le n}N_j$。
	则$\forall k>N$,我们有
	$\|X_k-A\|=\sqrt{\sum\limits_{j=1}^n|x_k^{(j)}-a^{(j)}|^2}<\varepsilon$,
	故$\lim\limits_{k\rightarrow\infty}X_k=A$。	
\end{proofs}

由此,收敛数列中与偏序关系无关的性质都可以推广到收敛点列。如
\begin{corollary}
	$\{X_k\}$为\textup{Cauchy}序列当且仅当对任意$1\le j\le n$,$\left\{x_k^{(j)}\right\}$均为\textup{Cauchy}数列。
\end{corollary}
\begin{corollary}
	$\mathbb{R}^n$完备,即$\mathbb{R}^n$中的\textup{Cauchy}列必收敛。
\end{corollary}

\begin{theorem}
	$\varOmega$中任意收敛点列$\{X_k\}$的极限$A \in \varOmega$,当且仅当$\varOmega$为闭集。
\end{theorem}
\begin{theorem}
	设$S\subseteq \mathbb{R}^n$,$A\in\mathbb{R}^n$,则$A$为$S$的极限点当且仅当$S\setminus\{A\}$中有点列$\{X_k\}$收敛到$A$。
\end{theorem}
\begin{proofs}
	(1)\textbf{必要性。}  若$A$为$S$的极限点,则$\forall k\ge 1$,
	$\exists X_k\in\mathring{B}\left(A,\dfrac{1}{k}\right)\cap S$,
	即$X_k\in S\setminus\{A\}$,
	$\|X_k-A\|<\dfrac{1}{k}$。	
	于是由夹逼原理可知点列$\{X_k\}$收敛到$A$。

	(2)\textbf{充分性。}  若$S\setminus\{A\}$中有点列$\{X_k\}$收敛到$A$,
	则$\forall \varepsilon>0$,$\exists N>0$ 使得$\forall k>N$,我们均有
	$\|X_k-A\|<\varepsilon$。
	由于$X_k\in S\setminus\{A\}$,故$X_k\in\mathring{B}(A,\varepsilon)\cap S$,
	由此立刻可知$A$为$S$的极限点。	
\end{proofs}

\begin{definition}
	设$\varOmega\subseteq\mathbb{R}^n$为非空集合,定义
	\begin{itemize}
	\item 令$d(\varOmega)=\sup\limits_{X, Y\in \varOmega}\|X-Y\|$,称为$\varOmega$的\textbf{直径}。

	\item 若$\varOmega$包含在某个(有限的)球中,则称$\varOmega$\textbf{有界}。
	\end{itemize}
	集合有界当且仅当它包含在某个以原点为中心的球中,当且仅当其直径有限。
	
	特别地,称$\mathbb{R}^n$中的点列$\{X_k\}$有界,若它们组成的集合有界,即$\exists r>0$ 使$\forall k\ge 1$,$\|X_k\|<r$。
\end{definition}

\di[闭集套定理]{
	设$\{F_k\}$为$\mathbb{R}^n$中的非空闭集组成的集列使得$F_1\supseteq F_2\cdots \supseteq F_k\supseteq\cdots$。若$\lim\limits_{k\rightarrow\infty}d(F_k)=0$,则交集$\bigcap\limits_{k=1}^{\infty}F_k$为单点集。
}
\di[Weierstrass定理]{
	$\mathbb{R}^n$中有界点列必有收敛子点列。
}

\subsection[$n$元函数与$n$元向量值函数]{$\boldsymbol{n}$\,元函数与\,$\boldsymbol{n}$\,元向量值函数}

\den[$\boldsymbol{n}$\,元向量值函数]{
	设$m,n\ge 1$为整数,$\varOmega\subseteq \mathbb{R}^n$为非空集。
	称任意映射$\boldsymbol{f}:\varOmega\rightarrow\mathbb{R}^m$为$\varOmega$上的$\mboba{n}$\,\tboba{元向量值函数},当$m=1$时,称为$\mboba{n}$\,\tboba{元数量值函数},简称为$\mboba{n}$\,\tboba{元函数}。
}
\begin{definition}
	对向量值函数,定义以下运算:
	\begin{description}
		\item[线性组合] 设$\boldsymbol{f},\boldsymbol{g}:\varOmega\rightarrow\mathbb{R}^m$为
		向量值函数,
		而$\lambda,\mu\in\mathbb{R}$。$\forall X\in \varOmega$,定义		
		$$(\lambda \boldsymbol{f}+\mu \boldsymbol{g})(X):=\lambda \boldsymbol{f}(X)+\mu \boldsymbol{g}(X)$$
		
		\item[乘、除法] 设$\boldsymbol{f}:\varOmega\rightarrow\mathbb{R}^m$为向量值函数,		
		而$g:\varOmega\rightarrow\mathbb{R}$为函数。$\forall X\in \varOmega$,定义
		\begin{equation*}
			(g\boldsymbol{f})(X):=g(X)\boldsymbol{f}(X),\qquad
			\left(\frac{\boldsymbol{f}}{g}\right)(X):\xlongequal{g(X)\neq 0} \frac{\boldsymbol{f}(X)}{g(X)}
		\end{equation*}

		\item[复合运算] 假设$l,m,n\ge1$为整数,
		$\varOmega_1\subseteq\mathbb{R}^n$,
		$\varOmega_2\subseteq \mathbb{R}^m$,$\boldsymbol{f}:\varOmega_1\rightarrow\varOmega_2$,
		$\boldsymbol{g}:\varOmega_2\rightarrow\mathbb{R}^l$为向量值
		函数。$\forall X\in\varOmega_1$,定义
		$$(\boldsymbol{g}\circ\boldsymbol{f})(X):=\boldsymbol{g}(\boldsymbol{f}(X))$$
	\end{description}
\end{definition}

设$\boldsymbol{f}:\varOmega\rightarrow\mathbb{R}^m$为$n$元
向量值函数,则$\forall X=(x_1,\cdots,x_n)\tp \in\varOmega$,均有
$\boldsymbol{f}(X)\in\mathbb{R}^m$,
记作$Y=(y_1,\cdots,y_m)\tp$,每个$y_j$
为$X$的函数:  $y_j=f_j(X)=f_j(x_1,\cdots,x_n)\tp$。
故$\boldsymbol{f}:\varOmega\rightarrow\mathbb{R}^m$与$m$个$n$元
函数$f_j:\varOmega\rightarrow\mathbb{R}$
等价。此时记作$\boldsymbol{f}=(f_1,\cdots,f_m)\tp$。

\subsubsection{多元函数的极限}

\de[向量值函数的极限]{
	设$m,n\ge 1$为整数,$\varOmega\subseteq \mathbb{R}^n$为非空集,$X_0\in\mathbb{R}^n$为$\varOmega$的极限点,$\boldsymbol{f}:\varOmega\rightarrow\mathbb{R}^m$为向量值函数,$A\in\mathbb{R}^m$。
	若
	\begin{center}
	\vspace{-0.5em}
	$\forall \varepsilon>0$,$\exists \delta>0$,使得$\forall X\in\varOmega$,当$0<\|X-X_0\|_n<\delta$时,$\|\boldsymbol{f}(X)-A\|_m<\varepsilon$
	\vspace{-0.5em}
	\end{center}
	则称$X$在$\varOmega$内趋于$X_0$时,
	$\boldsymbol{f}(X)$以$A$为\tboba{极限}(或\tboba{收敛到}~$A$),记作$\mboba{\lim\limits_{\varOmega\ni X\rightarrow X_0}f\hspace{-0.62em}f(X)=A}$。
}
这个定义等价于,$\lim\limits_{\varOmega\ni X\rightarrow X_0}\boldsymbol{f}(X)=A$
当且仅当
\begin{center}
\vspace{-0.5em}
$\forall \varepsilon>0$,
$\exists \delta>0$,
使得$\forall X\in \mathring{B}(X_0,\delta)\cap\varOmega$,
有$\boldsymbol{f}(X)\in B(A,\varepsilon)$
\vspace{-0.5em}
\end{center}
若记$\boldsymbol{f}=(f_1,\ldots,f_m)\tp$,$A=(a_1,\ldots,a_m)\tp$,
那么$\lim\limits_{\varOmega\ni X\rightarrow X_0}\boldsymbol{f}(X)=A$当且仅当
对于任意的$1\le j\le m$,均有$\lim\limits_{\varOmega\ni X\rightarrow X_0}f_j(X)=a_j$。

如果点$X_0$为$\varOmega\cup \{X_0\}$的内点,我们通常将
$\lim\limits_{\varOmega\ni X\rightarrow X_0}\boldsymbol{f}(X)$简记作$\boldsymbol{\lim\limits_{X\rightarrow X_0}f\hspace{-0.62em}f(X)}$。

\begin{theorem}
	向量值函数的极限有以下基本性质:
	\begin{itemize}
		\item 极限的唯一性、保序性、保号性及夹逼原理均仍成立。
		
		\item \textbf{四则运算\quad}设$f,g:\varOmega\subseteq \mathbb{R}^n\rightarrow \mathbb{R}$,
		$\lambda,\mu\in\mathbb{R}$,
		$\lim\limits_{\varOmega\ni X\rightarrow X_0}f(X)=A$,
		$\lim\limits_{\varOmega\ni X\rightarrow X_0}g(X)=B$存在,则
		\begin{align*}
			&\lim\limits_{\varOmega\ni X\rightarrow X_0}(\lambda f+\mu g)(X)=\lambda A+\mu B,\\[-10pt]
			&\lim\limits_{\varOmega\ni X\rightarrow X_0}(fg)(X)=AB,\qquad
			\lim\limits_{\varOmega\ni X\rightarrow X_0}\left(\dfrac{f}{g}\right)(X) \xlongequal{B\neq 0} \dfrac{A}{B}
		\end{align*}

		\item \textbf{复合法则\quad}假设$l,m,n\ge 1$为整数,
		$\varOmega_1\subseteq\mathbb{R}^n$,
		$\varOmega_2\subseteq\mathbb{R}^m$非空,
		而$\boldsymbol{f}: \varOmega_1\rightarrow\varOmega_2$,
		$\boldsymbol{g}:\varOmega_2\rightarrow\mathbb{R}^l$
		为向量值函数。
		如果$\lim\limits_{\varOmega_1\ni X\rightarrow X_0}\boldsymbol{f}(X)=Y_0$,且$\exists \delta > 0$使$\forall X \in \mathring{B}(X_0,\delta) \cap \varOmega_1$,$\boldsymbol{f}(X) \neq Y_0$,而$\lim\limits_{\varOmega_2\ni Y\rightarrow Y_0}\boldsymbol{g}(Y)=A$,则
		$\lim\limits_{\varOmega_1\ni X\rightarrow X_0}(\boldsymbol{g}\circ\boldsymbol{f})(X)=A$。

		\item \textbf{点列极限与函数极限的关系\quad}设$m,n\ge1$为整数,
		$\varOmega\subseteq\mathbb{R}^n$
		非空,$\boldsymbol{f}:\varOmega\rightarrow\mathbb{R}^m$为向量值函数,
		而$X_0\in\mathbb{R}^n$,
		$A\in\mathbb{R}^m$。那么$\lim\limits_{\varOmega\ni X\rightarrow X_0}\boldsymbol{f}(X)=A$
		当且仅当
		对$\varOmega\setminus\{X_0\}$中收敛到$X_0$的任意点列$\{X_k\}$,
		均有$\lim\limits_{k\rightarrow \infty}\boldsymbol{f}(X_k)=A$。
		
		\item \textbf{\textup{Cauchy}准则\quad}$\lim\limits_{\varOmega\ni X\rightarrow X_0}\boldsymbol{f}(X)$
		收敛当且仅当
		$\forall \varepsilon>0$,$\exists \delta>0$ 使$\forall X',X''\in \mathring{B}(X_0,\delta)\cap\varOmega$,
		均有$\|\boldsymbol{f}(X')-\boldsymbol{f}(X'')\|_m<\varepsilon$。
	\end{itemize}	
\end{theorem}

计算多变量函数的极限通常很复杂,目前唯一有效方法是将之转化成单变量函数极限。出于简便记号,后面我们将只讨论两个变量的函数极限$\lim\limits_{(x,y)\rightarrow (x_0,y_0)}f(x,y)$,称为\textbf{二重极限}。我们也可考虑极限$\lim\limits_{x\rightarrow x_0\atop y\rightarrow y_0}f(x,y)$。
由此我们还可以考虑单侧极限以及$x_0$或$y_0$为无穷的情形,比如$\lim\limits_{x\rightarrow x_0\atop y\rightarrow\infty}f(x,y)$,
$\lim\limits_{x\rightarrow \infty\atop y\rightarrow\infty}f(x,y)$,
$\lim\limits_{x\rightarrow \infty\atop y\rightarrow +\infty}f(x,y)$等。

\begin{example}
	计算$\lim\limits_{x\rightarrow \infty \atop y\rightarrow a}\left(1-\dfrac{1}{2x}\right)^{\frac{x^2}{x+y}}$,其中$a\in\mathbb{R}$。
\end{example}
\begin{solution}
	$\lim\limits_{x\rightarrow \infty \atop y\rightarrow a}\ln \left(1-\dfrac{1}{2x}\right)^{\frac{x^2}{x+y}}
	=\lim\limits_{x\rightarrow \infty \atop y\rightarrow a}\dfrac{x^2}{x+y}\ln \left(1-\dfrac{1}{2x}\right)
	=\lim\limits_{x\rightarrow \infty \atop y\rightarrow a}\dfrac{x^2}{x+y}\cdot \left(-\dfrac{1}{2x}\right)
	=\lim\limits_{x\rightarrow \infty \atop y\rightarrow a}\left(-\dfrac{1}{2}\right)\cdot \dfrac{1}{1+\dfrac{y}{x}}
	=-\dfrac{1}{2}$,故$\lim\limits_{x\rightarrow \infty \atop y\rightarrow a}\left(1-\dfrac{1}{2x}\right)^{\frac{x^2}{x+y}}=\e^{-\frac{1}{2}}$。
\end{solution}

\begin{example}
	试证明$f(x,y)=\dfrac{x^2}{x^2+y^2-x}$在$(x,y)$沿任何直线趋于$(0,0)$时,均会趋于$0$,但是当$(x,y)$趋于$(0,0)$时,极限却不存在。
\end{example}
\begin{solution}\adjline
	假设$a,b\in\mathbb{R}$不全为零。对于过$(0,0)$的任意直线
	$\begin{cases}
	x=at,\\[-3pt]
	y=bt	
	\end{cases}$\,($t\in\mathbb{R}$),我们有
	\begin{equation*}
		\lim_{t\rightarrow 0}f(at,bt)=\lim_{t\rightarrow 0}\frac{(at)^2}{(at)^2+(bt)^2-at}=\lim_{t\rightarrow 0}\frac{a^2}{a^2+b^2-\frac{a}{t}}=0
	\end{equation*}
	$\forall t\in\mathbb{R}$,
	定义$g(t)=(t^2,t)$。
	那么$\lim\limits_{t\rightarrow 0}g(t)=(0,0)$,
	且$g$在$\mathbb{R}\setminus\{0\}$上不等于$(0,0)$。注意到
	\begin{equation*}
	\lim_{t\rightarrow 0}f\circ g(t)=\lim_{t\rightarrow 0}\frac{(t^2)^2}{(t^2)^2+t^2-t^2}=1\neq 0
	\end{equation*}
	于是由复合函数极限法则可知 极限
	$\lim\limits_{(x,y)\rightarrow (0,0)}f(x,y)$
	不存在。
\end{solution}

\di[二重极限与累次极限的关系]{
	设有$\lim\limits_{(x,y)\rightarrow (x_0,y_0)}f(x,y)=A$,且在$x_0$的某去心邻域$U$内$\lim\limits_{y\rightarrow y_0}f(x,y)=\varphi(x)$存在,则$A=\lim\limits_{x\rightarrow x_0}\varphi(x)=\lim\limits_{x\rightarrow x_0}\lim\limits_{y\rightarrow y_0}f(x,y)$。
}
\begin{proofs}
	仅以$A\in\mathbb{R}$为例,由极限的定义可知,$\forall \varepsilon>0$,$\exists \delta>0$使得$\forall (x,y)\in\mathring{B}\bigl((x_0,y_0),\delta\bigr)$,均有$|f(x,y)-A|<\varepsilon$。则$\forall x\in U\cap (x_0-\delta,x_0+\delta)$,对$y$取极限可得$|\varphi(x)-A|\le \varepsilon$。故$\lim\limits_{x\rightarrow x_0}\varphi(x)=A$。
\end{proofs}
\begin{corollary}
	若二重极限与某一个累次极限均存在,则二者必然相等;即
	若$\lim\limits_{(x,y)\rightarrow (x_0,y_0)}f(x,y)=A$且
	$\lim\limits_{x\rightarrow x_0}\lim\limits_{y\rightarrow y_0}f(x,y)=B$存在,则$A=B$。
\end{corollary}
\begin{corollary}
	若累次极限存在但不相等,则二重极限不存在;即若$\lim\limits_{x\rightarrow x_0}\lim\limits_{y\rightarrow y_0}f(x,y)$
	与$\lim\limits_{y\rightarrow y_0}\lim\limits_{x\rightarrow x_0}f(x,y)$
	均存在但不相等,则$\lim\limits_{(x,y)\rightarrow (x_0,y_0)}f(x,y)$不存在。
\end{corollary}

\subsubsection{多元函数的连续性}

\de[向量值函数的连续性]{
	假设$m,n\ge 1$为整数,
	$\varOmega\subseteq\mathbb{R}^n$,
	$X_0\in\varOmega$
	为$\varOmega$的极限点,$\boldsymbol{f}:\varOmega\rightarrow\mathbb{R}^m$为向量值函数。若
	\begin{equation*}
	\lim_{\varOmega\ni X\rightarrow X_0}\boldsymbol{f}(X)=\boldsymbol{f}(X_0)
	\end{equation*}
	则称$\boldsymbol{f}$在点$X_0$处\tboba{连续}。
}
$\boldsymbol{f}$在点$X_0$连续
当且仅当
\begin{center}
\vspace{-0.5em}
	$\forall \varepsilon>0$,$\exists \delta>0$使得$\forall X\in \varOmega$,当$\|X-X_0\|_n<\delta$时,均有$\|\boldsymbol{f}(X)-\boldsymbol{f}(X_0)\|_m<\varepsilon$
\vspace{-0.5em}
\end{center}
若点$X_0$不为$\varOmega$的极限点,上述性质恒成立,此时我们也称$\boldsymbol{f}$在点$X_0$处连续。

若$\boldsymbol{f}:\varOmega\rightarrow\mathbb{R}^m$在$\varOmega$的每点连续,则称$\boldsymbol{f}$~\tboba{在\,$\mboba{\varOmega}$\,上连续},记作$\boldsymbol{f} \in \mathscr{C}(\varOmega; \mathbb{R}^m)$,即定义$\mathscr{C}(\varOmega; \mathbb{R}^m)=\{\boldsymbol{f}\ |\ \boldsymbol{f}:\varOmega\rightarrow\mathbb{R}^m~\text{连续}\}$。$m=1$时,简记为$f \in \mathscr{C}(\varOmega)$。

\di[开集上连续函数的等价性质]{
	设$\varOmega\subset \mathbb{R}^n$为开集,而$\boldsymbol{f}:\varOmega\rightarrow\mathbb{R}^m$为向量值函数。
	则$\boldsymbol{f}$连续当且仅当对$\mathbb{R}^m$中任意
	开集$G$,原像集$\boldsymbol{f}^{-1}(G)=\{x\in\varOmega\ |\ \boldsymbol{f}(x)\in G\}$
	均为开集。
}
\begin{proofs}
	\hang[2](1)\textbf{充分性。}假设对于$\mathbb{R}^m$中的任意开集$G$,
	其原像集$\boldsymbol{f}^{-1}(G)$为开集。取$X_0\in \varOmega$,$\forall \varepsilon>0$,
	令$G=B(\boldsymbol{f}(X_0),\varepsilon)$,由题设知$\boldsymbol{f}^{-1}(G)$为包含
	点$X_0$的开集,则$\exists \delta>0$使$B(X_0,\delta)\subseteq \boldsymbol{f}^{-1}(G)$,
	于是$\forall X\in B(X_0,\delta)$,均有$\|\boldsymbol{f}(X)-\boldsymbol{f}(X_0)\|_m<\varepsilon$。
	因此$\boldsymbol{f}$ 在点$X_0$处连续,从而$\boldsymbol{f}$为连续映射。

	\hang[2](2)\textbf{必要性。}  假设$\boldsymbol{f}$为连续函数,而$G$为$\mathbb{R}^m$中的任意非空开集。
	$\forall X_0\in \boldsymbol{f}^{-1}(G)$,均有$\boldsymbol{f}(X_0)\in G$。
	又$G$为开集,则$\exists \varepsilon>0$ 使得$B(\boldsymbol{f}(X_0),\varepsilon)\subseteq G$。
	$\boldsymbol{f}$在$X_0$连续,则$\exists \delta_1>0$
	使$\forall X\in\varOmega\cap B(X_0,\delta_1)$,
	我们有$\|\boldsymbol{f}(X)-\boldsymbol{f}(X_0)\|_m<\varepsilon$。又$\varOmega\cap B(X_0,\delta_1)$
	为开集,故$\exists \delta>0$使$B(X_0,\delta)\subseteq \varOmega\cap B(X_0,\delta_1)$,
	则$\forall X\in B(X_0,\delta)$,均有$\|\boldsymbol{f}(X)-\boldsymbol{f}(X_0)\|_m<\varepsilon$,也即有$B(X_0,\delta)\subseteq \boldsymbol{f}^{-1}\left(B(\boldsymbol{f}(X_0),\varepsilon)\right)\subseteq \boldsymbol{f}^{-1}(G)$,
	故$X_0$为$\boldsymbol{f}^{-1}(G)$的内点,进而$\boldsymbol{f}^{-1}(G)$为开集。
\end{proofs}
同理可证$\boldsymbol{f}$连续当且仅当对于$\mathbb{R}^m$中任意
闭集$G$,原像集$\boldsymbol{f}^{-1}(G)$为闭集。

\di[有界闭集上连续数量值函数的最值定理]{
	假设$\varOmega\subseteq \mathbb{R}^n$为有界闭集,而$f\in\mathscr{C}(\varOmega)$,则$f$在$\varOmega$上有最大值和最小值。
}
\begin{proofs}\adjline
	\hang 首先证明\textbf{$\boldsymbol{f}$在$\boldsymbol{\varOmega}$上有界}。
	否则,$\forall k\in\mathbb{N}^{*}$,
	$\exists X_k\in\varOmega$使得$|f(X_k)|>k$。
	由$\varOmega$的有界性可知
	$\{X_k\}$有一个子列$\{X_{\ell_k}\}$收敛,设其极限为$A$。
	又$\varOmega$为闭集,则$A\in \varOmega$,再由$f$的连续性以及
	夹逼原理可得$f(A)=\lim\limits_{k\rightarrow \infty}f(X_{\ell_k})=\infty$。矛盾!
	故假设不成立,从而$f$有界。

	\hang 下证\textbf{$\boldsymbol{f}$在$\boldsymbol{\varOmega}$上有最值}。用反证法,假设$f$没有
	最大值或最小值。不失一般性,可假设$f$没有
	最大值,否则可以考虑$-f$。令$M=\sup f(\varOmega)$。
	则$\forall X\in \varOmega$,$f(X)<M$。定义$F(X)=\dfrac{1}{M-f(X)}$,
	则$F\in\mathscr{C}(\varOmega)$。又由$M$的定义
	可知,$\forall k\in\mathbb{N}^{*}$,
	$\exists X_k\in \varOmega$使得$f(X_k)>M-\dfrac{1}{k}$,故$F(X_k)>k$,
	从而$F$在$\varOmega$上没有上界。矛盾!故所证成立。
\end{proofs}

继续讨论连通性,我们再定义
\de[弧连通集]{
	设$D\subseteq\mathbb{R}^n$,如果$\forall X,Y\in D$,
	均存在$D$中的连续曲线将$X,Y$连接起来,
	即存在向量值连续函数$\boldsymbol{\gamma}: [0,1]\rightarrow D$ 使得我们有$\boldsymbol{\gamma}(0)=X$,$\boldsymbol{\gamma}(1)=Y$,则称集合$D$~\tboba{弧连通}。
}
则知定义~\ref{连通集}~定义的折线连通集也为弧连通集。由连续函数的复合依然连续,若$f\in\mathcal{C}(\varOmega;\mathbb{R}^m)$而$\varOmega\subseteq\mathbb{R}^n$为弧连通,则$f(\varOmega)$为弧连通集。

\di[连通集上连续数量值函数的介值定理]{
	假设$\varOmega\subseteq \mathbb{R}^n$为弧连通集,
	而$f\in\mathscr{C}(\varOmega)$,则$\forall X_1,X_2\in \varOmega$以及介于$f(X_1)$,$f(X_2)$之间的任意实数$\mu$,$\exists X_0\in \varOmega$使得$f(X_0)=\mu$。
}
\begin{example}
	证明: 存在正实数$m,M$使得对于任意的$X=(x_1,\ldots,x_n)\in\mathbb{R}^n$,均有
	$$m\sum\limits_{j=1}^n|x_j|\le \|X\|\le M\sum\limits_{j=1}^n|x_j|$$
\end{example}
\begin{proofs}
	\adjline
	定义$S=\{Y\in\mathbb{R}^n\ |\ \|Y\|_n=1\}$,则$S$为
	有界闭集。$\forall Y=(y_1,\ldots,y_n)\in S$,令$f(Y)=\sum\limits_{j=1}^n|y_j|>0$,则$f$连续,从而有最小值$a>0$,最大值$b$。
	选取$m=\dfrac{1}{b}$,$M=\dfrac{1}{a}$。

	$\forall X\in \mathbb{R}^n\setminus\{\boldsymbol{0}\}$,
	都有$Y_X=\dfrac{1}{\|X\|_n}(x_1,\cdots,x_n)\in S$,则$a\le f(Y_X)\le b$,
	也即$a\le \dfrac{1}{\|X\|_n}\sum\limits_{j=1}^n|x_j|\le b$,从而我们有
	$\dfrac{1}{b}\sum\limits_{j=1}^n|x_j|\le \|X\|_n\le \dfrac{1}{a}\sum\limits_{j=1}^n|x_j|$,
	也就是说我们有$m\sum\limits_{j=1}^n|x_j|\le \|X\|_n\le M
	\sum\limits_{j=1}^n|x_j|$。

	而$X$为零向量时,该式也成立,故所证成立。
\end{proofs}

\newpage
%--------------------------------------------------------------
\section{多元函数微分学}

\subsection{偏导数与全微分}

\subsubsection{多元函数的可微性}

\begin{definition}
	设$n\ge 1$为整数,$\varOmega\subseteq \mathbb{R}^n$,而$X_0\in\mathbb{R}^n$	为$\varOmega$的极限点,$f:\varOmega\rightarrow\mathbb{R}$为函数。	
	
	\begin{itemize}[leftmargin=1em]
		\item 若$\lim\limits_{\varOmega\ni X\rightarrow X_0}f(X)=0$,
		称$f$在
		$\varOmega\ni X\rightarrow X_0$时
		为\textbf{无穷小函数}(或\textbf{无穷小量}),记作
		$$f(X)=o(1)\ (\varOmega\ni X\rightarrow X_0)$$
		可见$\lim\limits_{\varOmega\ni X\rightarrow X_0}f(X)=A$当且仅当
		$f(X)-A=o(1)\ (\varOmega\ni X\rightarrow X_0)$。
		
		\item 设$g:\varOmega\rightarrow\mathbb{R}$为函数,
		若存在$\beta>0$,$\delta>0$
		使$\forall X\in \varOmega\cap\mathring{B}(X_0,\delta)$,
		$|f(X)|\le\beta |g(X)|$,则记
		$$f(X)=O(g(X))\ (\varOmega\ni X\rightarrow X_0)$$
		若还有$g(X)=O(f(X))$,则称$f,g$为\textbf{同阶}。
		
		\item 设$k\ge 0$,
		若$\lim\limits_{\varOmega\ni X\rightarrow X_0}\dfrac{f(X)}{\|X-X_0\|^k}=0$,
		则称$f$在
		$\varOmega\ni X\rightarrow X_0$时为$\|X-X_0\|^k$的\textbf{高阶的无穷小},
		记作$$f(X)=o(\|X-X_0\|^k)\ (\varOmega\ni X\rightarrow X_0)$$

		\item 若$\lim\limits_{\varOmega\ni X\rightarrow X_0}\dfrac{f(X)}{\|X-X_0\|^k}=c\neq 0$,
		则我们称$f$在
		$\varOmega\ni X\rightarrow X_0$时为$\|X-X_0\|$的$\boldsymbol{k}$\,\textbf{阶的无穷小},
		此时$f$局部常号。若$k=0$,
		则$\lim\limits_{\varOmega\ni X\rightarrow X_0}f(X)=c$,
		因此我们通常不考虑$0$阶无穷小。
	\end{itemize}
\end{definition}

\begin{definition}
	称$L:\mathbb{R}^n\rightarrow\mathbb{R}$为\textbf{线性函数},
	若$\forall X,Y\in\mathbb{R}^n$
	以及$\forall \lambda,\mu\in\mathbb{R}$,我们均有
	$L(\lambda X+\mu Y)=\lambda L(X)+\mu L(Y)$。
	
\end{definition}

\de[全微分]{
	假设$X_0=\left(x_1^{(0)},\cdots,x_n^{(0)}\right)\in\mathbb{R}^n$,$r>0$,
	而$f: B(X_0,r)\subseteq\mathbb{R}^n\rightarrow \mathbb{R}$为函数。
	若存在线性函数$L:\mathbb{R}^n\rightarrow \mathbb{R}$ 使得当$X\rightarrow X_0$时,我们有
	\begin{equation}
		f(X)-f(X_0)= L(X-X_0)+o(\|X-X_0\|)
	\end{equation}
	则称$f$在点$X_0$处\tboba{可微}, 并将线性函数$L$记作
	$\dif f(X_0)$,称为$f$在点$X_0$处的\tboba{全微分}或\tboba{微分}。
}

\zhu[全微分的线性表示]{
	设$\hat{\boldsymbol{e}}_1,\cdots,\hat{\boldsymbol{e}}_n$为$\mathbb{R}^n$的自然基底,令$a_j=L(\hat{\boldsymbol{e}}_j)$。
	$\forall X=(x_1,\cdots,x_n)\tp \in\mathbb{R}^n$,
	我们有$X=\sum\limits_{j=1}^nx_j\hat{\boldsymbol{e}}_j$,
	由此可得$L(X)=\sum\limits_{j=1}^nL(\hat{\boldsymbol{e}}_j)x_j=\sum\limits_{j=1}^na_jx_j$。$\forall X=(x_1,\cdots,x_n)\tp \in\mathbb{R}^n$,我们有
	\begin{equation*}
		L(X)=(a_1,\cdots,a_n)
		\begin{pmatrix}
			x_1\\	\vdots\\	x_n
		\end{pmatrix} =(a_1,\cdots,a_n)X
	\end{equation*}
	于是线性函数$L:\mathbb{R}^n\rightarrow\mathbb{R}$可以与$n$阶行向量
	$(a_1,\cdots,a_n)$视为等同,故$f$在点$X_0$处可微
	当且仅当$\exists a_1,\ldots,a_n\in\mathbb{R}$ 使得$X\rightarrow X_0$时,
	\begin{equation*}
	f(X)-f(X_0)=L(X-X_0)+o(\|X-X_0\|)=\mboqi{\sum_{j=1}^na_j(x_j-x_j^{(0)})}+o(\|X-X_0\|)
	\end{equation*}
}

$f$在点$X_0$可微蕴含在该点连续,反之不对。

\begin{theorem}
	设$a_1,\cdots,a_n\in\mathbb{R}$,
	$L:\mathbb{R}^n\rightarrow\mathbb{R}$线性使得
	$\forall Y=(y_1,\cdots,y_n)\in\mathbb{R}^n$,
	均有$L(Y)=\sum\limits_{j=1}^na_jy_j$,则$L=\sum\limits_{j=1}^na_j \dif x_j$。
\end{theorem}
\begin{proofs}
	定义$ f: \mathbb{R}^n\rightarrow \mathbb{R}$ 为$ f(X)=x_j$,则在任意固定点$X_0=\left(x_1^{(0)},\cdots,x_n^{(0)}\right)$处,
	$f(X)-f(X_0)=x_j-x_j^{(0)}=(\underbrace{0,\cdots,0,1}_{j} ,0,\cdots,0)(X-X_0)+0$。
	从而由微分的定义可知
	\vspace{-2em}
	$$ \dif f(X_0)=\dif x_j^{(0)}= (\underbrace{0,\cdots,0,1}_{j} ,0,\cdots,0)=\hat{\boldsymbol{e}}_j \vspace{-1em}$$
	也即
	$\dif x_j= \hat{\boldsymbol{e}}_j$。
	由于$L(Y)=\sum\limits_{j=1}^na_jy_j=\sum\limits_{j=1}^na_jy_j\hat{\boldsymbol{e}}_j
	=\sum\limits_{j=1}^na_j \dif x_j(Y)$,因此
	所证结论成立。
\end{proofs}
\begin{theorem}
	假设$\varOmega\subseteq \mathbb{R}^n$为开集,$X_0\in \varOmega$,而函数
	$f,g:\varOmega\rightarrow\mathbb{R}$在点$X_0$可微。则下列性质成立:
	\begin{itemize}
		\item $\forall \lambda,\mu\in\mathbb{R}$,$\lambda f+\mu g$在点$X_0$处可微,并且
		$\dif (\lambda f+\mu g)(X_0)=\lambda \dif f(X_0)+\mu \dif g(X_0)$;
		
		\item $fg$在点$X_0$处可微,并且
		$\dif (fg)(X_0)=f(X_0) \dif g(X_0)+g(X_0) \dif f(X_0)$,
		
		\item 若$g(X_0)\neq 0$,则$\dfrac{f}{g}$在点$X_0$处可微,并且	
		$\dif \left(\dfrac{f}{g}\right)(X_0)=\dfrac{g(X_0)\dif f(X_0)-f(X_0)\dif g(X_0)}{(g(X_0))^2}$。
	\end{itemize}
\end{theorem}

\subsubsection{多元函数的可导性}

\de[偏导数]{
	设$X_0=\left(x_1^{(0)},\cdots,x_n^{(0)}\right)\in\mathbb{R}^n$,而函数$f$
	定义在点$X_0$的某邻域上。固定$1\le j\le n$。若
	\begin{align}
		\dfrac{\partial f}{\partial x_j}(X_0)&:=\lim_{h\rightarrow 0}\dfrac{f\left(x_1^{(0)},\cdots,x_{j-1}^{(0)},\mboba{x_j^{(0)}+h},x_{j+1}^{(0)},\cdots,x_n^{(0)}\right)-f(X_0)}{h} \nonumber\\[-3pt]
		&= \lim_{h\rightarrow 0}\dfrac{f(X_0+h\hat{\boldsymbol{e}}_j)-f(X_0)}{h}
	\end{align}
	存在,则称函数$f$在点$X_0$处关于第$j$个变量有\tboba{偏导数}\,$\mboba{\dfrac{\partial f}{\partial x_j}(X_0)}$,通常也会将之记作$\partial_jf(X_0)$或$f'_{x_j}(X_0)$。若对于$1\le j\le n$,偏导数$\dfrac{\partial f}{\partial x_j}(X_0)$
	均存在,则称函数$f$在点$X_0$处\tboba{可导}。
}
偏导数$\dfrac{\partial f}{\partial x_j}(X_0)$实际上表示平面
曲线$y=f\left(x_1^{(0)},\cdots,x_{j-1}^{(0)},\mbome{x},  x_{j+1}^{(0)},\cdots,x_n^{(0)}\right)$
在点$x=x_j^{(0)}$处的切线方向。

\begin{example}
	\textbf{$\boldsymbol{n\ge 2}$时$\boldsymbol{n}$元函数可导不连续的实例\quad}$\forall (x,y)\in\mathbb{R}^2$,定义
	$f(x,y)=\begin{cases}[ll]
		0,&\textrm{if }xy=0,\\[-3pt]
		1,&\textrm{else}
	\end{cases}$,
	则$\dfrac{\partial f}{\partial x}(0,0)=\dfrac{\partial f}{\partial y}(0,0)=0$,但$f$在原点不连续。
\end{example}
\di[可微性蕴含可导性]{
	若$f$在点$X_0$处可微,则它可导,且
	\begin{equation}
		\dif f(X_0)=\sum\limits_{j=1}^n\dfrac{\partial f}{\partial x_j}(X_0)\dif x_j
	\end{equation}
}
\begin{proofs}\adjline
	由题设可知 存在$a_1,\cdots,a_n\in\mathbb{R}$ 使得
	$\forall Y=(y_1,\cdots,y_n)\in\mathbb{R}^n$,我们均有
	$\dif f(X_0)(Y)=\sum\limits_{j=1}^na_jy_j=\sum\limits_{j=1}^na_j\dif x_j(Y)$,
	也即我们有$\dif f(X_0)=\sum\limits_{j=1}^na_j\dif x_j$。

	对任意的$1\le j\le n$,由微分定义,当$h\rightarrow 0$时,
	\begin{equation*}
	f(X_0+h\hat{\boldsymbol{e}}_j)-f(X_0)=\dif f(X_0)(h\hat{\boldsymbol{e}}_j)+o(\|h\hat{\boldsymbol{e}}_j\|)=a_jh+o(|h|)
	\end{equation*}
	由此我们立刻可得
	\begin{equation*}
	\lim_{h\rightarrow 0}\frac{f(X_0+h\hat{\boldsymbol{e}}_j)-f(X_0)}{h}=a_j
	\end{equation*}
	也即$f$在点$X_0$处关于第$j$个变量可导,并且
	$\dfrac{\partial f}{\partial x_j}(X_0)=a_j$,
	故所证结论成立。
\end{proofs}

\begin{example}
	$\forall (x,y)\in\mathbb{R}^2$,定义$f(x,y)=\sqrt{|xy|}$。讨论函数$f$在原点处的连续性,可导性与可微性。
\end{example}
\begin{solution}
	\adjline
	因 $0\le f(x,y)\le \sqrt{\frac{x^2+y^2}{2}}$, 
	则由夹逼原理可知$\lim\limits_{(x,y)\rightarrow (0,0)}f(x,y)=0=f(0,0)$,于是函数$f$在原点处连续。由偏导数的定义知
	\begin{equation*}
	\dfrac{\partial f}{\partial x}(0,0)=\lim_{x\rightarrow 0}\dfrac{f(x,0)-f(0,0)}{x}=0,\qquad
	\dfrac{\partial f}{\partial y}(0,0)=\lim_{y\rightarrow 0}\dfrac{f(0,y)-f(0,0)}{y}=0
	\end{equation*}
	下证$f$在原点不可微。用反证法,设$f$在原点可微,则当$(x,y)\rightarrow (0,0)$时,我们有	
	\begin{equation*}
	f(x,y)-f(0,0)
	=\frac{\partial f}{\partial x}(0,0) x+\frac{\partial f}{\partial y}(0,0)y+o(\sqrt{x^2+y^2})
	=o(\sqrt{x^2+y^2})
	\end{equation*}
	即 $\lim\limits_{(x,y)\rightarrow (0,0)}\dfrac{\sqrt{|xy|}}{\sqrt{x^2+y^2}}=0$。 
	进而由复合函数极限法则可知  $0=\lim\limits_{x\rightarrow 0}\dfrac{\sqrt{|x^2|}}{\sqrt{x^2+x^2}}=\dfrac{\sqrt{2}}{2}$。
	矛盾!由此得证。
\end{solution}

\subsubsection{多元函数的连续可导性}

\de[可导与连续可导]{
	设$\varOmega\subseteq\mathbb{R}^n$为非空开集,而$f: \varOmega\rightarrow\mathbb{R}$。
	\begin{itemize}[leftmargin=1em]
	\item 若$f$在$\varOmega$的每点可导,则称$f$在$\varOmega$上\tboba{可导},
	由此可以在$\varOmega$上定义$n$个函数$\dfrac{\partial f}{\partial x_1},\cdots,\dfrac{\partial f}{\partial x_n}$,
	将它们称为$f$在$\varOmega$上的\tboba{偏导函数}。
	
	\item 若$\dfrac{\partial f}{\partial x_1},\cdots,\dfrac{\partial f}{\partial x_n}$在点$X_0\in \varOmega$处连续,则称$f$
	在点$X_0$处\tboba{连续可导}。
	
	\item 若$f$在$\varOmega$每点均连续可导,则称$f$在$\varOmega$上\tboba{连续可导}。这样函数的集合记作$\mathscr{C}^{(1)}(\varOmega)$。
	\end{itemize}
}

\di[连续可导性蕴含可微性]{
	若$\varOmega\subseteq\mathbb{R}^n$为开集,而函数$f:\varOmega\rightarrow\mathbb{R}$
	在点$X_0\in \varOmega$处连续可导,则$f$在该点可微。
}
\begin{proofs}
	仅考虑$n=2$的情形。由于$f$	在点$X_0$处连续可导,于是$\exists r>0$使得函数$f$在$B(X_0,\sqrt{2}r)$上可导且其偏导函数在点$X_0$处连续。记$X_0=(x_1^{(0)},x_2^{(0)})$。$\forall h_1,h_2\in (-r,r)$,令
	\begin{align*}
		F(h_1,h_2)&=f(x_1^{(0)}+h_1,x_2^{(0)}+h_2)-f(x_1^{(0)},x_2^{(0)})\\
		&=\left(f(x_1^{(0)}+h_1,x_2^{(0)}+h_2)-\mbome{f(x_1^{(0)},x_2^{(0)}+h_2)}\right)+\left(\mbome{f(x_1^{(0)},x_2^{(0)}+h_2)}-f(x_1^{(0)},x_2^{(0)})\right)
	\end{align*}
	由Lagrange中值定理知,$\exists \theta_1,\theta_2\in (0,1)$使得
	\begin{equation*}
	F(h_1,h_2)=\frac{\partial f}{\partial x_1}(x_1^{(0)}+\mbome{\theta_1h_1,}x_2^{(0)}+h_2)h_1
	+\frac{\partial f}{\partial x_2}(x_1^{(0)},x_2^{(0)}+\mbome{\theta_2h_2})h_2
	\end{equation*}
	而由夹逼原理可知
	\begin{equation*}
	\lim_{(h_1,h_2)\rightarrow (0,0)}\theta_1h_1=\lim_{(h_1,h_2)\rightarrow (0,0)}\theta_2h_2=0
	\end{equation*}
	又$f$在点$X_0$连续可导,由复合函数极限法则,
	\begin{align*}
	&\lim_{(h_1,h_2)\rightarrow (0,0)}\frac{\partial f}{\partial x_1}(x_1^{(0)}+\mbome{\theta_1h_1},x_2^{(0)}+h_2)
	=\frac{\partial f}{\partial x_1}(x_1^{(0)},x_2^{(0)}),\\
	&\lim_{(h_1,h_2)\rightarrow (0,0)}\frac{\partial f}{\partial x_2}(x_1^{(0)},x_2^{(0)}+\mbome{\theta_2h_2})
	=\frac{\partial f}{\partial x_2}(x_1^{(0)},x_2^{(0)})
	\end{align*}
	于是当$(h_1,h_2)\rightarrow (0,0)$时,我们有	
	\begin{align*}
	F(h_1,h_2)&=\left(\frac{\partial f}{\partial x_1}(x_1^{(0)},x_2^{(0)})+o(1)\right)h_1
	+\left(\frac{\partial f}{\partial x_2}(x_1^{(0)},x_2^{(0)})+o(1)\right)h_2\\
	&=\frac{\partial f}{\partial x_1}(x_1^{(0)},x_2^{(0)})h_1+\frac{\partial f}{\partial x_2}(x_1^{(0)},x_2^{(0)})h_2+\mbome{o(1)h_1+o(1)h_2}\\
	&=\frac{\partial f}{\partial x_1}(x_1^{(0)},x_2^{(0)})h_1+\frac{\partial f}{\partial x_2}(x_1^{(0)},x_2^{(0)})h_2+\mbome{o(1)\sqrt{h_1^2+h_2^2}}\\
	&=\frac{\partial f}{\partial x_1}(x_1^{(0)},x_2^{(0)})h_1+\frac{\partial f}{\partial x_2}(x_1^{(0)},x_2^{(0)})h_2+\mbome{o(\|(h_1,h_2)\|)}
	\end{align*}
	这表明函数$f$在点$X_0$处可微。
\end{proofs}

\begin{theorem}
	若函数$f:\mathbb{R}^2\rightarrow\mathbb{R}$关于它的第一个变量连续,而关于第二个变量的偏导函数在$\mathbb{R}^2$上有界,求证: 函数$f$在$\mathbb{R}^2$上连续。
\end{theorem}
\begin{proofs}
	\adjline
	由题设可知,$\exists M>0$ 使得$\forall (x,y)\in\mathbb{R}^2$,
	$\left|\dfrac{\partial f}{\partial y}(x,y)\right|\le M$。取$(x_0,y_0)\in\mathbb{R}^2$。
	$\forall (x,y)\in\mathbb{R}^2$,由Lagrange中值定理,存在$\xi$介于$y_0,y$ 使得
	\begin{align*}
		|f(x,y)-f(x_0,y_0)| &\le |\mbome{f(x,y_0)}-f(x_0,y_0)|+|f(x,y)-\mbome{f(x,y_0)}| \\[-5pt]
		&=|\mbome{f(x,y_0)}-f(x_0,y_0)|+\left|\frac{\partial }{\partial y}f(x,\xi)\right||y-y_0| \\[-5pt]
		&\le |f(x,y_0)-f(x_0,y_0)|+M|y-y_0| \qedhere
	\end{align*}
\end{proofs}

\subsubsection{向量值函数的微分}

\begin{definition}
	设$X_0=(x_1^{(0)},\cdots,x_n^{(0)})\in\mathbb{R}^n$,$r>0$,而$\v{f} : B(X_0,r)\subset\mathbb{R}^n\rightarrow \mathbb{R}^m$,
	$g: B(X_0,r)\rightarrow \mathbb{R}$为
	映射。
	若$\lim\limits_{X\rightarrow X_0}\dfrac{\|\v{f}(X)\|}{|g(X)|}=0$, 则记
	$\v{f}(X)=\v{o}(|g(X)|)=|g(X)|\v{o}(1)$($X\rightarrow X_0$),称$\v{f}$是$g$的\textbf{高阶无穷小量}。

	如果记$\v{f}=(f_1,\cdots,f_m)\tp$,则上式成立当且仅当
	对任意的整数$1\le i\le m$,我们均有
	$f_i(X)=o(|g(X)|)$($X\rightarrow X_0$)。
\end{definition}

\de[向量值函数的全微分]{
	假设$X_0=(x_1^{(0)},\cdots,x_n^{(0)})\in\mathbb{R}^n$,$r>0$,
	$\v{f}: B(X_0,r)\subset\mathbb{R}^n\rightarrow \mathbb{R}^m$为向量值函数。如果
	存在线性映射$A: \mathbb{R}^n\rightarrow \mathbb{R}^m$ 使得$X\rightarrow X_0$时,
	\begin{equation}
		\v{f}(X)-\v{f}(X_0)=A(X-X_0)+\v{o}(\|X-X_0\|)
	\end{equation}
	则称$\v{f}$在点$X_0$~\tboba{可微},并将映射$A$记作$\mboba{\dif \V{f}(X_0)}$,
	称为$\v{f}$在点$X_0$的\tboba{全微分}或\tboba{微分}。线性映射$A$
	所对应的矩阵记作$\mboba{J\V{f}(X_0)}$,也记作$J_{\v{f}}(X_0)$,
	称为$\v{f}$在点$X_0$处的~\tboba{Jacobi矩阵}。
}
若记$\v{f}=(f_1,\cdots,f_m)\tp$,
则$A=(a_{ij})_{1\le i\le m\atop 1\le j\le n}$为
$\v{f}$在点$X_0$处的微分 当且仅当 $X\rightarrow X_0$时,
\textbf{对任意的整数$\boldsymbol{1\le i\le m}$},我们均有
\begin{equation*}
f_i(X)-f_i(X_0)=\sum_{j=1}^na_{ij}(x_j-x_j^{(0)})+o(\|X-X_0\|)
\end{equation*}
也即$f_i$在点$X_0$处可微,并且有$a_{ij}=\dfrac{\partial f_i}{\partial x_j}(X_0)$。
故$\dif \v{f}(X_0)$所对应的矩阵的第$i$个行向量正好
对应于$\dif f_i(X_0)$所对应的矩阵。由此可知$\v{f}$在
点$X_0$可微 当且仅当 $f_1,\cdots,f_m$在该点可微 且
\begin{equation}
	\dif\v{f}(X_0)=
	\begin{pmatrix}
		\dif f_1(X_0)\\
		\vdots\\
		\dif f_m(X_0)
	\end{pmatrix}
	=\begin{pmatrix}
		\sum\limits_{j=1}^n\dfrac{\partial f_1}{\partial x_j}(X_0) \dif x_j\\
		\vdots\\
		\sum\limits_{j=1}^n\dfrac{\partial f_m}{\partial x_j}(X_0) \dif x_j
	\end{pmatrix}
	=\mboba{\begin{pmatrix}
		\dfrac{\partial f_1}{\partial x_1}(X_0)&\cdots &\dfrac{\partial f_1}{\partial x_n}(X_0)\\
		\vdots&\cdots &\vdots\\
		\dfrac{\partial f_m}{\partial x_1}(X_0)&\cdots &\dfrac{\partial f_m}{\partial x_n}(X_0)
	\end{pmatrix}}
	\begin{pmatrix}
		\dif x_1\\
		\vdots\\
		\dif x_n
	\end{pmatrix}
\end{equation}
也即
\begin{equation}
	J_{\v{f}}(X_0)=\left(\dfrac{\partial f_i}{\partial x_j}(X_0)\right)_{1\le i\le m\atop 1\le j\le n}
\end{equation}
若将最右边那个列向量记作$\dif X$,则$\dif \v{f}(X_0)=J_{\v{f}}(X_0)\,\dif X$。

通常也将$J_{\v{f}}(X_0)$记作$\mboba{\dfrac{\partial (f_1,\cdots,f_m)}{\partial (x_1,\cdots,x_n)}(X_0)}$或$\dfrac{\partial (f_1,\cdots,f_m)}{\partial (x_1,\cdots,x_n)}\Bigg|_{X_0}$。
当$m=n$时,相应行列式被称为~\tboba{Jacobi行列式},
记作$\mboba{\dfrac{D(f_1,\cdots,f_m)}{D(x_1,\cdots,x_n)}(X_0)}$或$\dfrac{D(f_1,\cdots,f_m)}{D(x_1,\cdots,x_n)}\Bigg|_{X_0}$。

\subsection{特殊导数的计算}

\subsubsection{方向导数}

\de[方向导数]{
	设$\varOmega\subseteq\mathbb{R}^n$为开集,$X_0\in \varOmega$,$f:\varOmega\rightarrow\mathbb{R}$
	为函数,$\v{\beta}\not=0$为向量,$\v{\beta}^0=(\cos \alpha_1,\cdots,\cos\alpha_n)\tp$为其单位向量,即
	$\v{\beta}^0=\dfrac{\v{\beta}}{\|\v{\beta}\|}= (\cos \alpha_1,\cdots,\cos\alpha_n)\tp$,
	其中$\alpha_j$为$\v{\beta}$与$x_j$轴的夹角。若极限
	\begin{equation}
		\lim\limits_{h\rightarrow 0^{+}}\dfrac{f(X_0+h\v{\beta}^0)-f(X_0)}{h}
	\end{equation}
	存在,则称之为$f$在点$X_0$处沿$\v{\beta}$方向的\tboba{方向导数},记作$\mboba{\dfrac{\partial f}{\partial \V{\beta}}(X_0)}$,$\mboba{\dfrac{\partial f}{\partial \V{\beta}}\Bigg|_{X_0}}$或$\mboba{f'_{\V[-0.5]{\beta}}(X_0)}$。
}
方向导数只与方向有关,则显然有$\dfrac{\partial f}{\partial \v{\beta}}(X_0)=\dfrac{\partial f}{\partial \v{\beta}^{0}}(X_0)$。

\begin{definition}
	给定向量 
	$\v{A}=(a_1,a_2,\cdots,a_n),\v{B}=(b_1,b_2,\cdots,b_n)$,
	定义 
	$ \v{A}\cdot \v{B}:=a_1b_1+a_2b_2+ \cdots +a_nb_n$,
	称为两个向量的\textbf{内积}或者\textbf{数量积}。
\end{definition}
\begin{definition}
	对给定的映射$f: \varOmega \to \mathbb{R},\varOmega \subset \mathbb{R}^n,$ 记 
	$$\v{\nabla }f (X_0):=\bigg(\frac{\partial f}{\partial x_1}(X_0),\frac{\partial f}{\partial x_2}(X_0),\cdots,\frac{\partial f}{\partial x_n}(X_0)\bigg)$$
\end{definition}

\di[方向导数的存在性]{
	若$f$在点$X_0$可微,则$\dfrac{\partial f}{\partial \v{\beta}}(X_0)$存在且
	\begin{equation}
	\frac{\partial f}{\partial \v{\beta}}(X_0)=\sum_{j=1}^n\frac{\partial f}{\partial x_j}(X_0)\cos \alpha_j
	=\v{\nabla }f(X_0)\cdot \v{\beta}^0=\v{\nabla} f(X_0)\cdot \frac{\v{\beta}}{\|\v{\beta}\|}
	\end{equation}
}
\begin{proofs}
	当$h\rightarrow 0^{+}$时,
	$f(X_0+\v{\beta}^0h)-f(X_0)=\sum\limits_{j=1}^n\dfrac{\partial f}{\partial x_j}(X_0)h\cos \alpha_j+o(\|\v{\beta}^0h\|)=\bigg(\sum\limits_{j=1}^n\dfrac{\partial f}{\partial x_j}(X_0)\cos \alpha_j\bigg)h+o(|h|)$。
\end{proofs}

\de[数量场的梯度]{
	设$\varOmega\subseteq\mathbb{R}^n$为非空集。定义在$\varOmega$上的实值函数也称为$\varOmega$上的\tboba{数量场}。
	若$\varOmega\subseteq\mathbb{R}^n$为开集,$X_0\in \varOmega$,数量场$f:\varOmega\rightarrow\mathbb{R}$在点$X_0$沿任意方向有方向导数,其中沿$\v{e}$的方向导数的值最大且该值等于$\|\v{e}\|$,则称向量$\v{e}$为$f$在点$X_0$的\tboba{梯度},此时将$\v{e}$记作$\mathop{\mathrm{grad}}f(X_0)$或$\v{\nabla} f(X_0)$,也记作$\mathop{\overrightarrow{\mathrm{grad}}}f(X_0)$,$\mathop{\mathbf{grad}}f(X_0)$或$\nabla f(X_0)$。
}

\begin{theorem}
	若$f$在点$X_0$处可微,则$\mathrm{grad}f(X_0)=
	\begin{pmatrix}
		\dfrac{\partial f}{\partial x_1}(X_0)\\
		\vdots\\
		\dfrac{\partial f}{\partial x_n}(X_0)
	\end{pmatrix}$。
\end{theorem}
\begin{proofs}
	\adjline
	设右边为$\v{e}$,$\v{\beta}^0=(\cos \alpha_1,\cdots,\cos\alpha_n)\tp$,则
	\begin{equation*}
	\frac{\partial f}{\partial \v{\beta}^0}(X_0)=\sum\limits_{j=1}^{n}\frac{\partial f}{\partial x_j}(X_0)\cos \alpha_j
	=\v{e}\cdot \v{\beta}^0=\|\v{e}\|\cos\langle\v{e},\v{\beta}^0\rangle
	\le \|\v{e}\|
	\end{equation*}
	其中$\langle\v{e},\v{\beta}^0\rangle$表示$\v{e}$与$\v{\beta}^0$的夹角。
	由此得证。
\end{proofs}

\di[方向导数与梯度的关系]{
	\begin{equation}
		\frac{\partial f}{\partial \v{\beta}^0}(X_0)=\mathop{\mathrm{grad}}f(X_0)\cdot\v{\beta}^0
	\end{equation}
}

\subsubsection{高阶偏导数}

\de[高阶偏导数]{
	设$\varOmega\subset\mathbb{R}^n$为非空开集。
	若$f: \varOmega\rightarrow\mathbb{R}$	
	在$\varOmega$上可导,则我们可以在$\varOmega$上定义$n$个偏导	
	函数$\dfrac{\partial f}{\partial x_i}$($1\le i\le n$)。
	如果$\dfrac{\partial f}{\partial x_i}$关于第$j$个变量	
	有偏导数$\dfrac{\partial }{\partial x_j}\bigg(\dfrac{\partial f}{\partial x_i}\bigg)$,
	则称其为$f$对$x_i$再对$x_j$的\tboba{二阶偏导数},	
	记作$\mboba{\dfrac{\partial^2 f}{\partial x_j\partial x_i}}$或$\mboba{\partial_{ji}f}$。
	特别地,当$i=j$时,我们将之记作$\mboba{\dfrac{\partial^2 f}{\partial x_i^2}}$。

	如此递归下去,我们可定义三阶偏导数以及任意阶的偏导数。
}
\di[交换求导次序的条件]{
	设$\varOmega\subset\mathbb{R}^n$为非空开集。
	若$f: \varOmega\rightarrow\mathbb{R}$
	在$\varOmega$上有二阶偏导函数$\dfrac{\partial^2 f}{\partial x_j\partial x_i}$,
	$\dfrac{\partial^2 f}{\partial x_i\partial x_j}$($i\neq j$),
	并且它们当中的一个在点$X_0\in\varOmega$处连续,则
	\begin{equation}
	\frac{\partial^2 f}{\mbome{\partial  x_j}\boldsymbol{\color{meihong!50!black}\partial x_i}}(X_0)
	=\frac{\partial^2 f}{\boldsymbol{\color{meihong!50!black}\partial x_i }\mbome{\partial x_j}}(X_0)
	\end{equation}
}
\begin{proofs}
	\adjline
	出于简便,仅考虑$n=2$的情形。首先
	\begin{equation*}
	\frac{\partial f}{\partial x_2}(x_1,x_2)=\lim_{h_2\rightarrow 0}\frac{f(x_1,x_2+h_2)-f(x_1,x_2)}{h_2}
	\end{equation*}
	记$X_0=(a_1,a_2)$,于是我们有
	\begin{align*}
	\frac{\partial^2 f}{\partial x_1\partial x_2}(a_1,a_2)
	&=\lim_{h_1\rightarrow 0}\frac{\frac{\partial f}{\partial x_2}(a_1+h_1,a_2)-\frac{\partial f}{\partial x_2}(a_1,a_2)}{h_1}\\[-3pt]
	&=\lim_{h_1\rightarrow 0}\frac{1}{h_1}\bigg(\lim_{h_2\rightarrow 0}\frac{f(a_1+h_1,a_2+h_2)-f(a_1+h_1,a_2)}{h_2}\\[-3pt]&\quad-\lim_{h_2\rightarrow 0}\frac{f(a_1,a_2+h_2)-f(a_1,a_2)}{h_2}\bigg)\\[-3pt]
	&=\lim_{h_1\rightarrow 0}\lim_{h_2\rightarrow 0}\frac{1}{h_1h_2}\Big(\left(f(a_1+h_1,a_2+h_2)-f(a_1+h_1,a_2)\right)\\[-3pt]
	&\quad-\left(f(a_1,a_2+h_2)-f(a_1,a_2)\right)\Big)
	\end{align*}
	故等价于证明上述累次极限可交换次序。不失一般性,假设
	$\dfrac{\partial^2 f}{\partial x_2\partial x_1}$在
	点$X_0=(a_1,a_2)$处连续,并定义
	\begin{equation*}
	F(h_1,h_2)=\left(f(a_1+h_1,a_2+h_2)-f(a_1+h_1,a_2)\right)
	-\left(f(a_1,a_2+h_2)-f(a_1,a_2)\right)
	\end{equation*}
	则由偏导数的定义可知
	\begin{equation*}
	\frac{\partial^2 f}{\partial x_1\partial x_2}(a_1,a_2)=\lim_{h_1\rightarrow 0}\lim_{h_2\rightarrow 0}\frac{1}{h_1h_2}F(h_1,h_2),\qquad
	\frac{\partial^2 f}{\partial x_2\partial x_1}(a_1,a_2)=\lim_{h_2\rightarrow 0}\lim_{h_1\rightarrow 0}\frac{1}{h_1h_2}F(h_1,h_2)
	\end{equation*}
	令$\varphi(x_1)=f(x_1,a_2+h_2)-f(x_1,a_2)$。则我们有
	\begin{equation*}
	F(h_1,h_2)=\varphi(a_1+h_1)-\varphi(a_1)
	\end{equation*}
	因$\dfrac{\partial f}{\partial x_1}$存在,则$\varphi$可导,
	从而由Lagrange中值定理可知,$\exists \theta_1\in (0,1)$使得
	\begin{equation*}
	F(h_1,h_2)=\varphi'(a_1+\theta_1h_1)h_1
	=\bigg(\frac{\partial f}{\partial x_1}(a_1+\theta_1h_1,a_2+h_2)
	-\frac{\partial f}{\partial x_1}(a_1+\theta_1h_1,a_2)\bigg)h_1
	\end{equation*}
	由于$\dfrac{\partial^2 f}{\partial x_2\partial x_1}$存在,于是在上式中对第二个变量
	应用Lagrange中值定理可知,$\exists \theta_2\in (0,1)$使得 
	\begin{align*}
	F(h_1,h_2)&=\bigg(\frac{\partial f}{\partial x_1}(a_1+\theta_1h_1,a_2+h_2)-\frac{\partial f}{\partial x_1}(a_1+\theta_1h_1,a_2)\bigg)h_1\\[-3pt]
	&=\frac{\partial }{\partial x_2}\bigg(\frac{\partial f}{\partial x_1}\bigg)(a_1+\theta_1h_1,a_2+\theta_2h_2)h_1h_2
	\end{align*}
	由夹逼原理,连续性以及复合极限法则可知
	\begin{equation*}
	\lim_{(h_1,h_2)\rightarrow (0,0)}\frac{F(h_1,h_2)}{h_1h_2}
	=\lim_{(h_1,h_2)\rightarrow (0,0)}
	\frac{\partial^2 f}{\partial x_2\partial x_1}(a_1+\theta_1h_1,a_2+\theta_2h_2)
	=\frac{\partial^2 f}{\partial x_2\partial x_1}(a_1,a_2)
	\end{equation*}
	也即二重极限存在。故所证结论成立。
\end{proofs}

\begin{definition}
	设$\varOmega\subset\mathbb{R}^n$为开集,$k\ge 0$为整数,约定记号
	$\boldsymbol{\mathscr{C}\hspace{-0.82em}\mathscr{C}\hspace{-0.82em}\mathscr{C}^{(k)}(\varOmega)}=\{f \, |\, f\, \text{在}\, \varOmega\, \text{上的$k$阶偏导数连续} \}$。
	特别地,$\mathscr{C}^{(0)}(\varOmega)=\mathscr{C}(\varOmega)$。若$f\in \mathscr{C}^{(k)}(\varOmega)$,则称$f$在$\varOmega$上$\boldsymbol{k}$\ \textbf{阶连续可导},也称$\boldsymbol{k}$\ \textbf{阶连续可微}。
\end{definition}
\begin{theorem}
	设$k\ge 2$为整数。若$f\in \mathcal{C}^{(k)}(\varOmega)$,则对
	任意整数$1\le r\le k$,均有$f\in \mathcal{C}^{(r)}(\varOmega)$且$f$的
	任意一个$r$阶偏导数均与求偏导的次序无关。
\end{theorem}
\subsubsection{复合函数的偏导数}
\di[复合函数求偏导的链式法则]{
	如果 $ u=u(x,y),v=v(x,y)$  在点$(x,y)\in D$ 上具有一阶偏导数,而函数$z=f(u,v)$ 在点  $(u,v)=(u(x,y),v(x,y))$ 可微分,则复合函数
	$ z=F(x,y)=f(u(x,y),v(x,y))$在点$(x,y)$存在偏导数,且
	\begin{equation}
		\frac{\partial z}{ \mbome{\partial x}}= \frac{\partial z}{ \boldsymbol{\color{meihong!50!black}\partial u}}\cdot  \frac{\boldsymbol{\color{meihong!50!black}\partial u}}{\mbome{ \partial x}}+ \frac{\partial z}{ \boldsymbol{\color{meihong!50!black}\partial v}}\cdot  \frac{\boldsymbol{\color{meihong!50!black}\partial v}}{\mbome{ \partial x}},\qquad
		\frac{\partial z}{ \mbome{\partial y}}= \frac{\partial z}{ \boldsymbol{\color{meihong!50!black}\partial u}}\cdot  \frac{\boldsymbol{\color{meihong!50!black}\partial u}}{\mbome{ \partial y}}+ \frac{\partial z}{ \boldsymbol{\color{meihong!50!black}\partial v}}\cdot  \frac{\boldsymbol{\color{meihong!50!black}\partial v}}{\mbome{ \partial y}}
	\end{equation}
}
\begin{proofs}
	\adjline
	给$x$以增量 $\Delta x$,则变量$u,v$ 有对应的增量
	$$\Delta_x u= u(x+\Delta x,y)-u(x,y),\qquad\Delta_x v= v(x+\Delta x,y)-v(x,y)$$
	由于 $z=f(u,v)$ 在点$(u,v)$处是可微的,故
	$$ \Delta_x z:= \frac{\partial z}{ \partial u}\Delta_x u +\frac{\partial z}{ \partial v} \Delta_x v+o(\sqrt{\Delta_x u^2+\Delta_x v^2})$$
	于是
	$$ \frac{\Delta_x z}{\Delta x}=\frac{\partial z}{\partial u}\cdot \frac{\Delta_x u}{\Delta x}+\frac{\partial z}{\partial v}\cdot \frac{\Delta_x v}{\Delta x}+\frac{o(\sqrt{\Delta_x u^2+\Delta_x v^2})}{\Delta x}$$
	显然,当$\Delta x\to 0$ 时
	$$\Delta_x u=u(x+\Delta x,y)-u(x,y)\to 0,\qquad\Delta_x v=v(x+\Delta x,y)-v(x,y)\to 0$$ 
	又因为 $ \dfrac{\partial u}{{\partial x}},\dfrac{\partial v}{{\partial x}}$ 存在,那么当 $\Delta x\to 0$ 时,有
	$$\frac{\Delta_x u}{\Delta x}=\frac{ u(x+\Delta x,y)-u(x,y)}{\Delta x}\to  \frac{\partial u}{{\partial x}},\qquad\frac{\Delta_x v}{\Delta x}=\frac{ v(x+\Delta x,y)-v(x,y)}{\Delta x}\to  \frac{\partial v}{{\partial x}}$$	
	记$\sqrt{\Delta_x u^2+\Delta_x v^2}:=\rho$,则有$\rho=\sqrt{\Delta_x u^2+\Delta_x v^2}\to 0$,而$\dfrac{o(\rho)}{\rho}=\dfrac{o\bigl(\sqrt{\Delta_x u^2+\Delta_x v^2}\bigr)}{\sqrt{\Delta_x u^2+\Delta_x v^2}}\to 0$。于是,
	当 $\Delta x\to 0$ 时,如果 $\rho=0$,则  $ \dfrac{o(\rho)}{\Delta x}=0$ ;
	如果 $\rho\not=0$,则 
	$$ \frac{o(\rho)}{\Delta x}= \frac{o(\rho)}{\rho} \cdot \frac{\rho}{\Delta x}=\frac{o(\rho)}{\rho}\cdot \sqrt{ \left(\frac{\Delta_x u}{\Delta x}\right)^2+ \left(\frac{\Delta_x v}{\Delta x}\right)^2  }\cdot \frac{|\Delta x|}{\Delta x}\to 0$$
	代回到上面式子中,令$\Delta x\to 0$,即得$\dfrac{\partial z}{ \partial x}= \dfrac{\partial z}{\partial u}\cdot  \dfrac{\partial u}{ \partial x}+ \dfrac{\partial z}{ \partial v}\cdot  \dfrac{\partial v}{\partial x}$。对$y$同理可证。
\end{proofs}
\zhu[单中间变量的复合函数的偏导数]{
	对$z=f(x,v(x,y))$,由上即有$$\dfrac{\partial z}{\partial x}=\dfrac{\partial f}{\partial x}\dfrac{\partial x}{\partial x}+\dfrac{\partial f}{\partial v}\dfrac{\partial v}{\partial x}=\dfrac{\partial f}{\partial x}+\dfrac{\partial f}{\partial v}\dfrac{\partial v}{\partial x}$$
	这里的$\dfrac{\partial z}{\partial x}$和$\dfrac{\partial f}{\partial x}$并不是固定的写法,但这两处偏导数的含义不同。
}

\subsubsection{隐函数的导数}
\din[$\R^2$中的隐函数定理]{
	给定一点$X_0=(x_0,y_0)\in\mathbb{R}^2$,$r>0$,数量值
	函数$F:B(X_0,r)\rightarrow \mathbb{R}$满足$f \in \mathscr{C}^{(1)}\left(B(X_0,r)\right)$,$F(x_0,y_0)=0$,$\dfrac{\partial F}{\partial y}(x_0,y_0)\neq 0$,则
	
	\hang[2](1)$\exists \delta,\eta>0$满足$B(x_0,\delta)\times B(y_0,\eta)\subset B(X_0,r)$({\kai$\times$表示笛卡尔积}),使得$\forall x\in B(x_0,\delta)$,$\exists ! y\in B(y_0,\eta)$使得$F(x,y)=0$,即可以定义一个新映射$f:B(x_0,\delta)\rightarrow B(y_0,\eta)$使$f(x)=y$;

	(2)$f$为$\mathscr{C}^{(1)}$类函数,且
	$\forall x\in B(x_0,\delta)$,
	均有
	\begin{equation}
		f'(x)=-\cfrac{\dfrac{\partial F}{\partial x}(x,f(x))}{\dfrac{\partial F}{\partial y}(x,f(x))}
	\end{equation}
}
\begin{proofs}
	\adjline
	不失一般性,我们可假设$\dfrac{\partial F}{\partial y}(x_0,y_0)> 0$。
	否则考虑函数$-F$。
	
	(1)\textbf{存在性}。  由题设可知$\dfrac{\partial F}{\partial y}$连续,则$\exists \eta>0$ 使得
	$\forall (x,y)\in B(X_0,\sqrt{2}\eta)\subsetneqq B(X_0,r)$,$\dfrac{\partial F}{\partial y}(x,y)>0$。
	$\forall (x,y)\in B(X_0,\sqrt{2}\eta)$,我们令$g_x(y)=F(x,y)$。
	则对于每个固定的$x\in [x_0-\eta,x_0+\eta]$,函数$g_x$
	在$[y_0-\eta,y_0+\eta]$上可导 且$g_{x_0}'(y)=\dfrac{\partial F}{\partial y}(x_0,y)>0$,
	从而$g_{x_0}$为严格递增函数。又$g_{x_0}(y_0)=0$,故
	\begin{equation*}
	F(x_0,y_0-\eta)=g_{x_0}(y_0-\eta)<g_{x_0}(y_0)=0
	<g_{x_0}(y_0+\eta)=F(x_0,y_0+\eta)
	\end{equation*}
	注意到$F$连续,于是由连续函数的保号性知,
	$\exists \delta\in (0,\eta)$ 使得$\forall x\in (x_0-\delta,x_0+ \delta)$,均有
	\begin{equation*}
	g_x(y_0-\eta)=F(x,y_0-\eta)<0,\qquad
	g_x(y_0+\eta)=F(x,y_0+\eta)>0
	\end{equation*}
	又$\forall y\in[y_0-\eta,y_0+\eta]$,
	均有$g_x'(y)=\dfrac{\partial F}{\partial y}(x,y)>0$,
	因此$g_x$在$[y_0-\eta,y_0+\eta]$上严格递增且连续,
	由连续函数介值定理,$\exists ! y\in(y_0-\eta,y_0+\eta)$使得
	$F(x,y)=g_x(y)=0$。令$f(x)=y$。则$f$为所求。
	
	(2)\textbf{连续性}。  由前面讨论知,$\forall \varepsilon\in (0,\eta)$,$\exists \delta'\in (0,\varepsilon)$
	使$\forall x\in B(x_0,\delta')$,$\exists ! y\in B(y_0,\varepsilon)$使$F(x,y)=0$,此时$y=f(x)$,也即当$|x-x_0|<\delta'$时,我们有
	$|f(x)-f(x_0)|<\varepsilon$。故函数$f$在点$x_0$处连续。

	  取$x_1\in B(x_0,\delta)$,$y_1=f(x_1)$,则$F(x_1,y_1)=0$且$(x_1,y_1)\in B((x_0,y_0),\eta)$,因此$\dfrac{\partial F}{\partial y}(x_1,y_1)>0$。由前面的讨论可知,存在$\delta_1\in(0,\delta),\eta_1\in (0,\eta)$以及在$x_1$连续的函数$g: B(x_1,\delta_1)\rightarrow B(y_1,\eta_1)$使$F(x,g(x))=0$。另外可设$B(x_1,\delta_1)\subset B(x_0,\delta)$,由唯一性知$\forall x\in B(x_1,\delta_1)$,均有$f(x)=g(x)$,
	故$f$在点$x_1$处连续。
	
	(3)\textbf{可导性}。$\forall x\in B(x_0,\delta)$,$\exists h\in\mathbb{R}$,使$x+h\in B(x_0,\delta)$。
	令$y=f(x)$,$\Delta y=f(x+h)-f(x)$。由Lagrange
	中值定理可知,$\exists \theta_1,\theta_2\in (0,1)$ 使得
	\begin{align*}
		0&=F(x+h,y+\Delta y)-F(x,y)\\[-7pt]
		&=\left(F(x+h,y+\Delta y)-\mbome{F(x,y+\Delta y)}\right)+\left(\mbome{F(x,y+\Delta y)}-F(x,y)\right)\\[-3pt]
		&=\frac{\partial F}{\partial x}(x+\theta_1h,y+\Delta y)\mbome{h}+\frac{\partial F}{\partial y}(x,y+\theta_2\Delta y)\mbome{\Delta y}
	\end{align*}
	由于$\dfrac{\partial F}{\partial x},\dfrac{\partial F}{\partial y}$均连续,于是由夹逼原理以及复合函数极限法则可知
	\begin{equation*}
		f'(x)=\lim_{h\rightarrow 0}\frac{f(x+h)-f(x)}{h}=\mbome{\lim_{h\rightarrow 0}\frac{\Delta y}{h}}
		=-\lim_{h\rightarrow 0}\frac{\frac{\partial F}{\partial x}(x+\theta_1h,y+\Delta y)}{\frac{\partial F}{\partial y}(x,y+\theta_2\Delta y)}
		=-\frac{\frac{\partial F}{\partial x}(x,y)}{\frac{\partial F}{\partial y}(x,y)}=-\frac{\frac{\partial F}{\partial x}(x,f(x))}{\frac{\partial F}{\partial y}(x,f(x))}
	\end{equation*}
	上式同时表明$f'$为连续函数,故$f$连续可导。
\end{proofs}
\begin{theorem}
	\textbf{\textup{$\R^{n+1}$中的隐函数定理\quad}}给定$X_0\in\mathbb{R}^n$,$y_0\in\mathbb{R}$,$r>0$,数量值函数$F:B\left((X_0,y_0),r\right)\rightarrow\mathbb{R}$。若$F$满足 $F \in \mathscr{C}^{(1)}\Bigl(\left((X_0,y_0),r\right)\Bigr)$使$F(X_0,y_0)=0$,
	$\dfrac{\partial F}{\partial y}(X_0,y_0)\neq 0$,则
	
	(1)$\exists \delta,\eta>0$满足$B(X_0,\delta)\times B(y_0,\eta)\subset B\left((X_0,y_0),r\right)$,使得$\forall X\in B(X_0,\delta)$,
	$\exists ! y\in B(y_0,\eta)$使$F(X,y)=0$,即可以定义一个新映射$f:B(X_0,\delta)\rightarrow B(y_0,\eta)$,使得$f(X)=y$;

	(2)$f \in \mathscr{C}^{(1)}\bigl(B(X_0,\delta)\bigr)$,且 $\forall X\in B(X_0,\delta)$与任意整数$1\le i\le n$,
	$\dfrac{\partial f}{\partial x_i}(X)=-\dfrac{\frac{\partial F}{\partial x_i}(X,f(X))}{\frac{\partial F}{\partial y}(X,f(X))}$。
\end{theorem}

\zhu[隐函数求导]{
	上述最后一个等式$\dfrac{\partial f}{\partial x_i}(X)=-\dfrac{\frac{\partial F}{\partial x_i}(X,f(X))}{\frac{\partial F}{\partial y}(X,f(X))}$可由对恒等式
	$F(x_1,\cdots,x_n,f(x_1,\cdots,x_n))=0$
	求偏导数而得。事实上,对$x_i$求偏导数可得
	$$\frac{\partial F}{\partial x_i}(X,f(X))+\frac{\partial F}{\partial y}(X,f(X))\frac{\partial f}{\partial x_i}(X)=0$$
	由此我们可立刻导出
	\begin{equation*}
	\frac{\partial f}{\partial x_i}(X)=-\frac{\frac{\partial F}{\partial x_i}(X,f(X))}{\frac{\partial F}{\partial y}(X,f(X))}
	\end{equation*}	
}
\begin{example}
	设$F$为$\mathscr{C}^{(2)}$类,
	则由方程$F(x,y,z)=0$确定的隐函数$z=f(x,y)$为$\mathscr{C}^{(2)}$类,
	求$\dfrac{\partial^2z}{\partial y\partial x}$。
\end{example}
\begin{proofs}
	令$u=\dfrac{\partial F}{\partial z}(x,y,z(x,y))\neq 0$。由题设可得
	\begin{align*}
		\frac{\partial^2z}{\partial y\partial x}
		&=\frac{\partial}{\partial y}\left(-\frac{\mbome{\dfrac{\partial F}{\partial x}(x,y,z(x,y))}}{\boldsymbol{\color{meihong!50!black}\dfrac{\partial F}{\partial z}(x,y,z(x,y))}}\right)\\
		&=\mathop{-}\frac{1}{u^2}\bigg[\frac{\partial}{\partial y}\bigg(\mbome{\frac{\partial F}{\partial x}(x,y,z(x,y))}\bigg)
		\boldsymbol{\color{meihong!50!black}\frac{\partial F}{\partial z}(x,y,z(x,y))}\\
		&\quad\mathop{}-\mbome{\frac{\partial F}{\partial x}(x,y,z(x,y))}\frac{\partial}{\partial y}\bigg(\boldsymbol{\color{meihong!50!black}\frac{\partial F}{\partial z}(x,y,z(x,y))}\bigg)\bigg]\\
		&=-\frac{1}{u^2}
		\bigg[\bigg(\mbome{\frac{\partial^2 F}{\partial y\partial x}+\frac{\partial^2 F}{\partial z\partial x}\frac{\partial z}{\partial y}}\bigg)
		\frac{\partial F}{\partial z}-\frac{\partial F}{\partial x}\bigg(\boldsymbol{\color{meihong!50!black}\frac{\partial^2 F}{\partial y\partial z}+\frac{\partial^2 F}{\partial z^2}\frac{\partial z}{\partial y}}\bigg)\bigg]\\
		&=-\frac{1}{u^2}
		\bigg[\bigg[\frac{\partial^2 F}{\partial y\partial x}+\frac{\partial^2 F}{\partial z\partial x}\bigg(\mbome{-\frac{\frac{\partial F}{\partial y}}{u}}\bigg)\bigg]
		\frac{\partial F}{\partial z}-\frac{\partial F}{\partial x}\bigg[\frac{\partial^2 F}{\partial y\partial z}+\frac{\partial^2 F}{\partial z^2}\bigg(\boldsymbol{\color{meihong!50!black}-\frac{\frac{\partial F}{\partial y}}{u}}\bigg)\bigg]\bigg]\\
		&=-\frac{1}{u^3}
		\bigg[\bigg(\mbome{\frac{\partial F}{\partial z}}\bigg)^2\frac{\partial^2 F}{\partial y\partial x}
		-\frac{\partial F}{\partial y}\frac{\partial F}{\partial z}\frac{\partial^2 F}{\partial z\partial x}
		-\frac{\partial F}{\partial x}\frac{\partial F}{\partial z}\frac{\partial^2 F}{\partial y\partial z}
		+\frac{\partial F}{\partial x}\frac{\partial F}{\partial y}\frac{\partial^2 F}{\partial z^2}\bigg]\\
		&=-\frac{\bigg(\dfrac{\partial F}{\partial z}\bigg)^2\dfrac{\partial^2 F}{\partial y\partial x}
		-\dfrac{\partial F}{\partial y}\dfrac{\partial F}{\partial z}\dfrac{\partial^2 F}{\partial z\partial x}
		-\dfrac{\partial F}{\partial x}\dfrac{\partial F}{\partial z}\dfrac{\partial^2 F}{\partial y\partial z}
		+\dfrac{\partial F}{\partial x}\dfrac{\partial F}{\partial y}\dfrac{\partial^2 F}{\partial z^2}}{\bigg(\dfrac{\partial F}{\partial z}\bigg)^3} \qedhere
	\end{align*}
\end{proofs}

\subsubsection{反函数的导数}

我们考虑,向量值函数$\v{g}: \varOmega\subset \mathbb{R}^n\rightarrow\mathbb{R}^n$是否能有
反函数$\v{g}^{-1}$,这等价于问方程$X=\v{g}(Y)$是否有解$Y=\v{g}^{-1}(X)$,也即问方程
$$F(X,Y):=\v{g}(Y)-X=\v{0}$$
是否有隐函数解$Y=\v{g}^{-1}(X)$?

\di[反函数定理]{
	设$k\ge 1$为整数,$\varOmega\subset \mathbb{R}^n$为非空开集,
	$Y_0\in\varOmega$,$\v{g}:\varOmega\rightarrow \mathbb{R}^n$为$\mathscr{C}^{(k)}$类
	使得$J_{\v{g}}(Y_0)$可逆。
	令$X_0=\v{g}(Y_0)$,则$\exists\delta,\eta>0$满足$B(Y_0,\eta)\subset \varOmega$,使得
	\begin{center}
	存在$\v{f}: B(X_0,\delta)\rightarrow B(Y_0,\eta)$为$\mathscr{C}^{(k)}$类,使得$\forall X\in B(X_0,\delta)$,$\forall Y\in B(Y_0,\eta)$,\\等式$X=\v{g}(Y)$
	成立当且仅当$Y=\v{f}(X)$
	\end{center}
	即$\v{g}$在点$Y_0$处具有局部的反函数(局部可逆)。此时,$\forall X\in B(X_0,\delta)$,
	$J_{\v{f}}(X)=\left(J_{\v{g}}(\v{f}(X)\right)^{-1}=\left(J_{\v{g}}(Y)\right)^{-1}$。
}

\begin{example}
	\textbf{极坐标变换\quad}令$D=(0,+\infty)\times (-\pi,\pi)$。
	$\forall (\rho,\varphi)\in D$,定义
	\begin{equation*}
	\v{f}(\rho,\varphi)=\begin{pmatrix}
		x\\[-5pt]
		y
	\end{pmatrix}=
	\begin{pmatrix}
		\rho\cos \varphi\\[-5pt]
		\rho\sin \varphi
	\end{pmatrix}
	\end{equation*}
	则$\v{f}$为$\mathscr{C}^{(\infty)}$类向量值函数 且
	\begin{equation*}
	J_{\v{f}}(\rho,\varphi)=\begin{pmatrix}
		\cos \varphi &-\rho\sin \varphi\\[-5pt]
		\sin \varphi &\rho\cos \varphi
	\end{pmatrix}
	\end{equation*}
	从而Jacobi行列式$\det J_{\v{f}}(\rho,\varphi)=\rho>0$。于是
	$\v{f}$为局部可逆,其逆映射$\v{f}^{-1}$也为$\mathscr{C}^{(\infty)}$类 且
	\begin{equation*}
	J_{\v{f}^{-1}}(x,y)=\begin{pmatrix}
		\cos \varphi &-\rho\sin \varphi\\[-5pt]
		\sin \varphi &\rho\cos \varphi
	\end{pmatrix}^{-1}
	= \frac{1}{\rho}\begin{pmatrix}
		\rho\cos \varphi &\rho\sin \varphi\\[-5pt]
		-\sin \varphi &\cos \varphi
	\end{pmatrix}
	\end{equation*}
\end{example}

\begin{example}
	设隐函数$u=u(x,y)$由方程组
	$\begin{cases}
		u=f(x,y,z,t),\\
		g(y,z,t)=0,\\
		h(z,t)=0,
	\end{cases}$
	确定,求$\dfrac{\partial u}{\partial x},\dfrac{\partial u}{\partial y}$。
\end{example}
\begin{solution}
	由题设可知,利用方程组$\begin{cases}
		g(y,z,t)=0,\\
		h(z,t)=0
	\end{cases}$,可将$z,t$确定为$y$的函数,由此可得
	$\dfrac{\partial u}{\partial x}=\dfrac{\partial f}{\partial x}(x,y,z,t)$。

	由隐函数定理可知 
	\begin{equation*}
		\begin{pmatrix}
			\dfrac{\dif z}{\dif y}\\
			\dfrac{\dif t}{\dif y}
		\end{pmatrix}=-\begin{pmatrix}
			\dfrac{\partial g}{\partial z}&\dfrac{\partial g}{\partial t}\\
			\dfrac{\partial h}{\partial z}&\dfrac{\partial h}{\partial t}
		\end{pmatrix}^{-1}\begin{pmatrix}
			\dfrac{\partial g}{\partial y}\\
			\dfrac{\partial h}{\partial y}
		\end{pmatrix}=-\left|\dfrac{\partial (g,h)}{\partial (z,t)}\right|^{-1}
		\begin{pmatrix}
			\dfrac{\partial h}{\partial t}&-\dfrac{\partial g}{\partial t}\\
			-\dfrac{\partial h}{\partial z}&\dfrac{\partial g}{\partial t}
		\end{pmatrix}
		\begin{pmatrix}
			\dfrac{\partial g}{\partial y}\\
			0
		\end{pmatrix}
	\end{equation*}
	于是$\dfrac{\dif z}{\dif y}=-\dfrac{\dfrac{\partial h}{\partial t}\cdot\dfrac{\partial g}{\partial y}}{\Bigg|\dfrac{\partial (g,h)}{\partial (z,t)}\Bigg|}$,
	$\dfrac{\dif t}{\dif y}=\dfrac{\dfrac{\partial h}{\partial z}\cdot\dfrac{\partial g}{\partial y}}{\Bigg|\dfrac{\partial (g,h)}{\partial (z,t)}\Bigg|}$,
	进而可得
	\begin{equation*}
		\frac{\partial u}{\partial y}=\frac{\partial f}{\partial y}
		+\frac{\partial f}{\partial z}\cdot\frac{\dif z}{\dif y}
		+\frac{\partial f}{\partial t}\cdot\frac{\dif t}{\dif y}
		=\frac{\partial f}{\partial y}+\frac{\Bigg|\dfrac{\partial (h,f)}{\partial (z,t)}\Bigg|}{\Bigg|\dfrac{\partial (g,h)}{\partial (z,t)}\Bigg|}\cdot\frac{\partial g}{\partial y} \qedhere
	\end{equation*}
\end{solution}

\subsection{多元函数的Taylor展式}

\begin{definition}
	我们称函数$F:\mathbb{R}^n\times\mathbb{R}^n\rightarrow \mathbb{R}$为\textbf{双线性型},如果$\forall X_0,Y_0\in\mathbb{R}^n$,函数$Y \mapsto F(X_0,Y)$和函数 $X \mapsto F(X,Y_0)$均为$\mathbb{R}^n$上的线性函数。
\end{definition}

对双线性函数$F(X,Y)$,记$X=(x_1,\cdots,x_n)\tp=\sum_{i=1}^nx_i\mathbf{e}_i$,$Y=(y_1,\cdots,y_n)\tp=\sum_{j=1}^ny_j\mathbf{e}_j$,则
\begin{equation*}
F(X,Y)=F\left(\sum_{i=1}^nx_i\mathbf{e}_i,Y\right)=\sum_{i=1}^nx_i F(\mathbf{e}_i,Y)
=\sum_{i=1}^nx_i F\left(\mathbf{e}_i,\sum_{j=1}^ny_j\mathbf{e}_j\right)
=\sum_{i=1}^n\sum_{j=1}^n x_i y_j F(\mathbf{e}_i,\mathbf{e}_j)
\end{equation*}
令$a_{ij}=F(\mathbf{e}_i,\mathbf{e}_j)$,$A=(a_{ij})_{1\le i,j\le n}$,则
\begin{equation*}
	F(X,Y)=\sum_{i=1}^n\sum_{j=1}^n x_i y_j F(\mathbf{e}_i,\mathbf{e}_j)
	=\sum_{i=1}^nx_i \sum_{j=1}^na_{ij}y_j
	=\sum_{i=1}^nx_i(AY)_i
	=X\tp AY
\end{equation*}
即双线性函数可以用矩阵表示。

\di[多元函数的带Lagrange余项的一阶Taylor展式]{
	设$X_0\in\mathbb{R}^n$,$r>0$,实值函数 $f\in\mathscr{C}^{(2)}(B(X_0,r))$,则$\forall X\in B(X_0,r)$,$\exists \theta\in (0,1)$使得
	\begin{align*}
		f(X)&=f(X_0)+\sum_{j=1}^n\frac{\partial f}{\partial x_j}(X_0)(x_j-x_j^{(0)})+\frac{1}{2!}\sum_{i=1}^n\sum_{j=1}^n\frac{\partial^2f}{\partial x_i\partial x_j}(X_{\theta})(x_i-x_i^{(0)})(x_j-x_j^{(0)})\\[-7pt]
		&=f(X_0)+\mbome{J_{f}(X_0)}\Delta X+\frac{1}{2!}(\Delta X)\tp \mbome{H_f(X_{\theta})}\Delta X
	\end{align*}
	其中$\Delta X=X-X_0$,$X_{\theta}=X_0+\theta (X-X_0)$,$J_{f}(X_0) = \bigg(\dfrac{\partial f}{\partial x_1}(X_0),\dfrac{\partial f}{\partial x_2}(X_0),\cdots,\dfrac{\partial f}{\partial x_n}(X_0)\bigg)$是$f$在点$X_0$的的 Jacobi矩阵,
	$H_f(X_{\theta}) = \bigg(\dfrac{\partial^2f}{\partial x_i\partial x_j}(X_{\theta})\bigg)_{1\le i,j\le n}$为$f$在点$X_{\theta}$的~\tbome{Hessian矩阵}。
}
由这一定理,当$X\rightarrow X_0$时,$H_f(X_\theta)=H_f\left(X_0+\theta(X-X_0)\right)=H_f(X_0)+\v{o}(1)$,则得
\di[多元函数的带Peano余项的二阶Taylor展式]{
	设$X_0\in\mathbb{R}^n$,$r>0$,实值函数 $f\in\mathscr{C}^{(2)}(B(X_0,r))$,则$\forall X\in B(X_0,r)$,当$X\rightarrow X_0$时,
	$$
	f(X)=f(X_0)+J_{f}(X_0)\Delta X+\frac{1}{2!}(\Delta X)\tp H_f(X_0)\Delta X+o(\|\Delta X\|^2)
	$$
}

\subsection{多元函数的极值与条件极值}

\subsubsection{极值与 Hessian矩阵的正定性}

\de[极值]{
	设$\varOmega\subseteq\mathbb{R}^n$,$X_0\in \varOmega$,而$f: \varOmega\rightarrow \mathbb{R}$。
	\begin{itemize}[leftmargin=1em]
	\item 如果$\exists r>0$使得$\forall X\in B(X_0,r)\subseteq \varOmega$,均有
	$f(X)\ge f(X_0)$,那么称点$X_0$为$f$的(局部)
	\tboba{极小值点},而称$f(X_0)$为(局部)\tboba{极小值}。
	
	\item 如果$\exists r>0$使得$\forall X\in B(X_0,r)\subseteq \varOmega$,均有
	$f(X)\le f(X_0)$,那么称点$X_0$为$f$的(局部)
	\tboba{极大值点},而称$f(X_0)$为(局部)\tboba{极大值}。
	\end{itemize}
	极小值点和极大值点合称\tboba{极值点}。
}
\begin{definition}
	设$\varOmega\subseteq\mathbb{R}^n$,$X_0\in \varOmega$,而$f: \varOmega\rightarrow \mathbb{R}$。
	\begin{itemize}
		\item 若$\forall X\in \varOmega$,均有$f(X)\ge f(X_0)$,则称点$X_0$
		为$f$的\textbf{最小值点},而称$f(X_0)$为\textbf{最小值}。
		
		\item 若$\forall X\in \varOmega$,均有$f(X)\le f(X_0)$,则称点$X_0$
		为$f$的\textbf{最大值点},而称$f(X_0)$为\textbf{最大值}。
	\end{itemize}
	最小值点和最大值点合称\textbf{最值点}。
\end{definition}

极值点不一定是最值点,而最值点也不一定是极值点;但若最值点为内点,则它为极值点。

\di[Fermat定理的多元函数推广]{
	假设$\varOmega\subseteq\mathbb{R}^n$,$X_0$为$\varOmega$的内点,而函数
	$f: \varOmega\rightarrow \mathbb{R}$在该点可导。若$X_0$为$f$的极值点,
	则$J_f(X_0)=\v{0}$,即$\dfrac{\partial f}{\partial x_j}(X_0)=0$($1\le j\le n$)。
}
\begin{proofs}
	\adjline
	由于$X_0$为$\varOmega$的内点,于是$\exists r>0$使得
	$B(X_0,r)\subset \varOmega$。对任意整数$1\le j\le n$,设$\v{e}_j$为
	沿第$j$个坐标轴正向的单位向量。$\forall t\in (-r,r)$,
	定义$F(t)=f(X_0+t\v{e}_j)$。则由题设可知函数$F$
	在点$t=0$处可导,并且$F$还在该点处取极值,
	从而由Fermat定理可知$0=F'(0)=\dfrac{\partial f}{\partial x_j}(X_0)$。实际上,有
	\begin{equation*}
		F'(0)=\lim_{t\to 0}\frac{F(t)-F(0)}{t}
		=\lim_{t\to 0^+}\frac{f(X_0+t\v{e}_j)-f(X_0)}{t}=\frac{\partial f}{\partial \v{e}_j}\Bigg|_{X_0}
		= {\rm grad} f\Big|_{X_0}\cdot \v{e}_j= \frac{\partial f}{\partial x_j}(X_0) \qedhere
	\end{equation*}
\end{proofs}

\di[Role中值定理的多元函数推广]{
	设$\varOmega\subseteq \mathbb{R}^n$为有界开区域,
	而$f\in\mathscr{C}(\overline{\varOmega})$
	在$\varOmega$内可微并且在$\partial \varOmega$上取常值,
	则$f$在$\varOmega$内必有驻点,也即$\exists \xi\in \varOmega$使得$J_f(\xi)=\v{0}$。
}
\begin{proofs}
	如果$f$在$\overline{\varOmega}$上为常值函数,
	则$f$在$\varOmega$内
	也会为常值函数,从而$\forall\xi\in \varOmega$,均有$J_f(\xi)=\v{0}$。
	现在假设$f$不为常值函数。由于$\overline{\varOmega}$为有界闭集
	且$f\in\mathcal{C}(\overline{\varOmega})$,
	则$f$在$\overline{\varOmega}$上有最大值和最小值,但$f$在$\partial \varOmega$上取常值,
	故必有最值点$\xi\in\varOmega$,该点也为$f$的极值点,则$J_f(\xi)=\v{0}$。
\end{proofs}

我们在线性代数中已知:
\kai
\begin{itemize}
	\item 
	设$A=(a_{ij})_{1\le i,j\le n}$为实对称矩阵,特征根为$\lambda_1\le \lambda_2\le \cdots \le \lambda_n$,则存在正交矩阵$B$使得
	\begin{equation*}
		A=B\tp\begin{pmatrix}
			\lambda_1 &0          & \cdots & 0\\
			0         &\lambda_2  &\cdots  &0 \\
			\vdots    &\vdots     &\ddots  &0\\
			0         &0          &\cdots  &\lambda_n
		\end{pmatrix}B
	\end{equation*}
	$\forall X\in \mathbb{R}^n$,记$BX=(y_1,\cdots,y_n)\tp=Y$,则
	\begin{equation*}
		X\tp AX=\sum_{j=1}^n\lambda_j y_j^2\ge \lambda_1 \|Y\|^2=\lambda_1 Y\tp Y
		=\lambda_1X\tp B\tp BX=\lambda_1 X\tp X=\lambda_1\|X\|^2
	\end{equation*}
	同理可以证明$X\tp AX\le \lambda_n\|X\|^2$。

	\item
	对$n$阶实矩阵$A$、任意非零向量$\boldsymbol{x}\in \R^n$,如果
	\begin{itemize}[leftmargin=1em, labelsep=0.25em, itemindent=0em, itemsep=0pt, parsep=0pt, topsep=0pt, partopsep=0pt]
		\item 恒有$\boldsymbol{x}^\mathrm{T}A\boldsymbol{x}>0$,称$A$~\textbf{正定},这等价于说$\lambda_1 >0$;

		\item 恒有$\boldsymbol{x}^\mathrm{T}A\boldsymbol{x} \ge 0$,称矩阵$A$~\textbf{半正定},这等价于说$\lambda_n <0$;

		\item 恒有$\boldsymbol{x}^\mathrm{T}A\boldsymbol{x} < 0$,称矩阵$A$~\textbf{负定},这等价于说$\lambda_1 =0$;

		\item 恒有$\boldsymbol{x}^\mathrm{T}A\boldsymbol{x} \le 0$,称矩阵$A$~\textbf{半负定},这等价于说$\lambda_n =0$;

		\item $\boldsymbol{x}^\mathrm{T}A\boldsymbol{x}$符号不定,称矩阵$A$~\textbf{不定},这等价于说$\lambda_1 <0<\lambda_n$。
	\end{itemize}
\end{itemize}
\normalfont

\di[二阶连续可微函数极值点的判定定理]{
	设$X_0\in\mathbb{R}^n$,$r>0$,
	而$f:B(X_0,r)\rightarrow \mathbb{R}$
	为二阶连续可微且$J_f(X_0)=\v{0}$。
	
	(1)若$H_f(X_0)$正定,则$X_0$为$f$的极小值点;
	
	(2)若$H_f(X_0)$负定,则$X_0$为$f$的极大值点。
}

\zhu[可导函数极值点的判定方法]{
	(1)求一阶偏导数,确定驻点。

	(2)求二阶偏导数以便得到 Hessian 矩阵。
	
	(3)判断 Hessian 矩阵在驻点处的性态:
	\begin{itemize}[topsep=0pt]
		\item 正定则为极小值点,负定则为极大值点;
		\item 不定则不为极值点;
		\item 半正定或半负定则需要采用另外的方法来处理。
	\end{itemize}
}

\begin{example}
	设隐函数$z=z(x,y)$由方程
	$F(x,y,z)=2x^2+2y^2+z^2+8xz-z+8=0$
	确定,求其极值点。
\end{example}
\begin{solution}
	\adjline
	由隐函数定理可知,$z(x,y)$的驻点满足
	\begin{equation*}
	0=\frac{\partial z}{\partial x}=-\frac{\,\dfrac{\partial F}{\partial x}\,}{\dfrac{\partial F}{\partial z}}=-\frac{4x+8z}{2z+8x-1},\qquad
	0=\frac{\partial z}{\partial y}=-\frac{\,\dfrac{\partial F}{\partial y}\,}{\dfrac{\partial F}{\partial z}}=-\frac{4y}{2z+8x-1}。
	\end{equation*}
	于是$y=0$,$x=-2z$,代入隐函数方程可得两个驻点
	$\begin{cases}
		x_1=\dfrac{16}{7},\\
		y_1=0,\\
		z_1=-\dfrac{8}{7},
	\end{cases}\quad
	\begin{cases}
		x_2=-2,\\
		y_2=0,\\
		z_2=1\text{。}
	\end{cases}$
	进而可求出$H_{z(x,y)}(x,y)$。
\end{solution}

\begin{theorem}
	设$X_0\in\mathbb{R}^n$,$r>0$,
	而$f:  B(X_0,r)\rightarrow \mathbb{R}$
	为二阶连续可微函数。
	
	(1)若$X_0$为$f$的极小值点,则$H_f(X_0)$为正定
	或半正定。
	
	(2)若$X_0$为$f$的极大值点,则$H_f(X_0)$为负定
	或半负定。
\end{theorem}
\begin{proofs}
	\adjline
	只需证(1)。由于$X_0$为函数$f$的极小值点,于是
	我们就有$J_{f}(X_0)=\v{0}$。固定向量$X\in\mathbb{R}^n\setminus\{\v{0}\}$。
	由带Peano余项的Taylor公式可知,
	当$t\rightarrow0$时,
	\begin{equation*}
	f(X_0+t X)=f(X_0)+\mbome{J_{f}(X_0)}tX
	+\frac{1}{2!}X^T \mbome{H_f(X_0)} X\cdot t^2+o(\|tX\|^2)
	\end{equation*}
	注意到$f(X_0+t X)\ge f(X_0)$,于是
	$0\le \dfrac{1}{2!}X^TH_f(X_0) X\cdot  t^2+ t^2 o(1)$,
	进而可得知$0\le X^TH_f(X_0) X$。这表明$H_f(X_0)$
	为正定或半正定。
\end{proofs}

\begin{example}
	设$D=\big\{(x,y)\in\mathbb{R}^2\ |\ x^2+y^2\le 1\big\}$,
	而且函数$u:D\rightarrow\mathbb{R}$在$D$上为连续且在$D$的内部
	为二阶连续可导。若$\forall (x,y)\in \mathop{\mathrm{Int}} D$,均有
	\begin{equation*}
	\dfrac{\partial^2 u}{\partial x^2}(x,y)
	+\dfrac{\partial^2 u}{\partial y^2}(x,y)=u(x,y)
	\end{equation*}
	
	(1)若$\forall (x,y)\in \partial D$,成立$u(x,y)\ge 0$,求证:
	$\forall (x,y)\in D$,均有$u(x,y)\ge 0$;
	(2)若$\forall (x,y)\in \partial D$,成立$u(x,y) > 0$,求证:
	$\forall (x,y)\in D$,均有$u(x,y) > 0$。
\end{example}
\begin{proofs}
	\adjline
	(1)用反证法,假设函数$u$在$D$上不为非负,
	由题设可得$u$在$\mathop{\mathrm{Int}}D$上不为非负。又$u$连续。而且$D$为有界闭集,于是$u$在$D$上有最小值。
	将相应的最小值点记作$P_0$。由于$u$在$\partial D$上为
	非负但在$\mathop{\mathrm{Int}}D$上却不为非负,于是$P_0\in\mathop{\mathrm{Int}} D$
	并且$u(P_0)<0$,从而$P_0$为$u$的极小值点,由此
	立刻可得 Hessian 矩阵$H_u(P_0)$为正定或半正定。于是我们就有
	\begin{equation*}
	\frac{\partial^2 u}{\partial x^2}(P_0)=\begin{pmatrix}
		1 & 0
	\end{pmatrix} H_u(P_0)
	\begin{pmatrix}
		1\\[-3pt]
		0
	\end{pmatrix}\ge 0,\qquad
	\frac{\partial^2 u}{\partial y^2}(P_0)=\begin{pmatrix}
		0 & 1
	\end{pmatrix} H_u(P_0)
	\begin{pmatrix}
		0\\[-3pt]
		1
	\end{pmatrix}\ge 0
	\end{equation*}
	但由题设又可得知
	\begin{equation*}
	\frac{\partial^2 u}{\partial x^2}(P_0)
	+\frac{\partial^2 u}{\partial y^2}(P_0)=u(P_0) <0
	\end{equation*}
	矛盾!故所证结论成立。

	(2)由于$u$为连续函数而且$\partial D$为有界闭集,
	则$u$在$\partial D$上有最小值,设为$m$,于是$m>0$。
	$\forall (x,y)\in D$,现定义$v(x,y)=u(x,y)-\dfrac{m(\e^{x}+\e^y)}{2\e}$,
	则$v$在$D$上连续,在$D$的内部二阶连续可导
	且使得$\forall (x,y)\in \mathop{\mathrm{Int}} D$,我们均有
	\begin{equation*}
	\frac{\partial^2 v}{\partial x^2}(x,y)
	+\frac{\partial^2 v}{\partial y^2}(x,y)=v(x,y)
	\end{equation*}
	另外,$\forall (x,y)\in \partial D$,我们还有
	\begin{equation*}
	v(x,y)\ge m-\frac{m(\e^{x}+\e^y)}{2\e}\ge 0
	\end{equation*}
	由(1)知$v$在$D$上非负,因此$\forall (x,y)\in D$,我们均有
	\begin{equation*}
	u(x,y)=v(x,y)+\frac{m(\e^{x}+\e^y)}{2\e}\ge \frac{m(\e^{x}+\e^y)}{2\e}>0
	\end{equation*}
	因此所证结论成立。
\end{proofs}

\subsubsection{条件极值与 Lagrange乘数法}

\de[高维曲面]{
	设$n>k\ge 1$为整数,$\varOmega\subset \mathbb{R}^n$为开集,
	$\varphi_1,\cdots,\varphi_{n-k}: \varOmega\rightarrow \mathbb{R}$为$\mathscr{C}^{(1)}$类 使得$\forall X\in \varOmega$,
	矩阵$\dfrac{\partial (\varphi_1,\cdots,\varphi_{n-k})}{\partial (x_1,\ldots,x_n)}(X)$的秩为$n-k$。   令
	\begin{equation*}
	S=\big\{X\in\varOmega \ | \ \varphi_i(X)=0,\ 1\le i\le n-k \big\}
	\end{equation*}
	若$S\neq \varnothing$,则称$S$为$\mboba{k}$\,\tboba{维曲面}。 
}

\de[条件极值]{
	假设$S\subset \mathbb{R}^n$为$k$维曲面,$X_0\in S$,而
	$f: S\rightarrow \mathbb{R}$为函数。 
	\begin{itemize}[leftmargin=1em]
		\item 如果$\exists r>0$使得$\forall X\in B(X_0,r)\cap S$,均有$f(X)\ge f(X_0)$,则称$X_0$为$f$在$S$上的\tboba{(条件)极小值点},而称$f(X_0)$为\tboba{(条件)极小值}。 
		\item 如果$\exists r>0$使得$\forall X\in B(X_0,r)\cap S$,均有$f(X)\le f(X_0)$,则称$X_0$为$f$在$S$上的\tboba{(条件)极大值点},称$f(X_0)$为\tboba{(条件)极大值}。
	\end{itemize}
}
\begin{definition}
	\textup{\textbf{条件最值\quad}}假设$S\subset \mathbb{R}^n$为$k$维曲面,$X_0\in S$,而
	$f: S\rightarrow \mathbb{R}$为函数。 
	\begin{itemize}
		\item 如果$\forall X\in S$,
		均有$f(X)\ge f(X_0)$,则称$X_0$为$f$在$S$上的\textbf{最小值点},称$f(X_0)$为\textbf{最小值}。 
		\item 如果$\forall X\in S$,
		均有$f(X)\le f(X_0)$,则称$X_0$为$f$在$S$上的\textbf{最大值点},称$f(X_0)$为\textbf{最大值}。
	\end{itemize}
\end{definition}

\di[Lagrange乘数法]{
	设$\varOmega\subset \mathbb{R}^n$为开集,
	而$f,\varphi_1,\cdots,\varphi_{n-k}:  \varOmega\rightarrow \mathbb{R}$为$\mathscr{C}^{(1)}$类函数,使得
	$\forall X\in \varOmega$,Hessian矩阵$\dfrac{\partial (\varphi_1,\cdots,\varphi_{n-k})}{\partial (x_1,\cdots,x_n)}$的
	秩为$n-k$,确定的$k$维曲面为
	$$S=\big\{X\in\varOmega \ | \ \varphi_i(X)=0,\ 1\le i\le n-k\big\}\neq \varnothing$$
	$\forall X\in \varOmega$及$\forall \lambda=(\lambda_1,\cdots,\lambda_{n-k})\in\mathbb{R}^{n-k}$,定义\tbome{拉氏函数}
	$$L(X,\lambda)=f(X)+\sum\limits_{j=1}^{n-k}\lambda_j\varphi_j(X)$$
	如果点$X_0\in S$为函数$f$在$S$上的条件极值点,则$\exists \lambda \in\mathbb{R}^{n-k}$使得$(X_0,\lambda)$为$L$的驻点。
}

\zhu[对 Lagrange乘数法的讨论]{
	点$(X_0,\lambda)$为$L$的驻点当且仅当
	\begin{equation*}
	\begin{cases}[ll]
		\dfrac{\partial L}{\partial x_i}(X_0,\lambda)=\dfrac{\partial f}{\partial x_i}(X_0)+\sum\limits_{j=1}^{n-k}\lambda_j\dfrac{\partial\varphi_j}{\partial x_i}(X_0)=0, & \text{(}1\le i\le n\text{)}\\
		\dfrac{\partial L}{\partial \lambda_i}(X_0,\lambda)=\varphi_i(X_0)=0, & \text{(}1\le i\le n-k\text{)}
	\end{cases}
	\end{equation*}
}

\begin{example}
	求空间椭圆
	$S: \begin{cases}
		\dfrac{x^2}{a^2}+\dfrac{y^2}{b^2}+\dfrac{z^2}{c^2}=1,\\
		lx+my+nz=0
	\end{cases}$~
	的长、短半轴的长度,其中$l^2+m^2+n^2=1$。
\end{example}

\begin{solution}
	\adjline
	$\forall (x,y,z)\in\mathbb{R}^3$,定义$f(x,y,z)=x^2+y^2+z^2$。
	椭圆的长、短半轴的长度也就是$\sqrt{f}$在$S$上的
	最大值和最小值,于是我们只需求$f$在$S$上的
	最大值和最小值。又$S$为有界闭集并且$f$连续,
	故$f$在$S$上有最值。   $\forall (x,y,z,\lambda,\mu)\in\mathbb{R}^5$,令
	\begin{equation*}
	L(x,y,z,\lambda,\mu)=x^2+y^2+z^2
	+\lambda\bigg(\frac{x^2}{a^2}+\frac{y^2}{b^2}+\frac{z^2}{c^2}-1\bigg)+\mu(lx+my+nz)
	\end{equation*}
	由Lagrange乘数法知,最值点$(x,y,z)$满足:
	\begin{align*}
	&0=\frac{\partial L}{\partial x}=2x+\frac{2\lambda x}{a^2}+\mu l,
	&&0=\frac{\partial L}{\partial y}=2y+\frac{2\lambda y}{b^2}+\mu m,
	&&0=\frac{\partial L}{\partial z}=2z+\frac{2\lambda z}{c^2}+\mu n,\\[-3pt]
	&0=\frac{\partial L}{\partial \lambda}=\frac{x^2}{a^2}+\frac{y^2}{b^2}+\frac{z^2}{c^2}-1,
	&&0=\frac{\partial L}{\partial \mu}=lx+my+nz
	\end{align*}
	由前三个关系式立刻可得
	\begin{equation*}
		2(x^2+y^2+z^2)+2\lambda\left(\frac{x^2}{a^2}+\frac{y^2}{b^2}+\frac{z^2}{c^2}\right)
		+\mu(lx+my+nz)=0
	\end{equation*}
	即得$$\lambda=-(x^2+y^2+z^2)$$
	同时我们也有
	\begin{equation*}
	x=-\frac{a^2l}{2(a^2+\lambda)}\mu,\qquad
	y=-\frac{b^2m}{2(b^2+\lambda)}\mu,\qquad
	z=-\frac{c^2n}{2(c^2+\lambda)}\mu
	\end{equation*}	
	由于原点不在$S$上,则$\mu\neq 0$,从而我们有
	\begin{equation*}
	0=-\frac{1}{\mu}(lx+my+nz) 
	=\frac{a^2l^2}{2(a^2+\lambda)}+\frac{b^2m^2}{2(b^2+\lambda)}+\frac{c^2n^2}{2(c^2+\lambda)}
	\end{equation*}
	出于简化记号,定义
	\begin{align*}
	A&=a^2l^2+b^2m^2+c^2n^2,\\[-5pt]
	B&=\frac{1}{2}\left(a^2l^2(b^2+c^2)+b^2m^2(c^2+a^2)+c^2n^2(a^2+b^2)\right),\\[-5pt]
	C&=a^2b^2c^2
	\end{align*}
	于是由前面的关系式可知$A\lambda^2+2B \lambda+C=0$。 
	我们由此立刻可得
	\begin{equation*}
	f(x,y,z)=-\lambda=\frac{B\pm \sqrt{B^2-AC}}{A}  
	\end{equation*}
	拉氏函数的驻点所对应的$f$的值只有两个,而$f$在$S$上有最值,故椭圆的长、短半轴分别为
	\begin{equation*}
	a^{*}=\bigg(\frac{B+\sqrt{B^2-AC}}{A}\bigg)^{\frac{1}{2}},\qquad
	b^{*}=\bigg(\frac{B-\sqrt{B^2-AC}}{A}\bigg)^{\frac{1}{2}} \qedhere
	\end{equation*}
\end{solution}

\zhu[求有界闭区域上的最值的一般方法]{
	极值或最值问题常可被转化有界闭区域上的连续函数的最值问题,由于问题的解一定存在,关键在于如何确定最值点。一般步骤为:

	(1)求函数在区域内部的驻点并计算相应值。

	(2)将函数限制在边界上,求相应的拉氏函数的驻点,并计算原来那个函数的相应值。

	(3)比较上述值的大小,由此确定最值点。
}
\begin{example}
	设$P_1=(0,0)$,$P_2=(1,0)$,$P_3=(0,1)$,而$D$为三角形$\Delta P_1P_2P_3$所围区域。$\forall P=(x,y)\in D$,令
	\begin{equation*}
		f(P)=|PP_1|^2+|PP_2|^2+|PP_3|^2
	\end{equation*}
	求$f$在$D$上的最大值和最小值。
\end{example}
\begin{solution}
	\adjline
	$\forall (x,y)\in D$,我们有
	\begin{equation*}
	f(x,y)=x^2+y^2+(x-1)^2+y^2+x^2+(y-1)^2=3x^2+3y^2-2x-2y+2
	\end{equation*}
	于是我们也可以将$f$看成是定义在整个$\mathbb{R}^2$上的初等函数,故$f$为$\mathscr{C}^{(1)}$类函数。由于$D$为有界闭集,故函数$f$在$D$上有最值。
	
	(1)如果$f$在$D$上的最值点在$D$的内部,那么该点必为$f$的局部极值点,在该点处,我们有
	\begin{equation*}
		0=\dfrac{\partial f}{\partial x}=6x-2,\qquad 0=\dfrac{\partial f}{\partial y}=6y-2
	\end{equation*}
	于是该点为$\left(\dfrac{1}{3},\dfrac{1}{3}\right)$,并且$f\left(\dfrac{1}{3},\dfrac{1}{3}\right)=\dfrac{4}{3}$。

	(2)若$f$在$D$上的最值点位于$D$的边界,那么该点为$f$的条件极值点。除了顶点以外,$\partial D$由下述线段组成:
	$C_1: y=0,0<x<1$,
	$C_2: x=0,0<y<1$,
	$C_3: x+y=1,0<x<1$。
	于是我们需要来分别考虑$f$在$C_1,C_2,C_3$上的条件极值,相应的Lagrange函数为
	\begin{align*}
		&L_1(x,y,\lambda_1)=f(x,y)+\lambda_1 y,\\[-5pt]
		&L_2(x,y,\lambda_2)=f(x,y)+\lambda_2 x,\\[-5pt]
		&L_3(x,y,\lambda_3)=f(x,y)+\lambda_3 (x+y-1)
	\end{align*}
	拉氏函数$L_1$的驻点满足
	$\begin{cases}
		0=\dfrac{\partial L_1}{\partial x}=6x-2,\\[5pt]
		0=\dfrac{\partial L_1}{\partial y}=6y-2+\lambda_1,\\[5pt]
		0=\dfrac{\partial L_1}{\partial \lambda_1}=y,
	\end{cases}$~
	从而该点为$\left(\dfrac{1}{3},0,2\right)$,并且$f\left(\dfrac{1}{3},0\right)=\dfrac{5}{3}$;

	拉氏函数$L_2$的驻点满足
	$\begin{cases}
		0=\dfrac{\partial L_2}{\partial x}=6x-2+\lambda_2,\\[5pt]
		0=\dfrac{\partial L_2}{\partial y}=6y-2,\\[5pt]
		0=\dfrac{\partial L_2}{\partial \lambda_2}=x,
	\end{cases}$~
	则该点为$\left(0,\dfrac{1}{3},2\right)$,并且我们有$f\left(0,\dfrac{1}{3}\right)=\dfrac{5}{3}$;

	拉氏函数$L_3$的驻点满足
	$\begin{cases}
		0=\dfrac{\partial L_3}{\partial x}=6x-2+\lambda_3,\\[5pt]
		0=\dfrac{\partial L_3}{\partial y}=6y-2+\lambda_3,\\[5pt]
		0=\dfrac{\partial L_3}{\partial \lambda_3}=x+y-1,
	\end{cases}$~
	故该点为$\left(\dfrac{1}{2},\dfrac{1}{2},-1\right)$,并且我们有$f\left(\dfrac{1}{2},\dfrac{1}{2}\right)=\dfrac{3}{2}$。

	另外,在三个顶点处,我们有
	\begin{equation*}
	f(P_1)=2,\qquad f(P_2)=3,\qquad f(P_3)=3
	\end{equation*}
	由于$f$在$D$上有最值,故$f$在$D$上的最值点必在上述点中,通过比较$f$在这些点处的值知$f$在点$P_2,P_3$处取到最大值$3$,而在点$\left(\dfrac{1}{3},\dfrac{1}{3}\right)$处取到最小值$\dfrac{4}{3}$。
\end{solution}

\subsection{曲面与曲线}

我们已知,取$P_0(x_0,y_0,z_0)\in \mathbb{R}^3$,设 $\v{e}=(a,b,c)\tp \in\mathbb{R}^3$为非零向量,过$P_0$沿方向$\v{e}$的直线$\Gamma$的方程为
\begin{equation*}
	\dfrac{x-x_0}{a}=\dfrac{y-y_0}{b}=\dfrac{z-z_0}{c}
\end{equation*}
该直线也可以表示成
$\begin{cases}
	x=x_0+at,\\[-3pt]
	y=y_0+bt,\\[-3pt]
	z=z_0+ct\text{。}
\end{cases}$若$abc=0$,则零分量的对应分子也为零,形式上也写成上面形式。

过$P_0$并且与$\Gamma$垂直的平面$S$称为$\Gamma$过$P_0$的\tboba{法平面},它的方程为
$\begin{pmatrix}
	x-x_0\\[-3pt]
	y-y_0\\[-3pt]
	z-z_0
\end{pmatrix}\cdot\begin{pmatrix}
	a\\[-3pt]
	b\\[-3pt]
	c
\end{pmatrix}=0$,
也就是说$a(x-x_0)+b(y-y_0)+c(z-z_0)=0$。
我们称$\v{e}$为平面$S$的\tboba{法向量},$\Gamma$为$S$的\tboba{法线}。

\subsubsection{曲面的表示}

\paragraph{曲面的显函数表示法}

曲面$S:z=f(x,y)$,$(x,y)\in D\subset \mathbb{R}^2$。
假设$f$在点$(x_0,y_0)$处可微,令$z_0=f(x_0,y_0)$。当$(x,y)\rightarrow (x_0,y_0)$时,
\begin{align*}
	f(x,y)-z_0&=f(x,y)-f(x_0,y_0)\\
	&=\dfrac{\partial f}{\partial x}(x_0,y_0)(x-x_0)+\dfrac{\partial f}{\partial y}(x_0,y_0)(y-y_0)+o\left(\sqrt{(x-x_0)^2+(y-y_0)^2}\right)
\end{align*}
则定义曲面$S$在点$(x_0,y_0,z_0)$处的\tboba{切平面}方程为
\begin{equation*}
	z-z_0=\dfrac{\partial f}{\partial x}(x_0,y_0)(x-x_0)+\dfrac{\partial f}{\partial y}(x_0,y_0)(y-y_0)
\end{equation*}
即
\begin{equation*}
	\mboba{\dfrac{\partial f}{\partial x}(x_0,y_0)}(x-x_0)+ \mboba{\dfrac{\partial f}{\partial y}(x_0,y_0)}(y-y_0)+ \mboba{(-1)}(z-z_0)=0
\end{equation*}
或者
\begin{equation*}
	\mboba{\dfrac{\partial f}{\partial x}(x_0,y_0)}x +  \mboba{\dfrac{\partial f}{\partial y}(x_0,y_0)}y +  \mboba{(-1)}z=D
\end{equation*}
于是该切平面的法向量为
$\v{n}=\begin{pmatrix}
	\dfrac{\partial f}{\partial x}(x_0,y_0)\\[5pt]
	\dfrac{\partial f}{\partial y}(x_0,y_0)\\
	-1
\end{pmatrix}$,相应的法线方程为
$$\dfrac{x-x_0}{\mboba{\frac{\partial f}{{\partial x}}(x_0,y_0)}}=\dfrac{y-y_0}{\mboba{\frac{\partial f}{\partial y}(x_0,y_0)}}=\dfrac{z-z_0}{\mboba{-1}}$$

\paragraph{曲面的参数表示法}

考虑曲面$S$:   $\begin{cases}
	x=f_1(u,v),\\[-3pt]
	y=f_2(u,v),\\[-3pt]
	z=f_3(u,v),
\end{cases}
~(u,v)\in D\subset \mathbb{R}^2$。
设$(u_0,v_0)\in D$,$f_1,f_2,f_3$在点$(u_0,v_0)$可微。令
\begin{equation*}
	\begin{pmatrix}
		x_0\\[-3pt]
		y_0\\[-3pt]
		z_0
	\end{pmatrix}=\begin{pmatrix}
		f_1(u_0,v_0)\\[-3pt]
		f_2(u_0,v_0)\\[-3pt]
		f_3(u_0,v_0)
	\end{pmatrix}
\end{equation*}
当$(u,v)\rightarrow (u_0,v_0)$时,我们有
\begin{equation*}
	\begin{pmatrix}
		f_1(u,v)-x_0\\[-3pt]
		f_2(u,v)-y_0\\[-3pt]
		f_3(u,v)-z_0
	\end{pmatrix}=\dfrac{\partial (f_1,f_2,f_3)}{\partial (u,v)}(u_0,v_0)
	\begin{pmatrix}
		u-u_0\\[-3pt]
		v-v_0
	\end{pmatrix}
	+\v{o}\left(\sqrt{(u-u_0)^2+(v-v_0)^2}\right)
\end{equation*}
当矩阵$\dfrac{\partial (f_1,f_2,f_3)}{\partial (u,v)}(u_0,v_0)$ 的秩等于$2$时,曲面$S$在点$(x_0,y_0,z_0)$处有切平面
\begin{equation*}
	\begin{pmatrix}
		x-x_0\\[-3pt]
		y-y_0\\[-3pt]
		z-z_0
	\end{pmatrix}=\dfrac{\partial (f_1,f_2,f_3)}{\partial (u,v)}(u_0,v_0)\begin{pmatrix}
		u-u_0\\[-3pt]
		v-v_0
	\end{pmatrix}
\end{equation*}
即
\begin{align*}
	\begin{pmatrix}
		x-x_0\\[-3pt]
		y-y_0\\[-3pt]
		z-z_0
	\end{pmatrix}=\begin{pmatrix}
		\dfrac{\partial f_1}{\partial u}& \dfrac{\partial f_1}{\partial v}\\[5pt]
		\dfrac{\partial f_2}{\partial u}& \dfrac{\partial f_2}{\partial v}\\[5pt]
		\dfrac{\partial f_3}{\partial u}& \dfrac{\partial f_3}{\partial v}
	\end{pmatrix}
	\begin{pmatrix}
		u-u_0\\[-3pt]
		v-v_0
	\end{pmatrix}
	\Longleftrightarrow \begin{cases}
		x-x_0=\dfrac{\partial f_1}{\partial u}\cdot (u-u_0)+\dfrac{\partial f_1}{\partial v}\cdot (v-v_0)\\[5pt]
		y-y_0=\dfrac{\partial f_2}{\partial u}\cdot (u-u_0)+\dfrac{\partial f_2}{\partial v}\cdot (v-v_0)\\[5pt]
		z-z_0=\dfrac{\partial f_3}{\partial u}\cdot (u-u_0)+\dfrac{\partial f_3}{\partial v}\cdot (v-v_0)
	\end{cases}
\end{align*}
消去$u,v$,该切平面也可以表示成
\begin{equation*}
	\mboba{\dfrac{D(f_2,f_3)}{D(u,v)}(u_0,v_0)}(x-x_0)
	+\mboba{\dfrac{D(f_3,f_1)}{D(u,v)}(u_0,v_0)}(y-y_0)
	+\mboba{\dfrac{D(f_1,f_2)}{D(u,v)}(u_0,v_0)}(z-z_0)=0
\end{equation*}
从而曲面$S$在点$(x_0,y_0,z_0)$处的法线方程为
\begin{equation*}
	\dfrac{x-x_0}{\mboba{\frac{D(f_2,f_3)}{D(u,v)}(u_0,v_0)}}
	=\dfrac{y-y_0}{\mboba{\frac{D(f_3,f_1)}{D(u,v)}(u_0,v_0)}}
	=\dfrac{z-z_0}{\mboba{\frac{D(f_1,f_2)}{D(u,v)}(u_0,v_0)}}
\end{equation*}

\paragraph{曲面的隐函数表示法}

考虑$S:F(x,y,z)=0$。设$P_0(x_0,y_0,z_0)\in S$,而$F$在点$P_0$处可微。则当$S\ni P(x,y,z)\rightarrow P_0$时,我们有
\begin{align*}
	0&=F(x,y,z)-F(x_0,y_0,z_0)\\
	&=\dfrac{\partial F}{\partial x}(P_0)(x-x_0)+
	\dfrac{\partial F}{\partial y}(P_0)(y-y_0)+\dfrac{\partial F}{\partial z}(P_0)(z-z_0)
	+o(\|P-P_0\|)
\end{align*}
从而当$J_{F}(P_0)\neq \v{0}$时,曲面在点$P_0$有切平面
\begin{equation}
	\mboba{\dfrac{\partial F}{\partial x}(P_0)}(x-x_0)
	+\mboba{\dfrac{\partial F}{\partial y}(P_0)}(y-y_0)
	+\mboba{\dfrac{\partial F}{\partial z}(P_0)}(z-z_0)=0
\end{equation}
于是曲面$S$在点$P_0$处的法向量为
$\v{n}=\begin{pmatrix}
	\dfrac{\partial F}{\partial x}(P_0)\\[7pt]
	\dfrac{\partial F}{\partial y}(P_0)\\[7pt]
	\dfrac{\partial F}{\partial z}(P_0)
\end{pmatrix}
=\mathop{\mathrm{grad}}F(P_0)$,
相应的法线方程为
\begin{equation*}
	\dfrac{x-x_0}{\mboba{\frac{\partial F}{\partial x}(P_0)}}
	=\dfrac{y-y_0}{\mboba{\frac{\partial F}{\partial y}(P_0)}}
	=\dfrac{z-z_0}{\mboba{\frac{\partial F}{\partial z}(P_0)}}
\end{equation*}

\de[曲面的正交]{
	若两曲面在交线上每点的法线互相垂直,则称这两个曲面\tboba{正交}。
}
\di[曲面正交的充要条件]{
	曲面$S_1:  \ F_1(x,y,z)=0$和曲面$S_2:F_2(x,y,z)=0$正交的充分必要条件是对于交线上的每点$P_0(x_0,y_0,z_0)$,均有
	\begin{equation*}
	\dfrac{\partial F_1}{\partial x}\dfrac{\partial F_2}{\partial x}
	+\dfrac{\partial F_1}{\partial y}\dfrac{\partial F_2}{\partial y}
	+\dfrac{\partial F_1}{\partial z}\dfrac{\partial F_2}{\partial z}=0
	\end{equation*}
}
\begin{proofs}
	在上述两曲面的交线上任取一点$P_0$,它们的法向量分别为$\mathop\mathrm{grad}F_1(P_0)$,$\mathop\mathrm{grad}F_2(P_0)$,二者正交当且仅当$\mathop\mathrm{grad}F_1(P_0)\cdot \mathop\mathrm{grad}F_2(P_0)=0$。由此得证。
\end{proofs}

\subsubsection{空间曲线的表示}

\paragraph{空间曲线的参数表示法}

考虑$\Gamma: \begin{cases}
	x=x(t),\\[-3pt]
	y=y(t),\\[-3pt]
	z=z(t),
\end{cases}
~~t\in [\alpha,\beta]$,
若上述这些函数在点$t=t_0$处可微,则称曲线$\Gamma$在
相应点$P_0(x_0,y_0,z_0)$处可微,相应\tboba{切线}方程为
$\begin{cases}
	x-x_0=x'(t_0)(t-t_0),\\[-3pt]
	y-y_0=y'(t_0)(t-t_0),\\[-3pt]
	z-z_0=z'(t_0)(t-t_0)\text{。}
\end{cases}$~
假设$\big(x'(t_0),y'(t_0),$ $z'(t_0)\big)$不为零向量,该切线也可表述成
\begin{equation*}
	\dfrac{x-x_0}{x'(t_0)}=\dfrac{y-y_0}{y'(t_0)}=\dfrac{z-z_0}{z'(t_0)},
\end{equation*}
我们将经过点$P_0$并且与上述切线垂直的平面称为$\Gamma$在点$P_0$处的\tboba{法平面},其方程为
\begin{equation*}
	x'(t_0)(x-x_0)+y'(t_0)(y-y_0)+z'(t_0)(z-z_0)=0
\end{equation*}

\paragraph{空间曲线的隐函数表示法}

考虑$\Gamma: \begin{cases}
	F_1(x,y,z)=0,\\[-3pt]
	F_2(x,y,z)=0,
\end{cases}$~~
设$F_1,F_2$在点$P_0(x_0,y_0,z_0)$可微 且$\mathop\mathrm{grad}F_1(P_0)$,$\mathop\mathrm{grad}F_2(P_0)$不为零,则曲线$\Gamma$在该点的切线为
\begin{equation*}
	\begin{cases}
		\dfrac{\partial F_1}{\partial x}(P_0)(x-x_0)+
		\dfrac{\partial F_1}{\partial y}(P_0)(y-y_0)+
		\dfrac{\partial F_1}{\partial z}(P_0)(z-z_0)=0,\\[7pt]
		\dfrac{\partial F_2}{\partial x}(P_0)(x-x_0)+
		\dfrac{\partial F_2}{\partial y}(P_0)(y-y_0)+
		\dfrac{\partial F_2}{\partial z}(P_0)(z-z_0)=0
	\end{cases}
\end{equation*}
该切线的方向为
\begin{equation*}
	\v{T}=\mathop\mathrm{grad}F_1(P_0)\times \mathop\mathrm{grad}F_2(P_0)\\
	=\begin{pmatrix}
		\dfrac{\partial F_1}{\partial x}(P_0)\\[7pt]
		\dfrac{\partial F_1}{\partial y}(P_0)\\[7pt]
		\dfrac{\partial F_1}{\partial z}(P_0)
	\end{pmatrix}\times\begin{pmatrix}
		\dfrac{\partial F_2}{\partial x}(P_0)\\[7pt]
		\dfrac{\partial F_2}{\partial y}(P_0)\\[7pt]
		\dfrac{\partial F_2}{\partial z}(P_0)
	\end{pmatrix}
	=\begin{pmatrix}
		\dfrac{D(F_1,F_2)}{D(y,z)}(P_0)\\[7pt]
		\dfrac{D(F_1,F_2)}{D(z,x)}(P_0)\\[7pt]
		\dfrac{D(F_1,F_2)}{D(x,y)}(P_0)
	\end{pmatrix}
\end{equation*}
只有当$\v{T}\neq \v{0}$时,上述方程组才的确给出一条直线,此时Jacobi矩阵$\dfrac{\partial (F_1,F_2)}{\partial (x,y,z)}(P_0)$的秩等于$2$。
借助$\v{T}$,我们也可得到切线的另外一个表述: 
\begin{equation*}
	\dfrac{x-x_0}{\frac{D(F_1,F_2)}{D(y,z)}(P_0)}
	=\dfrac{y-y_0}{\frac{D(F_1,F_2)}{D(z,x)}(P_0)}
	=\dfrac{z-z_0}{\frac{D(F_1,F_2)}{D(x,y)}(P_0)}
\end{equation*}

\newpage
%----------------------------------------------------------
\section{多元函数积分学}

\subsection[$n$重积分]{$\boldsymbol{n}$\,重积分}

\begin{definition}
	定义$I=\big\{(x_1,\cdots,x_n) \in \R^n ~\big|~ a_j \le x_j \le b_j,j \in \{1,\cdots,n\}\big\}$为$\mathbb{R}^n$ 中的\textbf{区间}或者\textbf{坐标平行体},其$n$维体积被定义为
	\begin{equation*}
	\mu_n(I):=|I|:=\prod\limits_{j=1}^n(b_j-a_j)
	\end{equation*}
\end{definition}
\de[Riemann可积]{\adjline
	设$\varOmega\subset \mathbb{R}^n$为有界集,$f:\varOmega \rightarrow \mathbb{R}$为函数,
	\begin{itemize}[leftmargin=1em]
		\item 将 每 一 个区间$[a_j,b_j]$($1\le j\le n$) 分成更 小的子区间,由此而得到的小坐标平行体所组成的集合,称为$I$的一个\tboba{分割},记作$P$,$\lambda(P)=\max_{J\in P}d(J)$为分割$P$的\tboba{步长},其 中$d(J)$ 表  示  坐标平行体$J$ 的直径,即$J$ 中任意两点的最大距离;
		
		\item 假 设$P=\{I_j\ |\ 1\le j\le k\}$ 为$I$ 的分 割,对 任 意 的整数$1\le j\le k$,选取$\xi_j\in I_j$ ,记$\xi=(\xi_j)_{1\le j\le k}$,称$(P,\xi)$为$I$的\tboba{带点分割};
		
		\item 设$\widetilde{f}:I\rightarrow \mathbb{R}$为函数,定义
		$$\sigma(\widetilde{f};P,\xi)=\sum\limits_{j=1}^k\widetilde{f}(\xi_j)|I_j|$$
		称为$\widetilde{f}$关于带点分割$(P,\xi)$的~\tboba{Riemann和};
		
		\item 若$\exists A\in\mathbb{R}$使得$\forall \varepsilon>0$,$\exists \delta>0$使得对$I$的任意带点分割$(P,\xi)$,当$\lambda(P)<\delta$时,均有$|\sigma(\widetilde{f};P,\xi)-A|<\varepsilon$。此时,我们记$A=\lim\limits_{\lambda(P)\rightarrow 0}\sigma(\widetilde{f};P,\xi)$,并称之为$\widetilde{f}$在$I$上的积分,记作$$\dint_I\widetilde{f}(X) \dif X \quad\text{或}\quad \idotsint_I \widetilde{f}(x_1,\cdots, x_n) \dif x_1\cdots \dif x_n$$并称$\widetilde{f} : I\rightarrow\mathbb{R}$ 为~\tboba{Riemann 可积}。
		
		\item 对有 界 集$\varOmega\subset \mathbb{R}^n$,可找到坐标平行体$I$包 含$\varOmega$,$\forall X\in I$,定 义
		$$\widetilde{f}(X)=\begin{cases}[ll]
			f(X),	&	X\in \varOmega,\\
			0,		&	X\in I\setminus\varOmega
		\end{cases}$$
		如果$\widetilde{f}$ 在$I$ 上为Riemann可积,
		则称$f$在$\varOmega$上为~\tboba{Riemann可积},此时定义
		$\displaystyle \int_{\varOmega}f(X) \dif X=\int_I\widetilde{f}(X) \dif X$,可 以 证 明 上 述定义与坐标平行体$I$ 的选取无关。$\varOmega$ 上的所有的Riemann 可积函数的全体记作$\mathscr{R}(\varOmega)$,该集合可能「非常小」。
	\end{itemize}
}
同单变量情形一样, 可引入Darboux上和、Darboux下和、 振幅, 进而借助它们来刻画Riemann可积函数。
\begin{definition}
	设$\varOmega\subset\mathbb{R}^n$,$\forall X\in\mathbb{R}^n$,定义
	$1_{\varOmega}(X)=
	\begin{cases}[ll]
		1,&X\in \varOmega,\\
		0,&X\not\in\varOmega,
	\end{cases}$~
	并称$1_{\varOmega}$为集合$\varOmega$的\textbf{示性函数}。 
	若$\varOmega\subset\mathbb{R}^n$为有界集且使得其示性函数$1_{\varOmega}$为\textup{Riemann}可积,则称$\varOmega$为\textbf{\textup{Jordan}可测集},此时还称$\displaystyle \int_{\varOmega}1_{\varOmega}(X)\dif X$为$\varOmega$ 的\textbf{体积}或\textbf{测度},记作$|\varOmega|$。
\end{definition}

\di[Jordan可测集上的连续函数 Riemann可积]{
	设有界闭集$\varOmega\subset\mathbb{R}^n$为Jordan可测集,
	$f:\varOmega\rightarrow\mathbb{R}$连续,
	则$f$在$\varOmega$上Riemann可积。
}

\begin{theorem}
	\textup{Riemann}积分的基本性质。
	\begin{itemize}
		\item \textbf{\textup{有界性\quad}}若$f\in\mathscr{R}(\varOmega)$,则$f$为有界函数。
		
		\item \textbf{\textup{线性性\quad}}$\forall f_1,f_2 \in \mathscr{R}(\varOmega)$ 以及$\forall a,b\in\mathbb{R}$,我们均有$af_1+bf_2\in \mathscr{R}(\varOmega)$,并且
		\begin{equation*}
		\int_{\varOmega}\left(af_1(X)+bf_2(X)\right)\dif X
		=a \int_{\varOmega} f_1(X)\dif X+b \int_{\varOmega} f_2(X)\dif X
		\end{equation*}

		\item \textbf{\textup{区域可加性\quad}}设$\varOmega_1,\varOmega_2\subset \varOmega$ 为Jordan 可测集,而$\varOmega=\varOmega_1\cup \varOmega_2$, 且$\varOmega_1,\varOmega_2$没有公共的内点,则函数$f:\varOmega\rightarrow \mathbb{R}$在$\varOmega$上为\textup{Riemann}可积当且仅当$f$ 在$\varOmega_1,\varOmega_2$ 上\textup{Riemann} 可积,此时
		\begin{equation*}
		\int_{\varOmega}f(X)\dif X=\int_{\varOmega_1}f(X)\dif X+\int_{\varOmega_2}f(X)\dif X.
		\end{equation*}
		
		\item \textbf{\textup{保号性\quad}}如果$f\in\mathscr{R}(\varOmega)$ 使得$\forall X\in \varOmega$,我们均有$f(X)\ge 0$,则$\displaystyle\int_{\varOmega}f(X)\dif X\ge 0$。
		
		\item \textbf{\textup{严格保号性\quad}}若$f\in\mathscr{C}(\varOmega)$非负且不恒为零,则我们有$\displaystyle\int_{\varOmega}f(X)\dif X> 0$。
		
		\item \textbf{\textup{保序性\quad}}若$f,g\in\mathscr{R}(\varOmega)$使得$\forall X\in \varOmega$,我们均有$f(X)\le g(X)$,则
		\begin{equation*}
		\int_{\varOmega}f(X)\dif X\le \int_{\varOmega}g(X)\dif X
		\end{equation*}
		
		\item \textbf{\textup{绝对值不等式\quad}}若$f\in\mathscr{R}(\varOmega)$,
		则$|f|\in\mathscr{R}(\varOmega)$ 且
		\begin{equation*}
		\left|\int_{\varOmega}f(X)\dif X\right|\le \int_{\varOmega}|f(X)|\dif X.
		\end{equation*}
		
		\item \textbf{\textup{积分估计界\quad}}若$f\in\mathscr{R}(\varOmega)$,$M,m$ 为其上、下界,则
		\begin{equation*}
		m|\varOmega|\le \int_{\varOmega}f(X)\dif X\le M|\varOmega|
		\end{equation*}
		
		\item \textbf{\textup{积分中值定理\quad}}若$\varOmega$还为有界的闭连通集,
		而$f\in\mathscr{C}(\varOmega)$,
		则$\exists X_0\in \varOmega$使得
		\begin{equation*}
		\int_{\varOmega}f(X)\dif X=f(X_0)|\varOmega|. 
		\end{equation*}
		由此立刻可知, $\forall Y\in \mathrm{Int}\,\varOmega$,我们有
		\begin{equation*}
		f(Y)=\lim_{r\rightarrow 0^{+}}\frac{1}{|\bar{B}(Y,r)|}\int_{\bar{B}(Y,r)}f(X)\dif X
		\end{equation*}

		\item \textbf{\textup{变量替换定理\quad}}\label{变量替换定理}设$\varOmega_1,\varOmega_2\subset\mathbb{R}^n$ 为非空 开 集,
		$\varphi= (g_1,\cdots,g_n) : \varOmega_1\rightarrow \varOmega_2$ 为 连续可导的双射,并且它的逆映射$\varphi^{-1}:\varOmega_2\rightarrow \varOmega_1$ 也为连续可导。
		若$D_1\subset \varOmega_1$ 为\textup{Jordan}可测集,
		那么$D_2=\varphi(D_1)$也为\textup{Jordan}可测集,且$\forall f\in\mathscr{C}(D_2)$,均有
		\begin{align*}
			\int_{{D_2}} f(Y)\dif Y 
			&=\idotsint_{{\varphi(D_1)}}f(y_1,\cdots,y_n)\dif y_1\cdots \dif y_n\\[-3pt]
			&=\idotsint_{{D_1}} f(g_1(X),\cdots,g_n(X))\left|\frac{D(g_1,\cdots,g_n)}{D(x_1,\cdots,x_n)}\right|\dif x_1\cdots \dif x_n
		\end{align*}
	\end{itemize}
\end{theorem}



\subsection{二重积分的计算}

\subsubsection{直角坐标系下的二重积分}

\begin{circum}
	假设$f_1,f_2 :[a,b]\rightarrow\mathbb{R}$ 为连续函数使得
	$\forall x\in [a,b]$,均有$f_1(x)\le f_2(x)$。则
	\begin{equation*}
		D_1=\big\{(x,y)\in\mathbb{R}^2\ |\ a\le x\le b,\ f_1(x)\le y\le f_2(x)\big\}
	\end{equation*}
	为Jordan可测且$|D_1|=\dint_a^b\left(f_2(x)-f_1(x)\right)\dif x$.

	若$f:D_1\rightarrow\mathbb{R}$为连续函数,则
	\begin{equation*}
		\iint_{D_1}f(x,y)\dif x\dif y=\int_a^b\left(\int_{f_1(x)}^{f_2(x)}f(x,y)\dif y\right)\dif x
	\end{equation*}
\end{circum}

\begin{circum}
	假设$g_1,g_2:[c,d]\rightarrow\mathbb{R}$ 为连续函数使得
	$\forall y\in [c,d]$,均有$g_1(y)\le g_2(y)$。则
	\begin{equation*}
	D_2=\big\{(x,y)\in\mathbb{R}^2\ |\ g_1(y)\le x\le g_2(y),\ c\le y\le d\big\}
	\end{equation*}为Jordan可测且$|D_2|=\dint_c^d\left(g_2(y)-g_1(y)\right)\dif y$.

	若$f:D_2\rightarrow\mathbb{R}$为连续函数,则
	\begin{equation*}
	\iint_{D_2}f(x,y)\dif x\dif y=\int_c^d\left(\int_{g_1(y)}^{g_2(y)}f(x,y)\dif x\right)\dif y.
	\end{equation*}
\end{circum}

\begin{example}
	计算$\displaystyle I=\int_0^1\left(\int_x^1\e^{-y^2}\dif y\right)\dif x$。
\end{example}
\begin{solution}
	看作 X型区域上的二重积分式,即令$D= \big\{(x,y)\in\mathbb{R}^2\ |\ 0\le x \le 1,\ x\le y\le 1\big\}$,此时$\int_x^1\e^{-y^2}\dif y$无解析表示。但注意到$D=\big\{(x,y)\in\mathbb{R}^2\ |\ 0\le x\le y,\ 0\le y\le 1\big\}$,故可看作 Y型区域上的二重积分,有
	\begin{equation*}
	I=\iint_{D}\e^{-y^2}\dif x\dif y
	=\int_0^1\left(\int_0^y\e^{-y^2}\dif x\right)\dif y
	=\int_0^1y\e^{-y^2}\dif y
	=-\frac{1}{2}\e^{-y^2}\Big|_0^1=\frac{1}{2}(1-\e^{-1}) \qedhere
	\end{equation*}	
\end{solution}

\begin{example}
	\textbf{体积计算一例\quad}计算椭圆形的圆柱面$4x^2+y^2=1$ 与平面
	$z=1-y$以及$z=0$所围成的立体的体积。
\end{example}
\begin{solution}
	令$D= \big\{(x,y)\in\mathbb{R}^2\ |\ 4x^2+y^2\le 1\big\}$,则所求立体的体积为
	\begin{align*}
		V &=\iint_{D}(1-y)\dif x\dif y
		=\int_{-\frac{1}{2}}^{\frac{1}{2}}\left(\int_{-\sqrt{1-4x^2}}^{\sqrt{1-4x^2}}(1-y)\dif y\right)\dif x
		=2\int_{-\frac{1}{2}}^{\frac{1}{2}}\sqrt{1-4x^2}\dif x\\[-3pt]
		&\xlongequal{x=\frac{1}{2}\sin t} 4\int_0^{\frac{\pi}{2}}\cos t\dif \left(\frac{1}{2}\sin t\right)
		=2\int_0^{\frac{\pi}{2}}\cos^2t\dif t
		=\left(\frac{1}{2}\sin 2t+t\right)\Bigg|_0^{\frac{\pi}{2}}
		=\frac{\pi}{2} \qedhere
	\end{align*}
\end{solution}

\zhu[对称性在二重积分当中的应用]{
	假设积分区域$D$关于$x$ 轴对称,
	如果有
	\begin{itemize}[leftmargin=2em, labelsep=0em, itemindent=0em]
		\item[(a)]$f(x,-y)=-f(x,y)$,则$\displaystyle\iint_Df(x,y)\dif x\dif y=0$。
		\begin{proofs}
			\adjline
			由定理~\ref{变量替换定理}~中变量替换定理,
			\begin{align*}
			\iint_Df(x,y)\dif x\dif y &\xlongequal[y=-v]{x=u}\iint_Df(u,-v)\left|\frac{D(u,-v)}{D(u,v)}\right|\dif u\dif v \\[-3pt]
			&=-\iint_Df(u,v)\dif u\dif v = -\iint_Df(x,y)\dif x\dif y
			\end{align*}
			故$\displaystyle\iint_Df(x,y)\dif x\dif y=0$。
		\end{proofs}
		\item[(b)]$f(x,-y)=f(x,y)$,则$\displaystyle\iint_Df(x,y)\dif x\dif y=2\iint_{D'}f(x,y)\dif x\dif y$,其中$D'$为$D$位于$x$轴上侧(或下侧)的部分。
		\begin{proofs}
			\adjline
			由变量替换可得$\displaystyle\iint_{D''}  f(x,y)\dif u\dif v \xlongequal[y=-v]{x=u}\iint_{D'} f(u,-v)\left|\frac{D(u,-v)}{D(u,v)}\right|\dif u\dif v$,于是
			\begin{align*}
				\iint_D f(x,y)\dif x\dif y
				&=\iint_{D'}  f(x,y)\dif x\dif y+\iint_{D''}  f(x,y)\dif u\dif v\\[-3pt]
				&=\iint_{D'}  f(x,y)\dif x\dif y +\iint_{D'}  f(u,-v)\left|\frac{D(u,-v)}{D(u,v)}\right|\dif u\dif v\\[-3pt]
				&=2\iint_{D'}f(x,y)\dif x\dif y \qedhere
			\end{align*}
		\end{proofs}
	\end{itemize}
	同理,假设积分区域$D$关于$y$ 轴对称,
	如果有
	\begin{itemize}[leftmargin=2em, labelsep=0em, itemindent=0em]
		\item[(a)]$f(-x,y)=-f(x,y)$,则$\displaystyle\iint_Df(x,y)\dif x\dif y=0$。
		\item[(b)]$f(-x,y)=f(x,y)$,则$\displaystyle\iint_Df(x,y)\dif x\dif y=2\iint_{D'}f(x,y)\dif x\dif y$,其中$D'$为$D$位于$y$轴左侧(或右侧)的部分。
	\end{itemize}
	假设积分区域$D$关于原点对称,如果有$f(-x,-y)=-f(x,y)$,则	$\displaystyle\iint_Df(x,y)\dif x\dif y=0$。
}

\subsubsection{极坐标系下的二重积分}

极坐标$(\rho,\varphi)$和直角坐标$(x,y)$之间有映射关系$\v{f}:\begin{cases}
	x=\rho\cos\varphi,\\[-3pt] y=\rho\sin\varphi,
\end{cases}$~~其Jacobi行列式为
$\dfrac{D(x,y)}{D(\rho,\varphi)}=\begin{vmatrix}
	\cos\varphi & -\rho\sin\varphi \\[-3pt] \sin\varphi & \rho\cos\varphi
\end{vmatrix}=\rho$,于是由定理~\ref{变量替换定理}~中变量替换定理,即有
\begin{equation*}
	\iint_{\v{f}(D)} F(x,y)\dif x\dif y =\iint_{D} F(\rho\cos\varphi,\rho\sin\varphi)\rho \dif \rho\dif \varphi
\end{equation*}



\begin{example}
	计算由抛物线$y^{2}=px$,$y^{2}=qx$与双曲线$xy=a,xy=b$合起来所围成的平面区域$D$的面积,其中$q>p>0,b>a>0$。
\end{example}
\begin{solution}
	\adjline
	作变换$u=\dfrac{y^{2}}{x},v=xy$,则$D$变为$D_{1}=\left\{(u,v)\mid p\le u\le q,a\le v\le b\right\}$,知 Jacobi行列式
	$\dfrac{D(u,v)}{D(x,y)}=\begin{vmatrix}
		-\dfrac{y^2}{x^2}&\dfrac{2y}{x}\\y&x
	\end{vmatrix}=-\dfrac{3y^2}{x}$,即$\dfrac{D(x,y)}{D(u,v)}=-\dfrac x{3y^2}=-\dfrac1{3u}$。由此有
	\begin{equation*}
		S=\iint_D\dif x\dif y=\iint_{D_1}\left|\frac{D(x,y)}{D(u,v)}\right|\dif u\dif v
		=\iint_{D_1}\frac1{3u}\dif u\dif v=\int_a^b\left(\int_p^q\frac1{3u}\dif u\right)\dif v
		=\frac13(b-a)\ln\frac qp \qedhere
	\end{equation*}
\end{solution}

\subsection{三重积分的计算}

\subsubsection{空间直角坐标系下的三重积分}

\begin{circum}
	\textbf{XY-Z区域}\quad
	$D_1\subset \mathbb{R}^2$为Jordan 可测集,$f_1,f_2\in\mathscr{C}(D_1)$ 使得$\forall (x,y)\in D_1$,均有$f_1(x,y)\le f_2(x,y)$. 令
	\begin{equation*}
		\varOmega_1=\big\{(x,y,z)\ |\ f_1(x,y)\le z\le f_2(x,y),(x,y)\in D_1\big\}
	\end{equation*}
	则$\varOmega_1$为Jordan可测集且$\forall f\in\mathscr{C}(\varOmega_1)$,均有
	\begin{equation*}
		\iiint_{\varOmega_1} f(x,y,z)\dif x\dif y\dif z
		=\iint_{D_1}\mbome{\bigg(\int_{f_1(x,y)}^{f_2(x,y)} f(x,y,z)\dif z\bigg)}\dif x\dif y.
	\end{equation*}
\end{circum}
\begin{circum}
	\textbf{XZ-Y区域}\quad
	$D_2\subset \mathbb{R}^2$
	为Jordan 可测集,$g_1,g_2\in\mathscr{C}(D_2)$ 使得$\forall (x,z)\in D_2$,均有$g_1(x,z)\le g_2(x,z)$. 令
	\begin{eqnarray*}
		\varOmega_2=\big\{(x,y,z)\ |\ g_1(x,z)\le y\le g_2(x,z),(x,z)\in D_2\big\}
	\end{eqnarray*}
	则$\varOmega_2$为  Jordan 可测集且$\forall f\in\mathscr{C}(\varOmega_2)$,均有
	\begin{equation*}
		\iiint_{\varOmega_2} f(x,y,z)\dif x\dif y\dif z
		=\iint_{D_2}\mbome{\bigg(\int_{g_1(x,z)}^{g_2(x,z)} f(x,y,z)\dif y\bigg)}\dif x\dif z
	\end{equation*}
\end{circum}
\begin{circum}
	\textbf{YZ-X区域}\quad
	$D_3\subset \mathbb{R}^2$
	为Jordan 可测集,$h_1,h_2\in\mathscr{C}(D_3)$
	使得$\forall (y,z)\in D_3$,均有$h_1(y,z)\le h_2(y,z)$. 令
	\begin{equation*}
		\varOmega_3=\big\{(x,y,z)\ |\ h_1(y,z)\le x \le f_2(y,z),\ (y,z)\in D\big\}.
	\end{equation*}
	则$\varOmega_3$为Jordan可测集且$\forall f\in\mathscr{C}(\varOmega_3)$,均有
	\begin{equation*}
		\iiint_{\varOmega_3} f(x,y,z)\dif x\dif y\dif z
		=\iint_{D_3}\mbome{\bigg(\int_{h_1(y,z)}^{h_2(y,z)} f(x,y,z)\dif x\bigg)}\dif y\dif z
	\end{equation*}
\end{circum}

\begin{example}
	计算$I=\iiint_{\varOmega}(y+z)\dif x\dif y\dif z$,其中$\varOmega=\bigg\{(x,y,z)\ \bigg|\ \dfrac{x^2}{a^2}+\dfrac{y^2}{b^2}+\dfrac{z^2}{c^2}\le 1,\ z\ge 0\bigg\}$。
\begin{solution}
	知$\varOmega=\bigg\{(x,y,z)\ \bigg|\ \dfrac{x^2}{a^2}+\dfrac{y^2}{b^2}+\dfrac{z^2}{c^2}\le 1,\ z\ge 0\bigg\}=\bigg\{(x,y,z)\ \bigg|\ 0 \le z \le c\sqrt{1-\dfrac{x^2}{a^2}-\dfrac{y^2}{b^2}},\dfrac{x^2}{a^2}+\dfrac{y^2}{b^2} \le 1\bigg\}$。由线性性与对称性可知
	\begin{align*}
	I&=\iiint_{\varOmega}y\dif x\dif y\dif z+\iiint_{\varOmega}z\dif x\dif y\dif z
	=\iiint_{\varOmega}z\dif x\dif y\dif z
	=\iint_{\frac{x^2}{a^2}+\frac{y^2}{b^2}\le 1}
	\left(\int_0^{c\sqrt{1-\frac{x^2}{a^2}-\frac{y^2}{b^2}}}z\dif z\right)\dif x\dif y\\
	&=\iint_{\frac{x^2}{a^2}+\frac{y^2}{b^2}\le 1}\frac{c^2}{2}\left(1-\frac{x^2}{a^2}-\frac{y^2}{b^2}\right)\dif x\dif y
	=4\iint_{\frac{x^2}{a^2}+\frac{y^2}{b^2}\le 1\atop x,y\ge 0}\frac{c^2}{2}\left(1-\frac{x^2}{a^2}-\frac{y^2}{b^2}\right)\dif x\dif y\\
	&=2c^2\iint_{\frac{x^2}{a^2}+\frac{y^2}{b^2}\le 1\atop x,y\ge 0}\left(1-\frac{x^2}{a^2}-\frac{y^2}{b^2}\right)\dif x\dif y
	\xlongequal[y=b\rho\sin \varphi]{x=a\rho\cos \varphi}
	2c^2\int_0^{\frac{\pi}{2}}\left(\int_0^1(1-\rho^2)ab\rho\dif \rho\right)\dif \varphi\\
	&=2abc^2\cdot \frac{\pi}{2}\cdot \left(\frac{\rho^2}{2}-\frac{\rho^4}{4}\right)\Big|_0^1
	=\frac{\pi}{4}abc^2 \qedhere
	\end{align*}
\end{solution}
\end{example}

\subsubsection{柱坐标系下的三重积分}

柱坐标$(\rho,\varphi,z)$和直角坐标$(x,y,z)$之间有映射关系$\v{f}:\begin{cases}
	x=\rho\cos\varphi,\\[-3pt] y=\rho\sin\varphi,\\[-3pt] z=z,
\end{cases}$~~其Jacobi行列式为
$\dfrac{D(x,y,z)}{D(\rho,\varphi,z)}=\begin{vmatrix}
	\cos\varphi & -\rho\sin\varphi & 0 \\[-3pt] \sin\varphi & \rho\cos\varphi & 0 \\[-3pt] 0 & 0 & 1
\end{vmatrix}=\rho$,于是由定理~\ref{变量替换定理}~中变量替换定理,即有
\begin{equation*}
	\iiint_{\v{f}(D)} F(x,y,z)\dif x\dif y\dif z =\iiint_{D} F(\rho\cos\varphi,\rho\sin\varphi,z)\rho \dif \rho\dif \varphi\dif z
\end{equation*}

若取广义柱坐标系$\v{g}:\begin{cases}
	x=a\rho\cos\varphi,\\[-3pt] y=b\rho\sin\varphi,\\[-3pt] z=z,
\end{cases}$~~其Jacobi行列式为
$\dfrac{D(x,y,z)}{D(\rho,\varphi,z)}=\begin{vmatrix}
	a\cos\varphi & -a\rho\sin\varphi & 0 \\[-3pt] b\sin\varphi & b\rho\cos\varphi & 0 \\[-3pt] 0 & 0 & 1
\end{vmatrix}=ab\rho$,于是由定理~\ref{变量替换定理}~中变量替换定理,即有
\begin{equation*}
	\iiint_{\v{g}(D)} F(x,y,z)\dif x\dif y\dif z =\iiint_{D} F(\rho\cos\varphi,\rho\sin\varphi,z)ab\rho \dif \rho\dif \varphi\dif z
\end{equation*}

\begin{example}[2]
	求积分$\displaystyle\iiint_{\varOmega}(x^2+y^2)\dif x\dif y\dif z$,其中立体$\varOmega$为平面曲线$\begin{cases}
		y^2=2z,\\[-3pt]
		x=0,
	\end{cases}$\quad 绕$z$轴旋转一周形成的旋转面与平面$z=8$所围成的空间区域。
\begin{solution}
	在柱坐标系下
	$\varOmega_1=\bigg\{(\rho,\varphi,z)\ |\ 0\le \varphi\le 2\pi,\ 0\le \rho\le 4,\ \dfrac{1}{2}\rho^2\le z\le 8\bigg\}$,
	由此立刻可得
	\begin{equation*}
	\iint_{\varOmega}(x^2+y^2)\dif x\dif y\dif z
	=\int_0^{2\pi}\left(\int_0^4\left(\int_{\frac{1}{2}\rho^2}^8\rho^3\dif z\right)\dif \rho\right)\dif \varphi
	=\frac{1024}{3}\pi \qedhere
	\end{equation*}
\end{solution}
\end{example}

\begin{example}
	计算$\displaystyle\iiint_{\varOmega}x^2\dif x\dif y\dif z$,其中
	$\varOmega=\big\{(x,y,z)\ |\ \sqrt{x^2+y^2}\le z\le \sqrt{R^2-x^2-y^2}\big\}$。
\begin{solution}
	在柱坐标系下$\varOmega$变为
	$\varOmega_1=\bigg\{(\rho,\varphi,z)\ |\ \rho\le z\le \sqrt{R^2-\rho^2},\ 0\le \rho\le \dfrac{\sqrt{2}R}{2},\ 0\le \varphi\le 2\pi\bigg\}$,
	由此立刻可得
	\begin{align*}
	\iiint_{\varOmega} x^2\dif x\dif y\dif z
	&=\iiint_{\varOmega_1} (\rho\cos \varphi)^2\rho\dif \rho\dif \varphi\dif z
	=\int_0^{2\pi}\left(\int_0^{\frac{\sqrt{2}R}{2}}\left(
	\int_{\rho}^{\sqrt{R^2-\rho^2}}\rho^3\cos^2\varphi\dif z\right)\dif \rho\right)\dif \varphi\\
	&=\pi\int_0^{\frac{\sqrt{2}R}{2}}\rho^3(\sqrt{R^2-\rho^2}-\rho)\dif \rho
	=\frac{\pi R^5}{5}\left(\frac{2}{3}-\frac{5\sqrt{2}}{12}\right) \qedhere
	\end{align*}
\end{solution}
\end{example}

\subsubsection{球坐标系下的三重积分}

球坐标$(r,\theta,\varphi)$和直角坐标$(x,y,z)$之间有映射关系$\v{f}:\begin{cases}
	x=r\sin\theta\cos\varphi,\\[-3pt] y=r\sin\theta\sin\varphi,\\[-3pt] z=r\cos\theta,
\end{cases}$~~其Jacobi行列式为
$\dfrac{D(x,y,z)}{D(r,\theta,\varphi)}=\begin{vmatrix}
	\sin\theta\cos\varphi & r\cos\theta\cos\varphi & -r\sin\theta\sin\varphi \\[-3pt] \sin\theta\sin\varphi & r\cos\theta\sin\varphi & r\sin\theta\cos\varphi \\[-3pt] \cos\theta & -r\sin\theta & 0
\end{vmatrix}=r^2\sin\theta$,于是由定理~\ref{变量替换定理}~中变量替换定理,即有
\begin{equation*}
	\iiint_{\v{f}(D)} F(x,y,z)\dif x\dif y\dif z =\iiint_{D} F(r\sin\theta\cos\varphi,r\sin\theta\sin\varphi,r\cos\theta)r^2\sin\theta \dif r\dif \theta\dif \varphi
\end{equation*}
其中$r \ge 0,0 \le \theta \le \pi,0 \le \varphi < 2\pi$。

\subsection{重积分的物理与几何应用}

\subsubsection{物体的重心(质心)与形心问题}

\de[重心(质心)、形心]{
	假设$\varOmega\subset \mathbb{R}^n$ 中分布有质量,在点$X$处的密度为$\rho(X)$,则其\tboba{重心}~$\overline{X} = (\bar{x}_1,\cdots,\bar{x}_n)$为 
	$$ \bar{x}_j = \dfrac{\displaystyle \idotsint_\varOmega x_j\rho(x_1,\cdots,x_n) \dif x_1 \cdots \dif x_n}{\displaystyle \idotsint_\varOmega \rho(x_1,\cdots,x_n) \dif x_1 \cdots \dif x_n} $$
	当$\rho \equiv 1$时,将质心称为\tboba{形心}。
}

\begin{example}
	设曲面$S$在球坐标系下的方程为
	$r=a(1+\cos \theta)$($a>0$),
	令$\varOmega$ 为曲面$S$ 所围成的有界区域,求$\varOmega$在直角坐标系下的形心。 
\begin{solution}
	\adjline
	在球坐标系下$\varOmega$变为
	$\varOmega_1=\big\{(r,\theta,\varphi)\ |\ 0\le r\le a(1+\cos \theta),
	\ 0\le \theta\le \pi,\ 0\le \varphi\le 2\pi\big\}$,
	由此可得$\varOmega$的体积为
	\begin{equation*}
		|\varOmega|=\int_0^{2\pi}\left(\int_0^{\pi}\left(\int_0^{a(1+\cos\theta)}r^2\sin\theta\dif r\right)\dif \theta\right)\dif \varphi
		=\frac{8}{3}\pi a^3
	\end{equation*}
	设所求质心为$(\bar{x},\bar{y},\bar{z})$,则由对称性可知
	\begin{align*}
	\bar{x}&=\frac{1}{|\varOmega|}\iiint_{\varOmega}x\dif x\dif y\dif z=0,\\
	\bar{y}&=\frac{1}{|\varOmega|}\iiint_{\varOmega}y\dif x\dif y\dif z=0,\\
	\bar{z}&=\frac{1}{|\varOmega|}\iiint_{\varOmega}z\dif x\dif y\dif z
	=\frac{1}{|\varOmega|}\int_0^{2\pi}\left(\int_0^{\pi}\left(\int_0^{a(1+\cos\theta)}
	(r\cos \theta)(r^2\sin\theta)\dif r\right)\dif \theta\right)\dif \varphi
	=\frac{4}{5}a
	\end{align*}
	故所求质心为$\left(0,0,\dfrac{4}{5}a\right)$。
\end{solution}
\end{example}

\subsubsection{空间曲面的面积问题}

\di[空间曲面的面积公式]{
	设空间曲面$\varSigma$的参数方程为
	$\begin{cases}
		x=x(u,v),\\[-3pt]
		y=y(u,v),\\[-3pt]
		z=z(u,v),
	\end{cases}$~~
	$(u,v)\in D$,其中$D\subset\mathbb{R}^2$ 为Jordan 可测集,
	而$x,y,z$ 为连续可导函数使得$\dfrac{\partial (x,y,z)}{\partial (u,v)}$的秩为$2$。定义
	$E=\left(\dfrac{\partial x}{\partial u}\right)^2+\left(\dfrac{\partial y}{\partial u}\right)^2+\left(\dfrac{\partial z}{\partial u}\right)^2$,
	$G=\left(\dfrac{\partial x}{\partial v}\right)^2+\left(\dfrac{\partial y}{\partial v}\right)^2+\left(\dfrac{\partial z}{\partial v}\right)^2$,
	$F=\dfrac{\partial x}{\partial u}\dfrac{\partial x}{\partial v}+\dfrac{\partial y}{\partial u}\dfrac{\partial y}{\partial v}
	+\dfrac{\partial z}{\partial u}\dfrac{\partial z}{\partial v}$,
	则曲面$\varSigma$的面积为
	$$S=\displaystyle\iint_{D}\sqrt{EG-F^2}\dif u\dif v$$
}
\begin{proofs}
	\adjline
	定义
	\begin{equation*}
		\v{T}_u=\begin{pmatrix}
			\dfrac{\partial x}{\partial u}\\[5pt]
			\dfrac{\partial y}{\partial u}\\[5pt]
			\dfrac{\partial z}{\partial u}
		\end{pmatrix},\quad
		\v{T}_v=\begin{pmatrix}
			\dfrac{\partial x}{\partial v}\\[5pt]
			\dfrac{\partial y}{\partial v}\\[5pt]
			\dfrac{\partial z}{\partial v}
		\end{pmatrix}
	\end{equation*}
	则我们有
	\begin{equation*}
		\v{T}_u\times \v{T}_v=\begin{pmatrix}
			\dfrac{\partial x}{\partial u}\\[5pt]
			\dfrac{\partial y}{\partial u}\\[5pt]
			\dfrac{\partial z}{\partial u}
		\end{pmatrix}\times\begin{pmatrix}
			\dfrac{\partial x}{\partial v}\\[5pt]
			\dfrac{\partial y}{\partial v}\\[5pt]
			\dfrac{\partial z}{\partial v}
		\end{pmatrix}=\begin{pmatrix}
			\dfrac{D(y,z)}{D(u,v)}\\[5pt]
			\dfrac{D(z,x)}{D(u,v)}\\[5pt]
			\dfrac{D(x,y)}{D(u,v)}
		\end{pmatrix}
	\end{equation*}
	用以\(\dif v\)细分的\(u\)曲线和以\(\dif u\)细分的\(v\)曲线把曲面\( \varSigma \)分成小块,每个小块近似看作平行四边形,其面积就是\( \dif\sigma = \left\| \v{T}_u \dif u \times \v{T}_v \dif v \right\| = \left\| \v{T}_u \times \v{T}_v \right\| \dif u \dif v \),而
	\begin{align*}
		\left\|\v{T}_u\times\v{T}_v\right\|^2
		&=	\left(\dfrac{D(y,z)}{D(u,v)}\right)^2+\left(\dfrac{D(z,x)}{D(u,v)}\right)^2
			+\left(\dfrac{D(x,y)}{D(u,v)}\right)^2\\
		&=	\left(\dfrac{\partial y}{\partial u}\dfrac{\partial z}{\partial v}-\dfrac{\partial y}{\partial v}\dfrac{\partial z}{\partial u}\right)^2
			+\left(\dfrac{\partial z}{\partial u}\dfrac{\partial x}{\partial v}-\dfrac{\partial z}{\partial v}\dfrac{\partial x}{\partial u}\right)^2
			+\left(\dfrac{\partial x}{\partial u}\dfrac{\partial y}{\partial v}-\dfrac{\partial x}{\partial v}\dfrac{\partial y}{\partial u}\right)^2 \displaybreak\\
		&=	\left(\dfrac{\partial y}{\partial u}\right)^2\left(\dfrac{\partial z}{\partial v}\right)^2
			+\left(\dfrac{\partial y}{\partial v}\right)^2\left(\dfrac{\partial z}{\partial u}\right)^2
			+\left(\dfrac{\partial z}{\partial u}\right)^2\left(\dfrac{\partial x}{\partial v}\right)^2\\
			&\quad\,+\left(\dfrac{\partial z}{\partial v}\right)^2\left(\dfrac{\partial x}{\partial u}\right)^2
			+\left(\dfrac{\partial x}{\partial u}\right)^2\left(\dfrac{\partial y}{\partial v}\right)^2
			+\left(\dfrac{\partial x}{\partial v}\right)^2\left(\dfrac{\partial y}{\partial u}\right)^2\\
			&\quad\,-2\dfrac{\partial y}{\partial u}\dfrac{\partial z}{\partial v}\dfrac{\partial y}{\partial v}\dfrac{\partial z}{\partial u}
			-2\dfrac{\partial z}{\partial u}\dfrac{\partial x}{\partial v}\dfrac{\partial z}{\partial v}\dfrac{\partial x}{\partial u}
			-2\dfrac{\partial x}{\partial u}\dfrac{\partial y}{\partial v}\dfrac{\partial x}{\partial v}\dfrac{\partial y}{\partial u}\\
		&=	\left(\left(\dfrac{\partial x}{\partial u}\right)^2+\left(\dfrac{\partial y}{\partial u}\right)^2+\left(\dfrac{\partial z}{\partial u}\right)^2\right)
			\cdot\left(\left(\dfrac{\partial x}{\partial v}\right)^2+\left(\dfrac{\partial y}{\partial v}\right)^2+\left(\dfrac{\partial z}{\partial v}\right)^2\right)\\
			&\quad\,-\left(\dfrac{\partial x}{\partial u}\dfrac{\partial x}{\partial v}+\dfrac{\partial y}{\partial u}\dfrac{\partial y}{\partial v}
			+\dfrac{\partial z}{\partial u}\dfrac{\partial z}{\partial v}\right)^2\\
		&=	\left\|\v{T}_u\right\|^2\left\|\v{T}_v\right\|^2-\left\|\v{T}_u\cdot \v{T}_v\right\|^2
	\end{align*}
	定义
		$E=\left(\dfrac{\partial x}{\partial u}\right)^2+\left(\dfrac{\partial y}{\partial u}\right)^2+\left(\dfrac{\partial z}{\partial u}\right)^2$,
		$G=\left(\dfrac{\partial x}{\partial v}\right)^2+\left(\dfrac{\partial y}{\partial v}\right)^2+\left(\dfrac{\partial z}{\partial v}\right)^2$,
		$F=\dfrac{\partial x}{\partial u}\dfrac{\partial x}{\partial v}+\dfrac{\partial y}{\partial u}\dfrac{\partial y}{\partial v}
		+\dfrac{\partial z}{\partial u}\dfrac{\partial z}{\partial v}$,
	则面积微元为
	$\dif \sigma=\left\|\v{T}_u\times \v{T}_v\right\|\dif u\dif v=\sqrt{EG-F^2}\dif u\dif v$,
	故曲面的面积为$S=\displaystyle\iint_{D}\sqrt{EG-F^2}\dif u\dif v$。	
\end{proofs}

\subsection{曲线曲面积分}

\subsubsection{第一类曲线积分}

\de[第一类曲线积分]{
	假设$L\subset \mathbb{R}^3$为空间曲线,其起点为$A$,
	终点为$B$,而$F : L\rightarrow\mathbb{R}$ 为函数,对任意的整数
	$n\ge 1$,将$L$分割成$\wideparen{P_0P_1},\ \wideparen{P_1P_2},\ \cdots,\ \wideparen{P_{n-1}P_n}$共$n$段,
	其中$P_0=A$,$P_n=B$。
	在每个小段$\wideparen{P_{i-1}P_i}$上取点$P_i^\ast$,令
	\begin{equation*}
	d=\max_{1\le i\le n}|\wideparen{P_{i-1}P_i}|
	\end{equation*}
	并称之为分割的\tboba{步长},定义(若极限存在)
	\begin{equation*}
	\int_{L}F(x,y,z) \dif s=\lim_{d\rightarrow 0}\sum_{i=1}^nF(P_i^\ast)|\wideparen{P_{i-1}P_i}|
	\end{equation*}
	并且称之为$F$在曲线$L$上的\tboba{第一类曲线积分},
	也记之为$\displaystyle\int_{\wideparen{AB}} F(x,y,z) \dif s$,
	其中称$L$ 为\tboba{积分路径},$F$为\tboba{被积函数},$F(x,y,z) \dif s$为\tboba{被积分式},$ \dif s$为\tboba{曲线元素}或\tboba{弧微分}或\tboba{弧微元}。
}
上述极限存在,意味着$\exists a\in\mathbb{R}$,使得$\forall \varepsilon>0$,$\exists \delta>0$使得
\begin{center}
	\vspace{-0.5em}
	当$d<\delta$时,均有
	$\left|\sum\limits_{i=1}^nF(P_i^\ast)|\wideparen{P_{i-1}P_i}|-a\right|<\varepsilon$
	\vspace{-0.5em}
\end{center}
此时将$a$记作$\displaystyle\int_{L}F(x,y,z)\dif s$。若$L$为分段光滑曲线(也即$L$可分成有限多段,且每一段均有连续可导的参数表示),而$F$为连续函数,则$\displaystyle\int_{L}F(x,y,z)\dif s$存在。

\di[第一类曲线积分与曲线的方向无关]{
	函数$f$沿曲线$\wideparen{AB}$和$\wideparen{BA}$的积分相等:
	\begin{equation*}
		\int_{\wideparen{AB}}F(x,y,z)\dif s=\int_{\wideparen{BA}}F(x,y,z)\dif s
	\end{equation*}
}

\begin{definition}
	假设$L\subset\mathbb{R}^3$ 为分段光滑曲线,在它上面分布有质量使得在点$X\in L$处的密度为$\rho(X)$。若$\rho$连续,则$L$的总质量
	$M=\displaystyle\int_{L}\rho(x,y,z)\dif s$,
	$L$的\textbf{质心}~$(\bar{x},\bar{y},\bar{z})$为
	\begin{equation*}
		\bar{x}=\frac{1}{M}\int_{L}x\rho(x,y,z)\dif s,\quad
		\bar{y}=\frac{1}{M}\int_{L}y\rho(x,y,z)\dif s,\quad
		\bar{z}=\frac{1}{M}\int_{L}z\rho(x,y,z)\dif s	
	\end{equation*}
\end{definition}

\begin{circum}
	\textbf{由参数方程组给定的曲线}
	
	设分段光滑曲线$L$的参数方程为
	$\begin{cases}
		x=x(t),\\[-3pt]
		y=y(t),\\[-3pt]
		z=z(t)
	\end{cases}$~~($t\in[\alpha,\beta]$),
	则其弧微分为
	$$\dif s=\sqrt{(x'(t))^2+(y'(t))^2+(z'(t))^2}\dif t$$
	从而我们有
	\begin{equation}
		\int_{L}F(x,y,z)\dif s
		=\int_{\alpha}^{\beta}F(x(t),y(t),z(t))
		\sqrt{(x'(t))^2+(y'(t))^2+(z'(t))^2}\dif t
	\end{equation}
\end{circum}
\begin{example}[3]
	求柱面$x^2+y^2=2ax$($a>0$)被平面$z=0$以及曲面$z=\sqrt{x^2+y^2}$所截部分的面积。
\begin{solution}
	设$L$为圆周$x^2+y^2=2ax$,其参数方程为
	$\begin{cases}
		x=a+a\cos\varphi,\\[-3pt]
		y=a\sin \varphi
	\end{cases}$~($\varphi\in [0,2\pi]$),
	于是所求面积为
	\begin{equation*}
		S=\int_L\sqrt{x^2+y^2}\dif s
		=\int_0^{2\pi}\sqrt{(a+a\cos\varphi)^2+(a\sin \varphi)^2}\,a\dif \varphi
		=8a^2 \qedhere
	\end{equation*}
\end{solution}
\end{example}
\begin{circum}
	\textbf{由隐函数方程组给定的曲线}
	
	若曲线$L$由隐函数方程组
	$\begin{cases}
		\varphi_1(x,y,z)=0,\\[-3pt]
		\varphi_2(x,y,z)=0
	\end{cases}$~~
	给出,利用隐函数定理来局部求解上述方程组,由此得到曲线$L$的分段的参数表示,随后再对每段分别利用前面的公式计算。
\end{circum}

\subsubsection{第一类曲面积分}

\de[第一类曲面积分]{
	假设$S\subset \mathbb{R}^3$ 为曲面,
	$f : S\rightarrow\mathbb{R}$ 为函数。
	将$S$ 分割成$n$块$S_1,\cdots,S_n$,在每块$S_j$上
	取一点$X_j$,记$d$ 为所有$S_j$的直径中的最大者,
	我们定义(若极限存在)
	\begin{equation*}
	\iint_{S}f(x,y,z)\dif \sigma
	=\lim_{d\rightarrow 0}\sum_{j=1}^nf(X_j)|S_j|
	\end{equation*}
	并 称 之 为 函 数$f$ 在  曲 面$S$ 上的\tboba{第一类曲面积分},
	$S$ 为\tboba{积 分 曲 面},$f(x,y,z)\dif \sigma$ 为 被 积 分 式,
	$\dif \sigma$ 则为\tboba{面积元素}或\tboba{面积微分}或\tboba{面积微元}。
}
上述极限存在,意味着$\exists a\in\mathbb{R}$,使得$\forall \varepsilon>0$,
$\exists \delta>0$使得
\begin{center}
	\vspace{-0.5em}
	当$d<\delta$时,均有$\left|\sum\limits_{j=1}^nf(X_j)|S_j|-a\right|<\varepsilon$
	\vspace{-0.5em}
\end{center}
此时将$a$记作$\displaystyle\iint_{S}f(x,y,z)\dif \sigma$。若$S$ 为 分 片 光 滑曲面(即$S$ 可分成有限多片,每一片均有连续可导的参数表示),
而$f$为连续函数,则$\displaystyle\iint_{S}f(x,y,z)\dif \sigma$存在。

设分片光滑曲面$S$的参数方程为
$\begin{cases}
    x=x(u,v),\\[-3pt]
    y=y(u,v),\\[-3pt]
    z=z(u,v),
\end{cases}$~~
$(u,v)\in D$,其中$D\subset\mathbb{R}^2$ 为Jordan 可测集,
则面积微元为$\dif \sigma=\sqrt{EG-F^2}\dif u\dif v$,其中
$E=\left(\dfrac{\partial x}{\partial u}\right)^2+\left(\dfrac{\partial y}{\partial u}\right)^2+\left(\dfrac{\partial z}{\partial u}\right)^2$,
$G=\left(\dfrac{\partial x}{\partial v}\right)^2+\left(\dfrac{\partial y}{\partial v}\right)^2+\left(\dfrac{\partial z}{\partial v}\right)^2$,
$F=\dfrac{\partial x}{\partial u}\dfrac{\partial x}{\partial v}+\dfrac{\partial y}{\partial u}\dfrac{\partial y}{\partial v}
+\dfrac{\partial z}{\partial u}\dfrac{\partial z}{\partial v}$,
于是 
\begin{equation}
	\iint_{S} f(x,y,z)\dif \sigma
	=\iint_{D} f(x(u,v) ,y(u,v) ,z(u,v))
	\sqrt{EG-F^2}\dif u\dif v
\end{equation}

\subsubsection{第二类曲线积分}

\de[第二类曲线积分]{
	设$L\subset \mathbb{R}^3$ 为空间曲线,它的起点为$A$,终点为$B$,而$\v{F}=(F_1,F_2,F_3):L\rightarrow\mathbb{R}^3$为向量值函数
	$\v{F}(x,y,z)=\left(F_1(x,y,z),F_2(x,y,z),F_3(x,y,z)\right),
	(x,y,z)\in L\subset\mathbb{R}^3$。

	对任意整数$n\ge 1$,我们将曲线$L$分割成$n$小段$\wideparen{P_0P_1},\ \wideparen{P_1P_2},\ \cdots,\ \wideparen{P_{n-1}P_n}$,其中
	$P_0=A$,$P_n=B$,并记
	$P_i=(x_i,y_i,z_i)\ (0\le i\le n),\ \v{\ell}=(x,y,z)$。在每个小段$\wideparen{P_{i-1}P_i}$上取点$X_i=P_i^\ast=(x_i^\ast,y_i^\ast,z_i^\ast)$。令
	\begin{equation*}
		d=\max_{1\le i\le n}|\overline{P_{i-1}P_i}|
	\end{equation*}
	并称之为分割的\tboba{步长}。
	定义(若极限存在)
	\begin{align*}
	\int_{L(\wideparen{AB})}\v{F}(\v{\ell})\cdot \dif \v{\ell}
	&=\lim_{d\rightarrow 0}\sum_{i=1}^n\v{F}(X_i)\cdot \overrightarrow{P_{i-1}P_i}\\[-3pt]
	&=\lim_{d\rightarrow 0}\sum_{i=1}^n\big( F_1(X_i)(x_i-x_{i-1})
	 + F_2(X_i)(y_i-y_{i-1}) + F_3(X_i)(z_i-z_{i-1}) \big)
	\end{align*}
	并且称之为向量值函数$\v{F}$沿曲线$L$由点$A$到点$B$的\tboba{第二类曲线积分}。	
}

\paragraph{第二类曲线积分的计算}  

\begin{circum}
	\textbf{由参数方程组给定的曲线}

	设分段光滑曲线$L$的参数方程为\(\begin{cases}
		x=x(t),\\[-3pt]
		y=y(t),\\[-3pt]
		z=z(t),
	\end{cases}\)~
	\(t\in[\alpha,\beta]\),
	其中起点$A$ 和  终点 $B$所对应的参数分别为$\alpha,\beta$,则
	\begin{align}
		\int_{L(\wideparen{AB})} \v{F}(\v{\ell})\cdot \dif \v{\ell}
		&=\int_{L(\wideparen{AB})} F_1(x,y,z)\dif x
		+ \int_{L(\wideparen{AB})} F_2(x,y,z)\dif y
		+ \int_{L(\wideparen{AB})} F_3(x,y,z)\dif z \nonumber\\
		&=\int_{\alpha}^{\beta}F_1\left(x(t),y(t),z(t)\right)x'(t)\dif t
		+\int_{\alpha}^{\beta}F_2\left(x(t),y(t),z(t)\right)y'(t)\dif t \nonumber\\
		&\quad\,+\int_{\alpha}^{\beta}F_3\left(x(t),y(t),z(t)\right)z'(t)\dif t	
	\end{align}
\end{circum}

设路径$L\subset\mathbb{R}^3$是起点为$A$,终点为$B$的分段光滑曲线,其参数方程为\(\v{\ell}(t)=\left(x(t),y(t),z(t)\right),t\in [a,b]\),而$\v{F}=(F_1,F_2,F_3): L\rightarrow \mathbb{R}^3$为分段连续函数。
$\forall P\in L$,设$L$在点$P$处的单位切向量为\(\v{\tau}^{0}(P)=\left(\cos \alpha(P),\cos \beta(P),\cos \gamma(P)\right)\)。而$\forall t\in [a,b]$,我们有
\begin{equation*}
	\v{\tau}^{0}(\v{\ell}(t))=\frac{\v{\ell}'(t)}{\|\v{\ell}'(t)\|}
	=\frac{\left(x'(t),y'(t),z'(t)\right)}{\sqrt{(x'(t))^2+(y'(t))^2+(z'(t))^2}}
\end{equation*}
由此立刻可得
\begin{align*}
	\cos\alpha\left(\v{\ell}(t)\right)&=\frac{x'(t)}{\sqrt{(x'(t))^2+(y'(t))^2+(z'(t))^2}},\\
	\cos\beta\left(\v{\ell}(t)\right)&=\frac{y'(t)}{\sqrt{(x'(t))^2+(y'(t))^2+(z'(t))^2}},\\
	\cos\gamma\left(\v{\ell}(t)\right)&=\frac{z'(t)}{\sqrt{(x'(t))^2+(y'(t))^2+(z'(t))^2}}	
\end{align*}
注意到$\dif \ell =\sqrt{(x'(t))^2+(y'(t))^2+(z'(t))^2}\dif t$,即写出
\begin{equation*}
\dif x =x'(t)\dif t=\cos\alpha \dif \ell,\quad \dif y =y'(t)\dif t=\cos\beta\dif \ell,\quad \dif z =z'(t)\dif t=\cos\gamma\dif \ell
\end{equation*}
进而我们就有
\begin{align}
\int_{\L(\wideparen{AB})}\v{F}(\v{\ell})\cdot \dif \v{\ell}
&=\int_{L(\wideparen{AB})}F_1(\v{\ell})\dif x+F_2(\v{\ell})\dif y+F_3(\v{\ell})\dif z\\
&=\int_{a}^{b}\left(F_1(\v{\ell}(t))x'(t)+
F_2(\v{\ell}(t))y'(t) + F_3(\v{\ell}(t))z'(t)\right)\dif t \nonumber\\
&=\int_L\left(F_1(x,y,z)\cos\alpha + F_2(x,y,z)\cos\beta
+ F_3(x,y,z)\cos \gamma\right)\dif \ell \nonumber\\
&=\int_L(\v{F}\cdot\v{\tau}^{0})(x,y,z)\dif \ell \label{equ: 11.5}
\end{align}
其中,式~\ref{equ: 11.5}~已是第一类曲面积分的形式。
\paragraph{Green公式}  

\begin{definition}
	称$D \subset\mathbb{R}^2$为单连通集,若$D$中的任意
	闭曲线所围的区域仍包含在$D$中(也即$D$中的
	任意 闭 曲 线可连续地收缩成为一点)。  若$D$不为
	单连通集,则称之为复连通集。 
\end{definition}

\di[单连通平面向量场的Green公式]{
	假设$\varOmega\subset\mathbb{R}^2$为单连通的
	有界闭区域,它的边界$\partial \varOmega$为分段光滑闭曲线,该曲线的正方向为逆时针方向,记$\v{n}_{0}$为$\partial \varOmega$的单位外法向量。 如果$\v{F}= (F_1,F_2)\tp:\varOmega\rightarrow \mathbb{R}^2$为
	连续可导的向量值函数,则
	\begin{equation}
		\oint_{\partial \varOmega}\v{F}\cdot \v{n}_{0}\dif \ell
		=\iint_{\varOmega}\left(\dfrac{\partial F_1}{\partial x}(x,y)+\dfrac{\partial F_2}{\partial y}(x,y)\right)\dif x\dif y
	\end{equation}
}
\begin{corollary}
	\(\displaystyle|\varOmega|
	=\iint_{\varOmega}1\dif x\dif y
	=\oint_{\partial \varOmega^{+}}x\dif y
	=-\oint_{\partial \varOmega^{+}}y\dif x
	=\frac{1}{2}\oint_{\partial \varOmega^{+}}x\dif y-y\dif x\)。
\end{corollary}

$\forall P\in\partial \varOmega$,假设$\v{\tau}_{0}(P)=(\cos \alpha,\sin \alpha)\tp$为$\partial \varOmega$在点$P$处的单位切向量,则我们有\(\v{n}_{0}(P)=(\sin \alpha,-\cos \alpha)\tp\)。
又$\dif x=\cos \alpha \dif \ell$,$\dif y=\sin \alpha\dif \ell$,于是我们有
\begin{equation*}
\v{F}\cdot \v{n}^{0}\dif \ell=(F_1\sin \alpha-F_2\cos \alpha)\dif \ell
=F_1\dif y-F_2\dif x
\end{equation*}
从而 Green公式又可以表述成
\begin{equation*}
	\oint_{\partial \varOmega^{+}}F_1\dif y-F_2\dif x
	=\iint_{\varOmega}\left(\frac{\partial F_1}{\partial x}+\frac{\partial F_2}{\partial y}\right)\dif x\dif y
\end{equation*}
若将$F_2$换成$-F_1$,$F_1$换成$F_2$,则对\(\v{F} = (F_2, -F_1)\tp\)有
\begin{equation*}
	\oint_{\partial \varOmega}\v{F}\cdot \dif \v{\ell}
	=\oint_{\partial \varOmega^{+}}F_1\dif x+F_2\dif y
	=\iint_{\varOmega}\left(\frac{\partial F_2}{\partial x}-\frac{\partial F_1}{\partial y}\right)\dif x\dif y
\end{equation*}

\di[复连通平面向量场的Green公式]{
	如果 $\mathbb{R}^2$ 上的闭区域$\varOmega$ 是由有限条分段光滑的曲线围成的, 假设 $P,Q:  D \to \mathbb{R}$ 是连续的函数并且具有连续的偏导数,那么
	\begin{equation}
	\int_{\partial D} Pdx+ Q dy=  \iint_{D} \left(\frac{\partial Q}{\partial x}-\frac{\partial P}{\partial y}\right)dxdy
	\end{equation}
	其中$\partial D$ 是$ D $ 的边界的正向按照「左侧」原则定义:当一个人沿着 $\partial D $ 正向前行时,区域$D$总是在这个人的左手边。
}

\begin{example}
	计算$\displaystyle\int_{L_1^{+}}(1+y\e^x)\dif x+(x+\e^x)\dif y$,其中$L_1$沿$\dfrac{x^2}{a^2}+\dfrac{y^2}{b^2}=1$的上半周由$A(a,0)$到$B(-a,0)$。
\begin{solution}
	\adjline
	设$L^{+}=L_1^{+}\cup\overrightarrow{BA}$,并且将$L$所围成的区域
	记作$\varOmega$。 则由Green公式可知
	\begin{align*}
	\oint_{L^{+}}(1+y\e^x)\dif x+(x+\e^x)\dif y
	&=\iint_{\varOmega}\left(-\frac{\partial (1+y\e^x)}{\partial y}+\frac{\partial(x+\e^x)}{\partial x}\right)\dif x\dif y\\
	&=\iint_{\varOmega}(-\e^x+1+\e^x)\dif x\dif y
	=\iint_{\varOmega}1\dif x\dif y
	=\frac{\pi}{2}ab
	\end{align*}
	另一方面,我们也有
	\begin{equation*}
	\int_{\overrightarrow{BA}}(1+ye^x)\dif x+(x+e^x)\dif y
	=\int_{-a}^a1\dif x
	=2a
	\end{equation*}
	由此可得
	\begin{equation*}
	\int_{L_1^{+}}(1+y\e^x)\dif x+(x+\e^x)\dif y
	=\oint_{L^{+}}(1+y\e^x)\dif x+(x+\e^x)\dif y
	-\int_{\overrightarrow{BA}}(1+y\e^x)\dif x+(x+\e^x)\dif y
	=\frac{\pi}{2}ab-2a \qedhere
	\end{equation*}
\end{solution}
\end{example}

\begin{example}[1]
	计算$\displaystyle\int_{L}\frac{(x+y)\dif y+(x-y)\dif x}{x^2+y^2}$,其中$L$是: $x^{\frac{2}{3}}+y^{\frac{2}{3}}=1$,逆时针方向。\label{e.g: 与路径无关引例}
\begin{solution}
	假设曲线$L: x^{\frac{2}{3}}+y^{\frac{2}{3}}=1$ 所围的区域为$\varOmega$,那么原点为$\varOmega$的内点,但被积函数及\(P,Q\)在原点不连续,不能在$\varOmega$上使用Green公式。但是,存在$\delta>0$使得$\varOmega$包含$L_\delta: x^2+y^2=\delta^2$,可令$\varOmega_{\delta}$是以$L\cup L_{\delta}$为边界的区域,其中$L$沿逆时针方向,而$L_{\delta}$沿
	顺时针方向。

	则由Green公式可知
	\begin{align*}
	\oint_{L\cup L_{\delta}}\frac{(x+y)\dif y+(x-y)\dif x}{x^2+y^2}
	&=\iint_{\varOmega_{\delta}}\left(\frac{\partial}{\partial x}\frac{x+y}{x^2+y^2}-\frac{\partial}{\partial y}\frac{x-y}{x^2+y^2}\right)\dif x \dif y \\
	&=\iint_{\varOmega_{\delta}}\left(\frac{(x^2+y^2)-(x+y)(2x)}{(x^2+y^2)^2}-\frac{-(x^2+y^2)-(x-y)(2y)}{(x^2+y^2)^2}\right)\dif x \dif y \\
	&=\iint_{\varOmega_{\delta}}\left(\frac{y^2-x^2-2xy}{(x^2+y^2)^2}-\frac{y^2-x^2-2xy}{(x^2+y^2)^2}\right)\dif x \dif y
	=0
	\end{align*}
	由此可得
	\begin{align*}
	\oint_{L}\frac{(x+y)\dif y+(x-y)\dif x}{x^2+y^2}
	&=-\oint_{L_{\delta}}\frac{(x+y)\dif y+(x-y)\dif x}{x^2+y^2} \\
	&=-\left(-\int_0^{2\pi}\frac{(\delta\cos \varphi+\delta\sin \varphi)\dif (\delta\sin \varphi)}{\delta^2}
	+\frac{(\delta\cos \varphi-\delta\sin \varphi)\dif (\delta\cos \varphi)}{\delta^2}\right) \\
	&=\int_0^{2\pi}\left((\cos \varphi+\sin \varphi)\cos \varphi-(\cos\varphi-\sin \varphi)\sin \varphi\right)\dif \varphi \\
	&=\int_0^{2\pi}(\cos^2\varphi+\sin^2\varphi)\dif \varphi=2\pi \qedhere
	\end{align*}
\end{solution}
\end{example}
例~\ref{e.g: 与路径无关引例}~中,实际上不论\(L\)取何包围原点的闭合曲线,都有$\displaystyle\int_{L}\frac{(x+y)\dif y+(x-y)\dif x}{x^2+y^2}=2\pi$。

\paragraph{平面上第二类曲线积分与路径的无关性} 

\di[第二类曲线积分仅依赖始末点的充要条件]{
	假设$\varOmega\subset\mathbb{R}^2$ 为非空开集,
	$\v{F}= (F_1,F_2)^T$
	在$\varOmega$ 上连续可导,而$A,B\in \varOmega$ 为两个固定点,
	$L\subset \varOmega$为连接$A,B$的分段光滑曲线。那么,
	\(\displaystyle \int_{L(\wideparen{AB})}\v{F}\cdot \dif \v{\ell}\)
	仅依赖$A,B$而与路径$L$无关,\tbome{当且仅当}对于$\varOmega$中
	过$A,B$的任意分段光滑闭曲线$\varGamma$, 均有
	\(\displaystyle \oint_{\varGamma^{+}}\v{F}\cdot \dif \v{\ell}=0\)。
}

\begin{example}
	曲线积分$\displaystyle\int_{L(\wideparen{AB})}\frac{x\dif x+y\dif y}{\sqrt{x^2+y^2-1}}$在复连通域
	$\varOmega=\mathbb{R}^2\setminus\bar{B}((0,0);1)$上是否与路径无关? 若是,求其从$A(2,0)$到点$B(0,3)$的积分值。
	\begin{solution}
		设$\varGamma$为$\varOmega$中过$A,B$的分段光滑闭曲线 且
		参数方程为$x=x(t)$,$y=y(t)$,$t\in [a,b]$.  则
		\begin{equation*}
		\oint_{\varGamma^{+}}\frac{x\dif x+y\dif y}{\sqrt{x^2+y^2-1}}
		=\int_a^b\frac{x(t)x'(t)+y(t)y'(t)}{\sqrt{(x(t))^2+(y(t))^2-1}}\dif t
		=\left(\sqrt{(x(t))^2+(y(t))^2-1}\right)\big|_a^b
		=0
		\end{equation*}
		因此题中的曲线积分在$\varOmega$上与路径无关。

		特别地,若$A= (2,0)$,$B=(0,3)$,并设$L=\overrightarrow{AB}$,
		则其方程为$y=3-\dfrac{3}{2}x\ (0\le x\le 2)$,于是
		\begin{align*}
			\int_{(AB)}\frac{x\dif x+y\dif y}{\sqrt{x^2+y^2-1}}
			&=\int_{L(\wideparen{AB})}\frac{x\dif x+y\dif y}{\sqrt{x^2+y^2-1}} 
			=-\int_0^2\frac{x-\frac{3}{2}(3-\frac{3}{2}x)}{\sqrt{x^2+(3-\frac{3}{2}x)^2-1}}\dif x \\
			&=\left.-\sqrt{x^2+\left(3-\frac{3}{2}x\right)^2-1}\right|_0^2
			=2\sqrt{2}-\sqrt{3} \qedhere
		\end{align*}		
	\end{solution}
\end{example}

\subsubsection{第二类曲面积分}

设$S\subset \mathbb{R}^3$为连通的光滑曲面,其参数方程为\(\v{r}=(x,y,z)\),\(\begin{cases}
	x=x(u,v),\\[-3pt]
	y=y(u,v),\\[-3pt]
	z=z(u,v),
\end{cases}\)~~\( (u,v)\in D\subset \mathbb{R}^2\),
其中$x(u,v),y(u,v),z(u,v)$连续可微,则曲面法向量
\begin{equation*}
	\v{n}_{\pm}=\pm \begin{pmatrix}
	\dfrac{\partial x}{\partial u}\\[7pt]
	\dfrac{\partial y}{\partial u}\\[7pt]
	\dfrac{\partial z}{\partial u}
	\end{pmatrix} \times \begin{pmatrix}
	\dfrac{\partial x}{\partial v}\\[7pt]
	\dfrac{\partial y}{\partial v}\\[7pt]
	\dfrac{\partial z}{\partial v}
	\end{pmatrix}
	=\pm \begin{pmatrix}
	\dfrac{D(y,z)}{D(u,v)}\\[7pt]
	\dfrac{D(z,x)}{D(u,v)}\\[7pt]
	\dfrac{D(x,y)}{D(u,v)}
	\end{pmatrix}
\end{equation*}
$\forall P\in S$,$\v{n}_{+}(P),\v{n}_{-}(P)$在该点处给出曲面$S$的
「两个」侧面。 
\begin{definition}
	固定$P_0\in S$ 并在该点处取定单位法方向$\v{n}(P_0)$(如$\v{n}_{+}^{0}(P_0)$)为正方向。  如果在任意点$P\in S$处
	可确定单位法方向$\v{n}^{0}(P)$使得$\v{n}^{0}$在连接$P_0$的
	任意的光滑曲线上连续,则称$S$为\textbf{可定向曲面},
	否则称为\textbf{不可定向曲面}。
\end{definition}
\begin{theorem}
	设$S\subset \mathbb{R}^3$为连通的光滑曲面,则$S$为
	可定向曲面 当且仅当 按上面定义取得的法向量$\v{n}$永不为零向量。 
	此时曲面$S$只有两个定向,分别为$\v{n}_{+}$和$\v{n}_{-}$。
\end{theorem}
\begin{definition}
	假设$S\subset \mathbb{R}^3$为连通
	的可定向光滑曲面,
	在$S$上给定一个\textbf{定向}并且将相应的单位法向量
	记作$\v{n}^{0}_S$,此时我们将$S$称为\textbf{定向曲面}。
\end{definition}

\de[第二类曲面积分]{
	假设$\varOmega\subset\mathbb{R}^3$为开集,$S\subset \varOmega$为可定向曲面(正侧为$S^{+}$),而$\v{F}=(P,Q,R): \varOmega\rightarrow\mathbb{R}^3$
	为函数。  将$S$分成$k$小块:$S_1,\cdots,S_k$。在$S_j$上
	取点$X_j$,并令\tboba{有向面积}~$\v{S}_j=\v{n}^{0}_S(X_j)|S_j|$。 
	记$d$为所有$S_j$的直径当中的最大者。定义(若极限存在)
	\begin{equation}
	\iint_{S^{+}}\v{F}(x,y,z)\cdot \dif\v{\sigma}
	=\lim_{d\rightarrow 0}\sum_{j=1}^k\v{F}(X_j)\cdot \v{S}_j
	\end{equation}
	称为$\v{F}$在定向曲面$S^{+}$上的\tboba{第二类曲面积分}。
}
上述极限存在,意味着$\exists a\in\mathbb{R}$, 使得$\forall \varepsilon>0$,$\exists \delta>0$ 使得
\begin{center}
	\vspace{-0.5em}
	当$d<\delta$时,均有$\left|\sum\limits_{j=1}^k\v{F}(X_j)\cdot \v{S}_j-a\right|<\varepsilon$
	\vspace{-0.5em}
\end{center}
此时将$a$记作$\displaystyle\iint_{S^{+}}\v{F}(x,y,z)\cdot \dif\v{\sigma}$。若$\v{F}$为分片连续,则$\displaystyle\iint_{S^{+}}\v{F}(x,y,z)\cdot \mathrm{d}\v{\sigma}$存在。

若$S$为封闭曲面,常将外侧取为正侧,并且将第二类曲面积分记作$\displaystyle\varoiint_{S^{+}}\v{F}(x,y,z)\cdot \mathrm{d}\v{\sigma}$。

\paragraph{第二类曲面积分的计算}

由定义可知
\begin{equation*}
\iint_{S^{+}}\v{F}(x,y,z)\cdot \dif \v{\sigma}
=\lim_{d\rightarrow 0}\sum_{j=1}^k\v{F}(X_j)\cdot \v{S}_j
=\lim_{d\rightarrow 0}\sum_{j=1}^k {\left(\v{F}(X_j)\cdot \v{n}^{0}_S(X_j)\right) } |S_j| 
=\iint_S(\v{F}\cdot \v{n}^{0}_S)(x,y,z) \dif \sigma
\end{equation*}
也即我们有
\begin{equation}
	\dif \v{\sigma}=\v{n}^{0}_S(x,y,z)\dif \sigma
\end{equation}
若记$\v{n}^{0}_S=(\cos \alpha,\cos \beta,\cos \gamma)$,那么 这里的$\alpha,\beta,\gamma$ 就是该向量和$x$轴,$y$轴,$z$轴正向的夹角,则
\begin{align*}
	\iint_{S^{+}}\v{F}(x,y,z)\cdot \dif \v{\sigma}
	&=\iint_S \left(\v{F}\cdot \v{n}^{0}_S\right)(x,y,z)\dif \sigma\\
	&=\iint_S\left(P(x,y,z)\cos \alpha+ Q (x,y,z)\cos \beta + R (x,y,z)\cos \gamma\right)\dif \sigma
\end{align*}
现定义
\begin{equation*}
\dif y\wedge \dif z=\cos \alpha\dif \sigma,\quad
\dif z\wedge \dif x=\cos \beta\dif \sigma,\quad
\dif x\wedge \dif y=\cos \gamma\dif \sigma
\end{equation*}
则我们有
\begin{equation*}
\iint_{S^{+}}\v{F}(x,y,z)\cdot \dif \v{\sigma}=
\iint_{S^{+}}\left(P (x,y,z)\dif y\wedge \dif z
+ Q (x,y,z)\dif z\wedge \dif x+ R (x,y,z)\dif x\wedge \dif y\right)
\end{equation*}

设$S\subset \mathbb{R}^3$为光滑曲面,其参数方程为\(\begin{cases}
	x=x(u,v),\\[-3pt]
	y=y(u,v),\\[-3pt]
	z=z(u,v),
\end{cases}\)~~\( (u,v)\in D\subset \mathbb{R}^2\),
其中$D$为Jordan可测,$x,y,z$为连续可微,且
\begin{equation*}
\v{n}=\begin{pmatrix}
	\dfrac{\partial x}{\partial u}\\[7pt]
	\dfrac{\partial y}{\partial u}\\[7pt]
	\dfrac{\partial z}{\partial u}
	\end{pmatrix} \times \begin{pmatrix}
	\dfrac{\partial x}{\partial v}\\[7pt]
	\dfrac{\partial y}{\partial v}\\[7pt]
	\dfrac{\partial z}{\partial v}
	\end{pmatrix}
	=\begin{pmatrix}
	\dfrac{D(y,z)}{D(u,v)}\\[7pt]
	\dfrac{D(z,x)}{D(u,v)}\\[7pt]
	\dfrac{D(x,y)}{D(u,v)}
	\end{pmatrix}\neq \v{0}
\end{equation*}
$\forall (u,v)\in D$,我们记\(
\v{r}(u,v)=\begin{cases}
	x(u,v),\\[-3pt]
	y(u,v),\\[-3pt]
	z(u,v),
\end{cases}\)~~则$\v{n}(u,v)=\v{r}'_u(u,v)\times \v{r}'_v(u,v)$,并且
\begin{equation}
\dif \sigma=\sqrt{EG-F^2}\dif u\dif v=\|\v{n}(u,v)\|\dif u\dif v
\end{equation}
于是$\dif \v{\sigma}= \v{n}^{0}_S(\v{r}(u,v))\dif \sigma=\pm\v{n}(u,v)\dif u\dif v$,其中
$\pm$在$\v{n}$与$S^{+}$同向时取正号,反向时取负号。
由此我们立刻可得
\begin{align*}
\iint_{S^{+}}\v{F}(x,y,z)\cdot \dif \v{\sigma}
&=\pm\iint_{D}\v{F}(\v{r}(u,v))\cdot \v{n}(u,v)\dif u\dif v\\
&=\pm\iint_{D}\bigg(P(x(u,v),y(u,v),z(u,v))\frac{D(y,z)}{D(u,v)}\\
&\qquad\qquad\;\;+Q(x(u,v),y(u,v),z(u,v))\frac{D(z,x)}{D(u,v)}\\
&\qquad\qquad\;\;+R (x(u,v),y(u,v),z(u,v))\frac{D(x,y)}{D(u,v)}\bigg)\dif u\dif v
\end{align*}
其中$\pm$由任意一点处$\v{n},S^{+}$是否同向来定。

又由混合积$\v{F}\cdot (\v{r}'_u\times \v{r}'_v)$的表达式可知
\begin{align*}
\iint_{S^{+}}\v{F}(x,y,z)\cdot \dif \v{\sigma}
&=\pm\iint_{D}\v{F}(x(u,v),y(u,v),z(u,v))\cdot \mbome{ \v{n}(u,v)}\dif u\dif v\\
&=\pm\iint_{D}\left(\v{F}\cdot\mboba{(\v{r}'_u\times \v{r}'_v)}\right)(u,v)\dif u\dif v\\
&=\pm\iint_{D}\begin{vmatrix}	
	P & Q  &  R \\
	\dfrac{\partial x}{\partial u}& \dfrac{\partial y}{\partial u} &\dfrac{\partial z}{\partial u}\\[5pt]
	\dfrac{\partial x}{\partial v}& \dfrac{\partial y}{\partial v} &\dfrac{\partial z}{\partial v}
\end{vmatrix}
(u,v)\dif u\dif v
\end{align*}
形式上,我们有
\begin{align*}
\dif y\wedge \dif z&=\pm\frac{D(y,z)}{D(u,v)}\dif u\dif v=\cos \alpha \dif \sigma,\\
\dif z\wedge \dif x&=\pm\frac{D(z,x)}{D(u,v)}\dif u\dif v=\cos \beta \dif \sigma,\\
\dif x\wedge \dif y&=\pm\frac{D(x,y)}{D(u,v)}\dif u\dif v=\cos \gamma \dif \sigma
\end{align*}
这里 $\dif u\dif v>0 $。所以
$\pm\dfrac{D(y,z)}{D(u,v)}$  与  $\cos \alpha$ ,$ \pm\dfrac{D(z,x)}{D(u,v)}$  与  $ \cos \beta$,$ \pm\dfrac{D(x,y)}{D(u,v)}$  与  $ \cos \gamma$ 的符号必须分别一致。

\paragraph{Gauss公式}  

\di[Gauss公式]{
	设$\varOmega\subset\mathbb{R}^3$为有界闭区域,
	其边界$\partial \varOmega$为分片光滑可定向曲面且以外侧为正向,而$\v{F}=(P,Q,R)\in\mathscr{C}^{(1)}(\varOmega)$,则
	\begin{equation}
		\oiint_{\partial \varOmega^{+}}\v{F}\cdot \dif \v{\sigma}
		=\iiint_{\varOmega}\left(\frac{\partial P}{\partial x}+\frac{\partial Q}{\partial y}+\frac{\partial R}{\partial z}\right)\dif x \dif y \dif z
	\end{equation} 
}

\zhu[用微分形式表述的Gauss公式]{
	因$\displaystyle\oiint_{\partial \varOmega^{+}}\v{F}\cdot \dif \v{\sigma}=\oiint_{\partial \varOmega^{+}}P\dif y\wedge \dif z+Q\dif z\wedge \dif x+R\dif x\wedge \dif y$,于是Gauss公式也可以表述成
	\begin{equation}
		\oiint_{\partial \varOmega^{+}}P\dif y\wedge \dif z+Q\dif z\wedge \dif x
		+R\dif x\wedge \dif y
		=\iiint_{\varOmega}\left(\frac{\partial P}{\partial x}+\frac{\partial Q}{\partial y}
		+\frac{\partial R}{\partial z}\right)\dif x \dif y \dif z
	\end{equation}
	考虑引入~\tboqi{2次微分形式}
	\begin{equation*}
		\omega=P\dif y\wedge \dif z+Q\dif z\wedge \dif x+R\dif x\wedge \dif y
	\end{equation*}
	及其\tboqi{外微分}
	\begin{align*}
		\dif \omega &=\dif P\wedge \dif y\wedge \dif z+\dif Q\wedge \dif z\wedge \dif x+\dif R\wedge \dif x\wedge \dif y\\
		&=\left(\frac{\partial P}{\partial x}\dif x+\frac{\partial P}{\partial y}\dif y+\frac{\partial P}{\partial z}\dif z\right)\wedge \dif y\wedge \dif z
		+\left(\frac{\partial Q}{\partial x}\dif x+\frac{\partial Q}{\partial y}\dif y+\frac{\partial Q}{\partial z}\dif z\right)\wedge \dif z\wedge \dif x\\
		&\quad\mathop{}+\left(\frac{\partial R}{\partial x}\dif x+\frac{\partial R}{\partial y}\dif y+\frac{\partial R}{\partial z}\dif z\right)\wedge \dif x\wedge \dif y\\
		&=\left(\frac{\partial P}{\partial x}+\frac{\partial Q}{\partial y}
		+\frac{\partial R}{\partial z}\right)\dif x\wedge \dif y\wedge \dif z		
	\end{align*}
	则Gauss公式也可表述成
	\begin{equation}
		\oiint_{\partial \varOmega^{+}}\omega=\iiint_{\varOmega} \dif \omega
	\end{equation}
}

\paragraph{Stokes公式}

给定空间的一张有向曲面$S$,带有边界$\partial S$。若曲面的方向与边界曲线的方向满足:当你沿着边界曲线方向走的时候,你的头指向曲面的方向,而且与你临近的曲面在你的左侧;或者满足右手螺旋法则:右手握着该曲面,大拇指指向曲面的正向,其余四个指头就指向边界曲线的正向;则称曲面及其边界的定向\textbf{协调}。

\di[Stokes公式]{
	假设$\varOmega\subset\mathbb{R}^3$为非空开集,
	$S\subset \varOmega$为分片光滑可定向有界曲面,其边界$\partial S$
	为分段光滑闭曲线并且$S^{+}$与$\partial S^{+}$的定向协调,$\v{F}=(P,Q,R)\in\mathscr{C}^{(1)}(\varOmega)$,则
	\begin{equation}
	\oint_{\partial S^{+}}\v{F}\cdot \dif \v{\ell}
	=\oint_{\partial S^{+}}P\dif x+Q\dif y+R\dif z
	=\iint_{S^{+}}(\v{\nabla}\times \v{F} )\cdot \dif \v{\sigma}
	\end{equation} 
	其中$\v{\nabla}\times \v{F}$被称为向量场$\v{F}$的旋度。
}

\zhu[散度与旋度]{
	利用Nabla算子\(\v{\nabla}=\begin{pmatrix}
		\dfrac{\partial}{\partial x} & \dfrac{\partial}{\partial y} & \dfrac{\partial}{\partial z}
	\end{pmatrix}\tp\),可以定义
	\begin{definition}
		向量场\(\v{F} = \begin{pmatrix}
			P & Q & R
		\end{pmatrix}\tp\)的\tboqi{散度}为
		\renewcommand{\div}{\mathop{\mathrm{div}}}
		\begin{equation}
			\div \v{F} = \v{\nabla} \cdot \v{F} = \dfrac{\partial P}{\partial x} + \dfrac{\partial Q}{\partial y} + \dfrac{\partial R}{\partial z}
		\end{equation}
	\end{definition}
	\begin{definition}
		向量场\(\v{F} = \begin{pmatrix}
			P & Q & R
		\end{pmatrix}\tp\)的\tboqi{旋度}为
		\newcommand{\rot}{\mathop{\mathrm{rot}}}
		\begin{equation}
			\rot \v{F} = \v{\nabla} \times \v{F} = \begin{vmatrix}
				\hat{\boldsymbol{x}} & \hat{\boldsymbol{y}} & \hat{\boldsymbol{z}} \\
				\dfrac{\partial}{\partial x} & \dfrac{\partial}{\partial y} & \dfrac{\partial}{\partial z} \\
				P & Q & R
			\end{vmatrix} = \begin{pmatrix}
				\dfrac{\partial R}{\partial y} - \dfrac{\partial Q}{\partial z} \\[7pt]
				\dfrac{\partial P}{\partial z} - \dfrac{\partial R}{\partial x} \\[7pt]
				\dfrac{\partial Q}{\partial x} - \dfrac{\partial P}{\partial y}
			\end{pmatrix}
		\end{equation}
	\end{definition}
}

\subsection{含参积分}

\subsubsection{运算次序可交换性}\label{Sec: 运算次序可交换性}

\begin{theorem}
	如果$f:[a,b]\times [c,d]\rightarrow\mathbb{R}$为连续函数,
	则$\forall (x_0,y_0)\in [a,b]\times [c,d]$,均有
	\begin{equation}
		\lim_{y\rightarrow y_0}\lim_{x\rightarrow x_0}f(x,y)=f(x_0,y_0)=\lim_{x\rightarrow x_0}\lim_{y\rightarrow y_0}f(x,y)
	\end{equation}
\end{theorem}

\begin{definition}
	\textup{\textbf{含参变量积分\quad}}假设$f:[a,b]\times [c,d]\rightarrow\mathbb{R}$为函数,如果
	$\forall y\in [c,d]$,积分
	\(I(y)=\dint_a^bf(x,y)\dif x\)
	均有定义,则我们将之称为(以$y$为参变量的)\textbf{含参变量积分}。
\end{definition}

\di[极限与积分次序可交换性]{
	如果$f:[a,b]\times [c,d]\rightarrow\mathbb{R}$为连续函数,
	则$I(y)=\dint_a^bf(x,y)\dif x: [c,d]\rightarrow\mathbb{R}$也为连续函数,即有 
	\begin{equation}
		\lim\limits_{y\rightarrow y_0}\int_a^bf(x,y)\dif x
		=\int_a^b\lim\limits_{y\rightarrow y_0}f(x,y)\dif x
	\end{equation}
}

\di[求导与积分次序可交换性]{
	如果$f:[a,b]\times [c,d]\rightarrow\mathbb{R}$ 为连续函数
	使得偏导函数$\dfrac{\partial f}{\partial y}$在$[a,b]\times [c,d]$上存在且连续,
	则$I: [c,d]\rightarrow\mathbb{R}$连续可导且
	\begin{equation}
		I'(y)=\frac{\dif }{\dif y}\int_a^bf(x,y)\dif x=\int_a^b\frac{\partial f}{\partial y}(x,y)\dif x
	\end{equation}
}

\di[变限积分]{
	假设$f:[a,b]\times [c,d]\rightarrow\mathbb{R}$ 为连续函数
	使得偏导函数$\dfrac{\partial f}{\partial y}$在$[a,b]\times [c,d]$上存在且连续,
	而$\alpha,\beta:[c,d]\rightarrow [a,b]$可导。 $\forall y\in [c,d]$,定义
	\(J(y)=\dint_{\alpha(y)}^{\beta(y)}f(x,y)\dif x\),
	则$J: [c,d]\rightarrow\mathbb{R}$为可导函数且
	\begin{equation}
	J'(y)=\int_{\alpha(y)}^{\beta(y)}\frac{\partial f}{\partial y}(x,y)\dif x
	+f(\beta(y),y)\beta'(y)-f(\alpha(y),y)\alpha'(y)
	\end{equation}
}

\di[积分与积分次序可交换性]{
	若$f:[a,b]\times [c,d]\rightarrow\mathbb{R}$连续,则
	\begin{equation}
	\int_c^d\left(\int_a^bf(x,y)\dif x\right)\dif y=\int_a^b\left(\int_c^df(x,y)\dif y\right)\dif x
	\end{equation}
}
\begin{proofs}
	\adjline
	$\forall x\in [a,b]$以及$\forall t\in [c,d]$,定义
	\begin{equation*}
	F(x,t)=\int_c^tf(x,y)\dif y,\qquad
	g(t)=\int_a^bF(x,t)\dif x=\int_a^b\left(\int_c^tf(x,y)\dif y\right)\dif x
	\end{equation*}
	则$F:[a,b]\times[c,d]\rightarrow\mathbb{R}$连续且$\dfrac{\partial F}{\partial t}(x,t)=f(x,t)$,
	于是$\dfrac{\partial F}{\partial t}$为连续。再由求导与积分次序可交换性
	可知函数$g$连续可导且我们有
	\begin{equation*}
	g'(t)=\int_a^b\frac{\partial F}{\partial t}(x,t)\dif x=\int_a^bf(x,t)\dif x
	\end{equation*}
	由此我们立刻可得
	\begin{equation*}
	\int_c^d\left(\int_a^bf(x,y)\dif x\right)\dif y=\int_c^dg'(y)\dif y
	=g(d)-g(c)
	=\int_a^b\left(\int_c^df(x,y)\dif y\right)\dif x \qedhere
	\end{equation*}
\end{proofs}

\begin{example}
	$\forall \theta\in (-1,1)$,定义\(I(\theta)=\dint_0^{\pi}\ln(1+\theta\cos x)\dif x\),求\(I(\theta)\)。
	\begin{solution}
		由题设条件以及求导与积分次序可交换性可知,$I$为连续可导且$\forall \theta\in (-1,1)\setminus\{0\}$,均有
		\begin{equation*}
			I'(\theta)=\int_0^{\pi}\frac{\cos x}{1+\theta\cos x}\dif x
			=\int_0^{\pi}\frac{1}{\theta}\left(1-\frac{1}{1+\theta\cos x}\right)\dif x
			=\frac{\pi}{\theta}-\frac{1}{\theta}\int_0^{\pi}\frac{\dif x}{1+\theta\cos x}			
		\end{equation*}
		做变量替换,有 
		\begin{align*}
			\int_0^{\pi}\frac{\dif x}{1+\theta\cos x}
			&\xlongequal{t=\tan\frac{x}{2}}
			\int_0^{+\infty}\frac{\dif \left(2\arctan t\right)}{1+\theta\frac{1-t^2}{1+t^2}}
			=\int_0^{+\infty}\frac{1+t^2}{1+t^2+\theta(1-t^2)}\cdot \frac{2}{1+t^2}\dif t\\
			&=\int_0^{+\infty}\frac{2\dif t}{(1+\theta)+(1-\theta)t^2}
			=\frac{2}{\sqrt{1-\theta^2}}\int_0^{+\infty}\frac{\dif \sqrt{\frac{1-\theta}{1+\theta}}t}{1+\left(\sqrt{\frac{1-\theta}{1+\theta}}t\right)^2}\\
			&=\frac{2}{\sqrt{1-\theta^2}}\arctan\sqrt{\frac{1-\theta}{1+\theta}}t\Big|_0^{+\infty}
			=\frac{\pi}{\sqrt{1-\theta^2}}
		\end{align*}
		由此我们立刻可得
		\begin{equation*}
		I'(\theta)=\frac{\pi}{\theta}-\frac{\pi}{\theta}\cdot\frac{1}{\sqrt{1-\theta^2}}
		=\frac{\pi}{\theta}\cdot\frac{\sqrt{1-\theta^2}-1}{\sqrt{1-\theta^2}}
		=\frac{-\theta \pi}{(\sqrt{1-\theta^2}+1)\sqrt{1-\theta^2}}
		\end{equation*}
		注意到$I(0)=0$,故$\forall \theta\in (-1,1)$,我们有
		\begin{align*}
			I(\theta)&=\int_0^{\theta}I'(t)\dif t
			=\int_0^{\theta}\frac{-t\pi\dif t}{(\sqrt{1-t^2}+1)\sqrt{1-t^2}}=\int_0^{\theta}\frac{\pi\dif (\sqrt{1-t^2})}{\sqrt{1-t^2}+1}\\
			&=\pi\ln(\sqrt{1-t^2}+1)\Big|_0^{\theta}
			=\pi\ln\frac{\sqrt{1-\theta^2}+1}{2} \qedhere
		\end{align*}
	\end{solution}
\end{example}
\begin{example}
	计算$\displaystyle I=\int_0^1\frac{x^b-x^a}{\ln x}\dif x$($a,b>0$)。 

	\begin{pf}[breakable, enhanced jigsaw]\adjline
		\begin{proof}[\small\textit{\yan 解法一}]\small\kai
			\renewcommand{\qedsymbol}{$\circledS$}
			由积分与积分次序可交换性可知
			\begin{align*}
			I&=\int_0^1\frac{x^b-x^a}{\ln x}\dif x
			=\int_0^1\left(\int_a^bx^y\dif y\right)\dif x
			=\int_a^b\left(\int_0^1 x^y\dif x\right)\dif y\\
			&=\int_a^b\left(\frac{x^{y+1}}{y+1}\bigg|_0^1\right)\dif y
			=\int_a^b\frac{\dif y}{y+1}
			=\ln(y+1)\Big|_a^b=\ln\frac{b+1}{a+1} \qedhere
			\end{align*}
		\end{proof}
		\begin{proof}[\small\textit{\yan 解法二}]\small\kai
			\renewcommand{\qedsymbol}{$\circledS$}
			固定$a>0$,$\forall b>0$,定义
			\(I(b)=\dint_0^1\frac{x^b-x^a}{\log x}\dif x\),
			则$I(a)=0$且由求导与积分次序可交换性得
			\begin{equation*}
			I'(b)=\int_0^1\frac{\partial }{\partial b}\left(\frac{x^b-x^a}{\log x}\right)\dif x
			=\int_0^1 x^b\dif x
			=\frac{1}{b+1}
			\end{equation*}
			由此立刻可得
			\begin{equation*}
			I(b)=\int_a^bI'(t)\dif t
			=\int_a^b\frac{\dif t}{t+1}
			=\log\frac{b+1}{a+1} \qedhere
			\end{equation*}
		\end{proof}
	\end{pf}
\end{example}

\Subsubsection{广义含参积分}

\de[广义含参积分的收敛]{
	假设$f:[a,\omega)\times [c,d]\rightarrow\mathbb{R}$为\textbf{连续}函数,
	其中$\omega\in \mathbb{R}\cup \{+\infty\}$。若$y_0\in [c,d]$使广义积分
	\begin{equation*}
	\int_a^{\omega}f(x,y_0)\dif x=\lim_{A\rightarrow \omega^{-}}\int_a^Af(x,y_0)\dif x
	\end{equation*}
	收敛,则称广义含参积分$\dint_a^{\omega}f(x,y)\dif x$在
	点$y_0$处\tboba{收敛},否则则称之在该点\tboba{发散}。

	如果广义含参变量积分$\dint_a^{\omega}f(x,y)\dif x$在$y \in [c,d]$的
	每点均收敛,我们则称之在$[c,d]$上\tboba{收敛},由此
	得到$[c,d]$上的函数$I(y)=\dint_a^{\omega}f(x,y)\dif x$。
}

广义含参变量积分$\dint_a^{\omega}f(x,y)\dif x$在$[c,d]$上
收敛到函数$I(y)$,当且仅当
\begin{center}
	$\forall y\in[c,d]$,$\forall \varepsilon>0$,
	$\exists M\in [a,\omega)$,使得$\forall A\in [M,\omega)$,
	\(\left|\dint_a^{A}f(x,y)\dif x-I(y)\right|<\varepsilon\)
\end{center}
由Cauchy判别准则,广义含参变量积分
$\dint_a^{\omega}f(x,y)\dif x$在$[c,d]$上收敛,
当且仅当$\forall y\in [c,d]$,$\forall \varepsilon>0$,$\exists M\in[a,\omega)$,使得$\forall A',A''\in [M,\omega)$,
\begin{equation*}
\left|\int_{A'}^{A''}f(x,y)\dif x\right|
=\left|\int_a^{A''}f(x,y)\dif x-\int_a^{A'}f(x,y)\dif x\right|<\varepsilon
\end{equation*}

\de[广义含参积分的一致收敛]{
	若$\forall \varepsilon>0$,$\exists M\in [a,\omega)$,使得$\forall A\in [M,\omega)$,$\forall y\in [c,d]$,均有
	\(\left|\dint_a^{A}f(x,y)\dif x-I(y)\right|<\varepsilon\),
	则我们称广义含参积分$\dint_a^{\omega}f(x,y)\dif x$在区间
	$[c,d]$上\tboba{一致收敛}到函数$I(y)$。
}

由Cauchy判别准则,广义含参变量积分
$\dint_a^{\omega} f(x,y)\dif x$在$[c,d]$上一致收敛,
当且仅当$\forall \varepsilon>0$,$\exists M\in [a,\omega)$,使得
$\forall A',A''\in[M,\omega)$,
$\forall y\in[c,d]$,
$\left|\dint_{A'}^{A''}f(x,y)\dif x\right|<\varepsilon$。


\di[广义含参积分的Weierstrass判别法]{
	假设$f:[a,\omega)\times [c,d]\rightarrow\mathbb{R}$为连续函数,而函数
	$F:[a,\omega)\rightarrow[0,+\infty)$使得$\forall (x,y)\in [a,\omega)\times [c,d]$,
	均有$|f(x,y)|\le F(x)$。
	若$\dint_a^{\omega}F(x)\dif x$收敛,
	则$\dint_a^{\omega}f(x,y)\dif x$关于$y\in [c,d]$一致收敛。	
}
\begin{proofs}
	因$\dint_a^{\omega}F(x)\dif x$收敛,则由Cauchy准则知,
	$\forall \varepsilon>0$,$\exists M\in[a,\omega)$
	使得$\forall A',A''\in[M,\omega)$,均有
	$\left|\dint_{A'}^{A''}F(x)\dif x\right|<\varepsilon$。
	则$\forall y\in [c,d]$,我们有
	$$\left|\dint_{A'}^{A''}f(x,y)\dif x\right|\le \left|\int_{A'}^{A''}|f(x,y)|\dif x\right|
	\le \left|\int_{A'}^{A''}F(x)\dif x\right|<\varepsilon$$
	从而由Cauchy判别准则可知所证结论成立。
\end{proofs}

\di[广义含参积分的Abel和Dirichlet判别法]{
	设$f,g: [a,\omega)\times [c,d]\rightarrow \mathbb{R}$为函数,使得
	$\forall y\in [c,d]$,$f(x,y),g(x,y)$在$x\in[a,\omega)$的任意的闭子区间上均可积。那么,$\dint_a^{\omega}f(x,y)g(x,y)\dif x$关于$y\in [c,d]$一致收敛,如果

	\tbome{Abel\quad}$\dint_a^{\omega}f(x,y)\dif x$关于$y\in [c,d]$一致收敛,$g(x,y)$有界且关于\(x\)单调。

	\hang[2]\tbome{Dirichlet\quad} $\forall y\in [c,d]$以及$\forall A\in [a,\omega)$,
	$F(A,y)=\dint_a^Af(x,y)\dif x$有界,
	$g(x,y)$关于\(x\)单调且$\lim\limits_{x\rightarrow \omega^{-}}g(x,y)=0$关于$y\in [c,d]$一致成立。
}

\begin{example}[1]
	求证:广义含参变量积
	分$\dint_1^{+\infty}\frac{\sin (tx)}{x}\dif x$关于
	$t\in[c,+\infty)$一致收敛,其中$c>0$。
	\begin{proofs}
		$\forall (x,t)\in[1,+\infty)\times[c,+\infty)$,
		定义函数
		$f(x,t)=\sin (tx)$,$g(x,t)=\dfrac{1}{x}$,
		那么$g$关于$x$单调,
		且$\lim\limits_{x\rightarrow+\infty}g(x,t)=0$关于$t\in[c,+\infty)$一致成立。
		又$\forall A>1$,$\left|\dint_1^A\sin (tx) \dif x\right|=
		\dfrac{1}{t}|\cos t-\cos (At)|\le\dfrac{2}{c}$,
		则由Dirichlet判别准则可知$\dint_1^{+\infty}\frac{\sin (tx)}{x}\dif x$关于
		$t\in[c,+\infty)$一致收敛。
	\end{proofs}
\end{example}

广义含参积分也具有~\ref{Sec: 运算次序可交换性}~节所述性质,即有
\begin{theorem}
	\textup{\textbf{广义含参积分的分析性质\quad}}设$f:[a,\omega)\times [c,d]\rightarrow \mathbb{R}$为连续函数。
	\begin{itemize}[leftmargin=2em]
		\item \textup{\textbf{极限与积分可交换性\quad}}
		若广义含参变量积分
		\(I(y)=\dint_a^{\omega}f(x,y)\dif x\)
		关于$y\in [c,d]$一致收敛,则$I$在$[c,d]$上连续。
		\item \textup{\textbf{求导与积分可交换性\quad}}
		若$I(y)=\dint_a^{\omega}f(x,y)\dif x$在区间$[c,d]$上收敛,
		偏导函数$\dfrac{\partial f}{\partial y}$在$[a,\omega)\times [c,d]$上连续并且使得
		广义含参积分$\dint_a^{\omega}\frac{\partial f}{\partial y}(x,y)\dif x$关于$y\in [c,d]$为
		一致收敛,则$I$在$[c,d]$上连续可导,且
		\(I'(y)=\dint_a^{\omega}\frac{\partial f}{\partial y}(x,y)\dif x\)。
		\item \textup{\textbf{积分与积分可交换性\quad}}
		若$I(y)=\dint_a^{\omega}f(x,y)\dif x$
		关于$y\in[c,d]$一致收敛,
		则$I$在$[c,d]$上可积,且
		\(\dint_c^d\Big(\int_a^{\omega}f(x,y)\dif x\Big)\dif y=\int_a^{\omega}\Big(\int_c^d f(x,y)\dif y\Big)\dif x\)。
	\end{itemize}
\end{theorem}

\subsubsection{Gamma 函数 和 Beta 函数}

\de[Gamma函数]{
	Gamma函数定义为\(\varGamma(s) = \dint_0^{+\infty} x^{s-1}\e^{-x} \dif x\)。
}
\begin{example}[1]
	讨论Gamma函数的收敛域。
	\begin{solution}
		当$s<1$时,$x=0$是瑕点,但它又是无穷积分。把它拆成两部分,考虑
		$$\int_{0}^{+\infty}x^{s-1}\mathrm{e}^{-x}\mathrm{d}x=\int_{0}^{1}x^{s-1}\mathrm{e}^{-x}\mathrm{d}x+\int_{1}^{+\infty}x^{s-1}\mathrm{e}^{-x}\mathrm{d}x$$
		当$x\to0$时,
		$x^{s-1}\mathrm{e}^{-x} \to x^{s-1}$,
		所以第一个积分当$s>0$时收敛;当$x\to+\infty$时,
		$x^2\cdot x^{s-1}\mathrm{e}^{-x}\to0$,
		故对充分大的$x$;恒有
		$x^{s-1}\mathrm{e}^{-x}<\dfrac1{x^2}$,
		所以第二个积分不论$s$为何值时都收敛。因而原积分当$s>0$时收敛。
	\end{solution}
\end{example}

\begin{theorem}
	\textup{Gamma}函数具有以下性质:

	(1)\(\varGamma \in \mathscr{C}^{(\infty)}(0,+\infty)\);

	(2)\(\forall s>0,\varGamma(s)>0\),且\(\varGamma(1)=1\);

	(3)\(\forall s>0,\varGamma(s+1) = s\varGamma(s)\);

	(4)\(\ln \varGamma(s)\)是\((0,+\infty)\)上的凸函数。
\end{theorem}

\di[Bohr-Mollerup 定理]{
	如果一个定义在\((0,+\infty)\)上的函数\(f(x)\)有以下性质:

	(1)\(\forall x>0,f(x)>0\),且\(f(1)=1\);

	(2)\(\forall x>0,f(x+1) = xf(x)\);

	(3)\(\ln f(x)\)是\((0,+\infty)\)上的凸函数,

	那么\(f(x) = \varGamma(x) = \lim\limits_{n \to \infty} \dfrac{n^x n!}{x(x+1)\cdots(x+n)}\)。
}

\de[Beta函数]{
	Beta函数定义为\(B(p,q) = \dint_0^1 x^{p-1}(1-x)^{q-1} \dif x\)。
}
\begin{example}
	讨论Beta函数的收敛域。
	\begin{solution}
		当$p<1$时,$x=0$是瑕点;$q<1$时,$x=1$是瑕点。为了分别考虑函
		数在这两点附近的情况,把积分拆成两部分:
		$$\int_{0}^{1}x^{p-1}(1-x)^{q-1}\mathrm{d}x=\int_{0}^{a}x^{p-1}(1-x)^{q-1}\mathrm{d}x\\+\int_{a}^{1}x^{p-1}(1-x)^{q-1}\mathrm{d}x$$
		其中$a\in(0,1)。$当$x\to0$时
		$x^{p-1}(1-x)^{q-1}\to x^{p-1}$,
		故当$p>0$时,第一个积分收敛。当$x\to1$时,
		$x^{p-1}(1-x)^{q-1}\to (1-x)^{q-1}$,
		第二个积分当$q>0$时收敛。因此原积分在$p>0,q>0$时收敛。
	\end{solution}
\end{example}

\di[Gamma函数与Beta函数的关系]{
	\(\forall p,q>0\),
	\begin{equation}
		B(p,q) = \dfrac{\varGamma(p)\varGamma(q)}{\varGamma(p+q)}
	\end{equation}
}
\begin{proofs}
	可证\(f(p) = \dfrac{B(p,q)\varGamma(p+q)}{\varGamma(q)}\)满足定理~\ref{Bohr-Mollerup 定理}的三条性质。
\end{proofs}

\begin{theorem}
	\textup{Beta}函数具有以下性质:

	(1)\(B(p,q) \in \mathscr{C}^{(\infty)}(0,+\infty)^2\);

	(2)\(\forall p,q>0,B(p,q) = B(q,p)\);

	(3)\(\forall p,q>0,B(p+1,q+1) = \dfrac{pq}{(p+q+1)(p+q)}B(q,p)\)。
\end{theorem}
%----------------------------------------------------------
\end{document}
